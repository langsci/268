\newcommand{\appref}[1]{Appendix \ref{#1}}
\newcommand{\fnref}[1]{Footnote \ref{#1}}

\newenvironment{langscibars}{\begin{axis}[ybar,xtick=data, xticklabels from table={\mydata}{pos},
        width  = \textwidth,
	height = .3\textheight,
    	nodes near coords,
	xtick=data,
	x tick label style={},
	ymin=0,
	cycle list name=langscicolors
        ]}{\end{axis}}

\newcommand{\langscibar}[1]{\addplot+ table [x=i, y=#1] {\mydata};\addlegendentry{#1};}

\newcommand{\langscidata}[1]{\pgfplotstableread{#1}\mydata;}

\renewcommand{\lsChapterFooterSize}{\footnotesize}

\makeatletter
\let\thetitle\@title
\let\theauthor\@author
\makeatother

\newcommand{\togglepaper}[1][0]{
%   \bibliography{../localbibliography}
  \papernote{\scriptsize\normalfont
    \theauthor.
    \thetitle.
    To appear in:
    Change Volume Editor \& in localcommands.tex
    Change volume title in localcommands.tex
    Berlin: Language Science Press. [preliminary page numbering]
  }
  \pagenumbering{roman}
  \setcounter{chapter}{#1}
  \addtocounter{chapter}{-1}
  \WarningsOff[microtype]
    \WarningFilter{microtype}{Unknown slot}
    \WarningFilter{scrbook}{package incompatibility}
  }

\usetikzlibrary{shapes.arrows,shadows}

% \tikzfading[name=arrowfading, top color=transparent!0, bottom color=transparent!95]
% \tikzset{arrowfill/.style={top color=white, bottom color=white}}
\tikzset{arrowstyle/.style={draw=black, thick, single arrow, minimum height=#1, single arrow,
single arrow head extend=.1cm}}

\newcommand{\tikzfancyarrow}[1][1cm]{\tikz[baseline=-0.5ex]\node [arrowstyle=#1]{} ;}

\providecommand{\LastRefSteppedCounter}{}

\DeclareNewSectionCommand
  [
    counterwithin = chapter,
    afterskip = 2.3ex plus .2ex,
    beforeskip = -3.5ex plus -1ex minus -.2ex,
    indent = 0pt,
    font = \usekomafont{section},
    level = 1,
    tocindent = 1.5em,
    toclevel = 1,
    tocnumwidth = 2.3em,
    tocstyle = section,
    style = section
  ]
  {appendixsection}

\renewcommand*\theappendixsection{\Alph{appendixsection}}
\renewcommand*{\appendixsectionformat}{\appendixname~\theappendixsection\autodot\enskip}
\renewcommand*{\appendixsectionmarkformat}{\appendixname~\theappendixsection\autodot\enskip}

\DeclareNewSectionCommand
  [
    counterwithin = appendixsection,
    afterskip = 1.5ex plus .2ex,
    beforeskip = -3.25ex plus -1ex minus -.2ex,
    indent = 0pt,
    font = \usekomafont{subsection},
    level = 2,
    tocindent = 3.8em,
    toclevel = 2,
    tocnumwidth = 3.2em,
    tocstyle = section,
    style = section
  ]
  {appendixsubsection}

\pgfplotscreateplotcyclelist{langsci}{
   RdYlBu-A,every mark/.append style={fill=RdYlBu-A!80!black},mark=*\\
   RdYlBu-F!60!black,every mark/.append style={fill=RdYlBu-F!80!black},mark=otimes*\\
   RdYlBu-L,every mark/.append style={fill=RdYlBu-L!80!black},mark=diamond*\\
}
