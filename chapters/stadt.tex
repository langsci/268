\documentclass[output=paper,modfonts,nonflat, newtxmath]{langsci/langscibook}
\author{Rosalinde Stadt\affiliation{University of Amsterdam}\and Aafke Hulk\affiliation{University of Amsterdam}\lastand Petra Sleeman\affiliation{University of Amsterdam}}
\title{{L1} {Dutch} {vs} {L2} {English} {and} {in} {the} {initial} {stages} {of} {L3} {French} {acquisition} }
\abstract{With this paper, we attempt to define the role of background languages in third language (L3) acquisition in the classroom by focusing on the influence of first language (L1) Dutch and second language (L2) English verb placement in L3 French amongst Dutch secondary school pupils (aged 11–13) who are in the initial stages of L3 acquisition of French (\textit{N} = 23). To detect possible transfer from Dutch, we count errors based on V2 surface structures in sentences containing a sentence-initial adverb, and in order to detect transfer from English, we count errors based on the Adv-V word order in the middle field of the clause. We collected data from a grammaticality judgement task to account for receptive knowledge and a gap-filling task to measure learners’ guided production. We found a considerable amount of transfer from the L1 in the initial stages, in both the grammaticality judgement task and the gap-filling task.\\
\textbf{Keywords} L3 acquisition, initial stages, L1 transfer, L2 transfer, syntax, verb placement, L1 Dutch, L2 English, L3 French.
}
\IfFileExists{../localcommands.tex}{
  % add all extra packages you need to load to this file
\usepackage{tabularx}
\usepackage{url}
\urlstyle{same}

\usepackage{listings}
\lstset{basicstyle=\ttfamily,tabsize=2,breaklines=true}


%%%%%%%%%%%%%%%%%%%%%%%%%%%%%%%%%%%%%%%%%%%%%%%%%%%%
%%%                                              %%%
%%%           Examples                           %%%
%%%                                              %%%
%%%%%%%%%%%%%%%%%%%%%%%%%%%%%%%%%%%%%%%%%%%%%%%%%%%%
%% to add additional information to the right of examples, uncomment the following line
% \usepackage{jambox}
%% if you want the source line of examples to be in italics, uncomment the following line
% \renewcommand{\exfont}{\itshape}
\usepackage{langsci-optional}
%\usepackage{langsci-optional}
\usepackage{langsci-gb4e}
% \usepackage{langsci-lgr}
\makeatletter
\let\pgfmathModX=\pgfmathMod@
\usepackage{pgfplots,pgfplotstable}%
\let\pgfmathMod@=\pgfmathModX
\makeatother
\usepgfplotslibrary{colorbrewer}
\usetikzlibrary{fit}
%\usetikzlibrary{positioning}

\definecolor{lsDOIGray}{cmyk}{0,0,0,0.45}

\usepackage{xassoccnt}
\newcounter{realpage}
\DeclareAssociatedCounters{page}{realpage}
\AtBeginDocument, free-standing-units, input-open-uncertainty= , input-close-uncertainty= ,table-align-text-pre=false,uncertainty-separator={\,},group-digits=false,detect-inline-weight=math}
\DeclareSIUnit[number-unit-product={}]{\percent}{\%}
\makeatletter \def\new@fontshape{} \makeatother
\robustify\bfseries % For detect weight to work

  \newcommand{\appref}[1]{Appendix \ref{#1}}
\newcommand{\fnref}[1]{Footnote \ref{#1}}

\newenvironment{langscibars}{\begin{axis}[ybar,xtick=data, xticklabels from table={\mydata}{pos},
        width  = \textwidth,
	height = .3\textheight,
    	nodes near coords,
	xtick=data,
	x tick label style={},
	ymin=0,
	cycle list name=langscicolors
        ]}{\end{axis}}

\newcommand{\langscibar}[1]{\addplot+ table [x=i, y=#1] {\mydata};\addlegendentry{#1};}

\newcommand{\langscidata}[1]{\pgfplotstableread{#1}\mydata;}

\makeatletter
\let\thetitle\@title
\let\theauthor\@author
\makeatother

\newcommand{\togglepaper}[1][0]{
%   \bibliography{../localbibliography}
  \papernote{\scriptsize\normalfont
    \theauthor.
    \thetitle.
    To appear in:
    Change Volume Editor \& in localcommands.tex
    Change volume title in localcommands.tex
    Berlin: Language Science Press. [preliminary page numbering]
  }
  \pagenumbering{roman}
  \setcounter{chapter}{#1}
  \addtocounter{chapter}{-1}
  }

\usetikzlibrary{shapes.arrows,shadows}

% \tikzfading[name=arrowfading, top color=transparent!0, bottom color=transparent!95]
% \tikzset{arrowfill/.style={top color=white, bottom color=white}}
\tikzset{arrowstyle/.style={draw=black, thick, single arrow, minimum height=#1, single arrow,
single arrow head extend=.1cm}}

\newcommand{\tikzfancyarrow}[1][1cm]{\tikz[baseline=-0.5ex]\node [arrowstyle=#1]{} ;}

  %\input{../locahyphenation}
  \togglepaper[1]%%chapternumber
  \addbibresource{../localbibliography.bib}
}{}

\begin{document}
\maketitle


\section{Introduction}%1
\label{sec:stadt:1}

In today’s Dutch secondary school curriculum, foreign language learning plays an important role. English is mandatory throughout the whole duration of the secondary school programme and two other foreign languages (mostly French, German or Spanish) besides English are mandatory in the first three years of secondary education. In this paper, we investigate to what extent the background languages – in this case, first language (L1) Dutch and second language (L2) English – play a role in third language (L3) acquisition of French among secondary school pupils.\footnote{ \textrm{In this paper we use the notions ‘L3 acquisition’ and ‘L3 learning’ interchangeably, although we acknowledge, with \citet[fn 1]{BardelFalk2012}, that the situation that we describe is strictly speaking ‘learning’.}} We especially concentrate on pupils who have just entered secondary school and who are therefore in the initial stages of learning French, but who already started learning English in primary school.

  In previous studies (\citealt{StadtEtAl2016, StadtEtAl2018Exposure}) conducted in the same secondary school as in this study, we tested the L2 status factor hypothesis – according to which the L2 takes on a stronger role than the L1 in L3 acquisition (L3A) – on four groups of Dutch secondary school pupils (L2 English, L3 French): third- and fourth-year pupils who had either been enrolled in an L2 English bilingual stream programme or in a mainstream Dutch secondary school curriculum (\citealt{StadtEtAl2016, StadtEtAl2018Exposure}). We only found partial support for the L2 status factor \citep{BardelFalk2007, FalkBardel2011}, as explained in more detail in \sectref{sec:stadt:2.3}. In both bilingual groups, where pupils had received more daily L2 exposure than the mainstream groups, and in the fourth-year mainstream group, where pupils had received more L2 exposure than the third-year mainstream group, we found significantly more possible transfer from English than from Dutch (supporting the L2 status factor hypothesis). However, in the third-year mainstream group, where pupils had been less exposed to English, the L1 and L2 were equally important sources of transfer. Hence, the smaller amount of L2 exposure the pupils received in the daily context and through the years affected L2 transfer in L3A (see \citealt{Hammarberg2001, Hammarberg2009, Tremblay2006}).

  The effect of L2 English exposure on L3 French learning that we found in our previous research led us to an interest in investigating pupils in the initial stages of L3A to learn more about the extent to which the L1 and the L2 play different roles in pupils who have just entered the bilingual stream education and are not yet exposed to the L2 in the daily school context. Therefore, the general aim of this study is to investigate L1/L2 transfer in L3 French in first-year pupils (aged 11–13). These pupils were in their second week of secondary school at the time of testing and thus in the initial stages of L3 French learning.\footnote{ \textrm{In L3A literature, initial state and initial stages are both used to indicate learners at the onset of L3A. Since the first-year pupils relevant to this paper are in their second week of learning (L3) French (having received some L3 vocabulary input) and based on the distinction made by \citet{GarciaMayoRothman2012}, we use the term ‘initial stages’ to indicate the developmental stage these L3 French learners are in.}} The research question relevant to this paper is: How does L1 Dutch and L2 English word order affect L3 French learning in the initial stages of acquisition? We gathered data from 23 first-year pupils enrolled in the same Dutch secondary school as the third- and fourth-year pupils we studied in \citet{StadtEtAl2016, StadtEtAl2018Exposure}. They had not yet started their L2 English bilingual stream education, and as a consequence, they were not yet surrounded by English in their daily school practice as were their third- and fourth-year fellow pupils. However, let us stress that the first-year pupils were not at the onset of learning English. Although the quality and quantity of teaching and therefore proficiency differs \citep{UnsworthEtAl2015}, most primary schools in the Netherlands offer English as a subject from at least the penultimate year of primary school (ages 10–11) \citep{Rose2016}.\footnote{66\% of the Dutch primary schools offer English in the last two years of school (when pupils are aged 10–12 years) A small number of schools (17\%) offer English before that (from the age of 8–9) or, also in 17\% of the cases, even earlier, from the age of 4–7 \citep{ThijsEtAl2011}.} To investigate the linguistic behaviour of the first-year pupils and to be able to compare the results to previous results, we use the same constructions as in our previous studies: To detect possible influence from Dutch on French, we look at errors based on XVS(O) (V2) word order, in sentences containing a sentence-initial adverb: \textit{Vandaag} {\textit{eet}} \textit{Jan} \textit{een} \textit{appel} *‘Today {eats} John an apple’, \textit{*Aujourd’hui} {\textit{mange}} \textit{Jean} \textit{une} \textit{pomme}, and to detect transfer from English into French, we look at errors based on Adv-V word order in sentences containing a manner/frequency adverb or a floating quantifier: ‘John often {eats} an apple’ \textit{*Jean} \textit{souvent} {\textit{mange}} \textit{une} \textit{pomme}. We collected data from a grammaticality judgement task (GJT) to account for receptive knowºledge and a gap-filling task to measure the learners’ guided production.

  The paper is structured as follows. In \sectref{sec:stadt:2}, we 1) discuss some studies in the L3A field of research that also focus on the initial stages of L3A, 2) review some studies on verb placement, and 3) give a short overview of the background for this study, that is, a recap of the third- and fourth-year results published in \citet{StadtEtAl2016, StadtEtAl2018Exposure}. We set out our design in \sectref{sec:stadt:3}, and in \sectref{sec:stadt:4} we report the results. In \sectref{sec:stadt:5}, we discuss our results, and in \sectref{sec:stadt:6} we present some concluding remarks.

\section{Theoretical background}%2
\label{sec:stadt:2}
\subsection{ L3 learning in the initial stages of acquisition}%2.1.
\label{sec:stadt:2.1}

 \citet[15]{GarciaMayoRothman2012}, following \citet{SchwartzSprouse1996}, define the initial state of L3 acquisition as “the set of linguistic hypotheses with which the learner begins the acquisition process”. L3 acquisition resembles L2 acquisition with respect to linguistic knowledge of (an)other language(s) in advance of L2/L3 input. However, it differs in the sense that the learner has already learned a foreign language, which makes him or her a more ‘advanced’ language learner (\citealt{CenozValencia1994, Jessner2006}) and the learner has two systems available instead of one to make predictions about the L3. \citet[fn. 12]{GarciaMayoRothman2012} notice that in L3 acquisition models the initial state has also been defined as the ‘initial stages’ of the acquisition process, which they consider a more liberal definition than ‘initial state’, which includes the actual initial state as intended by \citet{SchwartzSprouse1996}. Various studies have been conducted on the initial stages of acquisition, finding both transfer from the L1 and the L2. In what follows, we briefly discuss some studies that proposed L1 or L2 transfer in the initial stages and which are therefore relevant to this study.

According to the Full Access Full Transfer Hypothesis (\citealt{SchwartzSprouse1996}), the L1 grammar is fully transferred to the L2 initial state grammar. Although the hypothesis was originally formulated for L2 acquisition, it could be extended to L3 learning. The Cumulative Enhancement Model \citep{FlynnEtAl2004} claims that transfer from both the L1 and the L2 can occur at the initial stages of acquisition, as long as it is facilitative. According to the Typological Primacy Model (\citealt{Rothman2010, Rothman2011, Rothman2015}), perceived typological resemblance with L3 determines transfer from either L1 or L2.

Evidence of L1 transfer has been found in studies such as \citep{Hermas2014Morphosyntax, Hermas2014Relatives}, who found preferred transfer from the L1 (Arabic) rather than from the L2 (French) in L3 English in studies on the acquisition of the null subject parameter \citep{Hermas2014Morphosyntax} and of restrictive relative clauses \citep{Hermas2014Relatives}. Similarly, \citet{FalkEtAl2015} found transfer from both the L1 and the L2 depending on the degree of metalinguistic knowledge (MLK) in the L1. They investigated the oral production of adjectives in 40 participants who were in the initial stages of learning L3 French and who had different degrees of explicit MLK of the L1. They found L1 transfer in learners with a high MLK in the L1, whereas learners with a low MLK (incorrectly) transferred language related information from the L2.

In turn, evidence of L2 transfer has been found in, for example, \citet{FalkBardel2011}. They tested 44 L3 German intermediate learners with either L1 French-L2 English (\textit{n} = 22) or L1 English-L2 French (\textit{n} = 22) on the acquisition of object-pronoun placement in main clauses (‘je {le} vois’, ‘I see {him}’, ‘Ich sehe {ihn’} – L1/L2 French ${\neq}$ L1/L2 English = L3 German) and subordinate clauses (‘You know that I see {him}’, ‘Tu sais que je {le} vois’, ‘Du weisst dass ich {ihn} sehe’– L1/L2 English ${\neq}$ L1/L2 French = L3 German) \citep[60]{FalkBardel2011}. Data were gathered with a grammaticality judgement and correction task. The results demonstrated that learners were more influenced by the L2 than by the L1. Further evidence of L2 transfer was obtained in various studies by \citet{Sánchez2012, Sánchez2015, Sánchez2016} that investigated the occurrence of word order transfer (OV/VO orders) in the L3 English of bilingual Spanish/Catalan learners with prior knowledge of L2 German. The findings showed a systematic transfer from the L2 German in the initial stages by learners of the same age as those investigated in this study.

Other empirical evidence for L1 and L2 transfer in the initial stages is presented in \sectref{sec:stadt:2.2}, in which we focus specifically on transfer of the XVS(O) and the Adv-V word orders.

\subsection{{Studies} {on} {verb} {placement} }%2.2.
\label{sec:stadt:2.2}

In what follows, we discuss a few studies that concentrated particularly on the acquisition of the two word order constructions that are under examination in this study: Adv-V word order (+English, $-$French) and XVS(O) word order (+Dutch, $-$French).

\subsubsection{{Adv-V} {word} {order}}%2.2.1
\label{sec:stadt:2.2.1}

Verb placement has already been the object of investigation in L3A, and various studies have more specifically focused on the Adv-V word order in the middle field of the clause. In English, manner and frequency adverbs and floating quantifiers appear pre-verbally, such as in a clause like ‘Manon sometimes goes to the zoo’. In the field of L3A, \citet{WestergaardEtAl2016} found empirical evidence for the Linguistic Primacy Model – according to which facilitative or non-facilitative property-by-property transfer from one or both previously acquired languages occurs in L3A when a linguistic property in the L3 input reveals abstract structural similarity with linguistic properties of the previously learned languages – by investigating the acquisition of Adv-V word order in English (as an L2 and an L3). They studied Adv-V word order in monolingual Norwegian and monolingual Russian learners learning English as an L2 and Norwegian-Russian bilinguals learning English as an L3. Russian and English share the Adv-V word order, whereas Norwegian has the V-Adv word order in the same type of sentences. In that study, in which data were gathered by means of a GJT, they found that Russian monolinguals and Norwegian-Russian bilinguals could benefit from the Russian Adv-V word order, resulting in a more target-like use in English (compared to the Norwegian monolinguals).

  \citet{Falk2010} found evidence for the L2 status factor hypothesis in intermediate learners. She studied (amongst other things) the influence of English Adv-V word order in declarative main clauses in L3 German by means of a grammaticality judgement and a correction task. She investigated 60 learners of German (L3): one group of 30 L1 English/L2 French learners and one group of 30 L1 French/L2 English learners. French and German share a V-Adv word order, whereas English has an Adv-V word order (‘Il mange souvent du chocolat’, ‘Er isst oft Schokolade’, *He eats often chocolate, ‘He often eats chocolate’). Thus for the L1 English–L2 French–L3 German group, the L2 and L3 shared the same word order, whereas it was the L1 and the L3 that showed similarity in this structure for the L1 French–L2 English–L3 German group. The results of this study showed that the L2 was the default transfer source in the acquisition of an L3, regardless of whether the influence was facilitative or not.

  \citet{Hermas2010}, on the other hand, found empirical evidence for L1 transfer in a study on the initial stages of L3 learning. He investigated the acquisition of Adv-V word order in L3 English by means of an acceptability judgement test and a preference test. He tested 20 native speakers of Arabic in the initial stages of L3 English (L2 French). In Arabic, both the Adv-V and V-Adv word order are used, whereas French (V-Adv) differs from English (Adv-V) in this respect. The results of this study showed that the L3 learners of English, who had Arabic as an L1 and French as an L2, reached a high rate of accuracy in judging the Adv-V word order in L3 English, which \citet{Hermas2010} interpreted as facilitative transfer from the Adv-V Arabic (L1) word order. On the basis of the data, Hermas argued that transfer from L1 can also be non-facilitative.

\subsubsection{{XVS(O)} {word} {order}}

Most Germanic languages are so-called V2 languages (e.g. German, Norwegian, Swedish, Danish, and Dutch). In these languages, the finite verb occupies the second position of the declarative main clause, including after a sentence initial adverb, resulting in Dutch sentences such as the following:

\ea%1
\label{ex:stadt:1}
\gll Vandaag \textbf{gaat} John naar Parijs.\\
today goes John to Paris\\
\glt ‘Today John goes to Paris.’
\z


In what follows, we will discuss the results of two studies that demonstrated that a –V2 language (as an L2) negatively influenced the acquisition of the V2 property in the (+V2) L3. Although various studies in the field of second language acquisition claim that the V2 rule (XVS(O)) is difficult to learn since Subject-Verb-Object (SVO) is the canonical word order \citep{KleinPerdue1997, Pienemann1998, WahlstromMcKay2001, Bohnacker2006}, \citet{Bohnacker2006} found V2 placement in two groups of Swedish (+V2) beginning learners of German (+V2), which she attributed to positive transfer of the V2 rule. In Bohnacker’s study, one group was learning German as an L2, and the other as an L3 after English (–V2). The oral production data indicated that at least for the L2 German group the acquisition of the V2 rule was not at all complicated (100\% target-like after four months of study). The other group, however, experienced more difficulties, which was attributed by the researcher to the negative influence of the L2 (English).

\citet{BardelFalk2007} studied verb placement in various groups of L1 and L2 backgrounds in the initial stages of L3 Swedish/L3 Dutch (both +V2 languages). In the first group, the L1 was a +V2 language (Dutch when the L3 was Swedish or Swedish when the L3 was Dutch) and the L2 was a –V2 language (English). In the second group, the L1 was a –V2 language (English, Italian, Hungarian or Albanian) and the L2 a +V2 language (Dutch in the case of L3 Swedish or Swedish in the case of L3 Dutch). The results of this study demonstrated both positive and negative transfer of the V2 property in L3 Swedish/L3 Dutch, but only when the L2 was a +V2 language. The participants in the L1 +V2 group showed less transfer of the V2 rule in L3 Swedish or L3 Dutch. \citet[480]{BardelFalk2007} concluded that “in L3 learning, the L2 acts like a filter, making the L1 inaccessible”.

\subsection{{Verb} {placement} {in} {Dutch} {secondary} {school} {pupils:} {The} {background} {of} {this} {study}}%2.3.
\label{sec:stadt:2.3}

In two previous studies (\citealt{StadtEtAl2016, StadtEtAl2018Exposure}), we studied verb placement amongst four groups of third- and fourth-year secondary school pupils who were either enrolled in an international Middle Years Programme (MYP) (a Dutch-English bilingual stream programme) or a mainstream Dutch curriculum. The MYP is a bilingual stream programme of the International Baccalaureate (IBO 2019). In this programme, pupils receive over 50\% of their subjects in English, whereas in the Dutch curriculum, pupils receive three hours a week of English as a school subject. \tabref{tab:stadt:1} presents a systematic description of the background of the pupils and the distribution across groups.

%%please move \begin{table} just above
\begin{table}
\caption{ Description of the background of the pupils}
\label{tab:stadt:1}

\begin{tabularx}{.85\textwidth}{p{3.8cm} Y Y Y}
\lsptoprule
Pupils &  Number of \newline pupils & Age & Instruction \newline in English\\
	\midrule
3rd year bilingual &  16 & 13-15 & 1,942 h\\
3rd year mainstream &  11 & 13-15 & 236 h\\
4th year bilingual &   12 & 14-16 & 2,572 h\\
4th year mainstream &  11 & 14-16 & 315 h\\
\lspbottomrule
\end{tabularx}
\end{table}

\footnotetext{ \textrm{A school year consists of} \textrm{approximately 35 weeks per school year and a class takes 45 minutes. The 3rd bilingual stream pupils have received 1,627 hours of instruction in English at the end of Y3 and 1,863 hours of instruction in English at the end of Y4: 17 classes per week (446 hours) in Y1, 21 classes per week (551 hours) in Y2, 24 classes per week (630 hours) in Y3 and 9 classes per week (236 hours) in Y4. Additionally, the 3rd year bilingual stream pupils also receive four 45-minute classes of English as a subject in Y1, Y2 and Y3 (315 hours by the end of Y3) and three 45-minute classes of English as a subject in Y4 (394 hours by the end of Y4). The regular stream pupils only receive three hours of English as a subject (236 hours by the end of Y3 and 315 hours by the end of Y4).} }

In these previous studies, we compared the extent to which the third- and fourth-year pupils made errors based on the V2 rule (+ Dutch, –English, –French) versus the Adv-V word order (–Dutch, +English, –French) by means of a GJT.\footnote{ \textrm{We only looked at negative transfer, i.e. errors, since it is difficult to distinguish positive transfer from L3 knowledge, i.e. the command of a structure in the L3. Neither construction had explicitly been part of the French curriculum.}} The results of these studies are summarised in \tabref{tab:stadt:2} and \tabref{tab:stadt:3}. The results demonstrated that the third-year bilingual stream pupils made significantly more errors based on the L2 English Adv-V word order than on the L1 Dutch XVS(O) word order in French. In both fourth-year groups, this was also the case. However, the mainstream third-year pupils were influenced to the same degree by Dutch as by English

%%please move \begin{table} just above \begin{tabular
\begin{table}
\caption{Adv-V and V2 errors in third-year bilingual stream and mainstream group}
\label{tab:stadt:2}
\footnotesize
\begin{tabularx}{\textwidth}{l @{\hskip 0.03cm} r Y p{0.5cm} p{0.5cm} p{0.5cm} p{0.5cm} p{0.0001cm} p{0.5cm}p{0.3cm}p{0.3cm} p{0.3cm} r}
\lsptoprule
	 &  &  & \multicolumn{4}{c}{Adv-V errors} & &  \multicolumn{4}{c}{V2 errors} & \\
	 \cmidrule{4-7}  \cmidrule{9-12}
 3rd year & pupils & items & \# & \% & \textit{M} & \textit{SD} & & \# & \% & \textit{M} & \textit{SD} &  {\textit{p}-value}\\
\midrule
Bilingual & 16 & 224 (14x16) & 95/224 & 42.4 & 5.94 & 2.24 & & 55/224 &  24.6 & 3.44 & 1.79 & 0.005\\
\tablevspace
Mainstream & 11 & {154} (14x11) & 53/154 & 34.4 & 4.82 & 2.82 & & 65/154 & 37 & 5.18 & 2.23 & 0.742\\
\lspbottomrule
\end{tabularx}
%}
%%please move \begin{table} just above \begin{tabular
\end{table}


\begin{table}
\caption{Adv-V and V2 errors in fourth-year bilingual stream and mainstream group}
\label{tab:stadt:3}
\footnotesize
\begin{tabularx}{\textwidth}{l @{\hskip 0.03cm} r Y p{0.5cm} p{0.5cm} p{0.5cm} p{0.5cm} p{0.0001cm} p{0.5cm}p{0.3cm}p{0.3cm} p{0.3cm} r}
\lsptoprule
	 &  &  & \multicolumn{4}{c}{Adv-V errors} & &  \multicolumn{4}{c}{V2 errors} & \\
	\cmidrule{4-7}  \cmidrule{9-12}
 4th year & pupils & items & \# & \% & \textit{M} & \textit{SD} & & \# & \% & \textit{M} & \textit{SD} &  {\textit{p}-value}\\
\midrule
Bilingual & 12 & { 168} (14x12) & 55/168 & 32.7 & 4.58 & 1.83 & & 19/163 & 11.3 & 1.58 & 1.98 & 0.005\\
Mainstream & 11 & { 154} (14x11) & 39/154 & 25.3 & 3.55 & 1.92 & & 4/154 & 2.6 & 0.36 & 0.5 & 0.005\\
\lspbottomrule
\end{tabularx}
\end{table}

The results of the same study also showed that the third-year mainstream pupils made significantly more XVS(O) word order errors (based on the Dutch word order) than the bilingual stream pupils. In the mainstream group, 65 out of 154 errors could be traced to Dutch (37\%) and in the bilingual stream group, 55 out of 224 (24.6\%) errors were attributed to Dutch XVS(O) word order (\textit{p} = 0.033). Although the difference between mainstream and bilingual stream education particularly concerns the amount of L2 exposure (and L2 use) in the school context, the role of L1 Dutch differs considerably across groups. We suggested in \citet{StadtEtAl2016, StadtEtAl2018Exposure} that the relatively stronger role of the L1 in the third-year mainstream group could be indirectly due to less L2 exposure in this particular group. In other words, these results show that the mainstream school environment affects the extent to which the L1 is suppressed by the L2 and that the special L2 status does not come into play when the L2 is not sufficiently activated to suppress the L1 (cf. \citealt{Hammarberg2001}).

  However, in the fourth-year mainstream group, where Dutch is also more present in the daily school context, pupils barely made any XVS(O) errors. We interpreted this as evidence that the learners had ‘unlearned’ the V2 rule.\footnote{ \textrm{For a more detailed overview of the results and the discussion, see \citet{StadtEtAl2016, StadtEtAl2018Exposure}.}} The slight decrease in the role of the L2 might be due to the fact that the pupils had an increased overall L3 proficiency. Nevertheless, the role of the L2 remained statistically stable across both years and types of education,\footnote{ \textrm{Although we found a tendency, the} \textrm{fourth-year} \textrm{bilingual stream pupils did not make significantly fewer Adv-V errors than the third-year bilingual stream pupils (year 4: 55/168 = 32.7\% and year 3: 95/224 = 42.4\%;} \textrm{\textit{p} }\textrm{= 0.099) and the fourth-year mainstream pupils did not make significantly fewer Adv-V errors than the third-year mainstream pupils (year 4: 39/154 = 25.3\% and year 3: 53/154 = 34.4\%;} \textrm{\textit{p} }\textrm{= 0.272).}} whereas acceptance of the Dutch word order decreased.

\section{{The} {present} {study} {and} {research} {question}}%3
\label{sec:stadt:3}

Taking into consideration the results presented above, it is relevant to study to what extent L1 and L2 transfer occur in first-year secondary school pupils who have not yet started their bilingual L1 Dutch/L2 English education and who are at the onset of French learning, i.e. in the very first stages of L3 acquisition. We address the following research question: How does the L1 Dutch and L2 English word order affect L3 French learning in the initial stages of acquisition?

\subsection{{Hypothesis} }%3.1.
\label{sec:stadt:3.1}

We hypothesise that the first-year pupils will show more XVS(O) errors (based on the Dutch word order) than Adv-V errors (based on the English word order) in the initial stages of acquisition. This hypothesis is based on our previous study \citep{StadtEtAl2016} in which we found that the mainstream third-year pupils – who were receiving less L2 exposure compared to the third-year bilingual stream – made significantly more errors based on L1 Dutch than the third-year bilingual stream pupils and on \citep{Hermas2010, Hermas2014Morphosyntax, Hermas2014Relatives}, who found that the L1 plays a strong role at the onset of L3 acquisition. This hypothesis is also motivated by a theory such as the Full Access Full Transfer Hypothesis (\citealt{SchwartzSprouse1996}), see \sectref{sec:stadt:2.1}

  In the next section, we will concentrate on the background of the participants and set out the design.

\subsection{{Participants}}
\label{sec:stadt:3.2}

We tested 118 first-year pupils, but we had to exclude the vast majority of the pupils because they did not meet all criteria. One criterion was that pupils had to be aware of the Adv-V word order in English. To transfer the Adv-V word order into the L3, it was necessary that the pupils be sufficiently familiar with this particular word order in English. We tested this by means of an English gap-filling task, more details of which are given in section \sectref{sec:stadt:3.3.2}. 43 (out of 118) pupils passed the English gap-filling task. Their overall proficiency was assessed using the standardised online Anglia placement test to make sure that all pupils had at least the Anglia ‘elementary’ level in English. The Anglia test showed that all pupils had at least the required ‘elementary’ level (most of them had even a higher level), so we did not exclude any pupil on the basis of this test. The reason for having at least the ‘elementary’ level may be that most primary school pupils start learning English in the 7th grade (aged 9–11) according to the Dutch curriculum. Furthermore, English is quite ubiquitous in the Netherlands. Therefore, youngsters generally receive a great deal of English input because of media such as music, films and the Internet (\citealt{VerspoorEtAl2007, VerspoorEtAl2010}), which may also explain why all pupils had at least the ‘elementary level’ in the Anglia test.

\subsection{{Instruments} }%3.3.
\label{sec:stadt:3.3}
\subsubsection{{Linguistic} {tasks} }%3.3.1
\label{sec:stadt:3.3.1}

The data collection for this study consisted of two linguistic tasks: a GJT and a guided production gap-filling task (GFT). The tasks were constructed in a way that should make them easy to perform and vocabulary training took place in the first week before the students participated in the actual linguistic tasks (see \sectref{sec:stadt:3.4} for more detailed information on the vocabulary training). Simple sentences with many cognates known in all three languages (such as \textit{pizza}, \textit{chocolate}, \textit{film}, \textit{banana}, \textit{adore}, \textit{visit}) were used. Vocabulary items were repeated in several sentences in order to reduce the learning task. Instead of nouns, proper names familiar in the three languages were used, if possible. Since pupils had been trained on vocabulary during the preparation, they were familiar with all the vocabulary used in the tasks. In this way we tried to make sure that in all cases the pupils were able to understand the sentences and to judge their grammaticality. The test sentences were presented in the same order to all pupils. Note that the first-year tests were simplified versions of the third- and fourth-year tests. Nevertheless, the types of sentences were similar.

The GJT consisted of 32 items: seven items testing the Adv-V word order of which four items were ungrammatical and three items were grammatical; seven items tested the V2 rule of which four items were ungrammatical and three items were grammatical and 18 fillers. Two example items are provided in \REF{ex:stadt:1} and \REF{ex:stadt:2}.

  \ea%1
  \label{ex:stadt:2}
  \gll {Manon} {aime} {vraiment} {les} {biscuits.}    c / i\\
  		Manon loves really     the biscuits\\
  \glt ‘Manon really loves biscuits.’
  \z

  \ea%3
  \label{ex:stadt:3}
  \gll  {Aujourd’hui} {mange} {Jean} {une} {pomme.}  c / i   c / i\\
		  today eats Jean an apple\\
  \glt ‘Today Jean is eating an apple.’
  \z



The GFT contained 24 items: eight testing the Adv-V word order (reflecting the English word order) and eight testing the V2 rule (reflecting the Dutch word order). The test contained 8 fillers. Two example items are provided in \REF{ex:stadt:3} and \REF{ex:stadt:4}.


  \ea%4
  \label{ex:stadt:4}
  \gll  {Les} {enfants} {........} tous {........} un téléphone utilisent  c / i\\
  	the children {}             all {}           a telephone use\\
  \glt ‘The children all use a telephone.’
  \z


  \ea%5
  \label{ex:stadt:5}
  \gll  {Aujourd’hui} {........} {Jean} {........} {au} {cinéma.} {va} \\
 		 today {}          Jean {}           to.the cinema goes\\
  \glt ‘Today Jean is going to the movies.’
  \z


\subsubsection{{Other} {instruments} }
\label{sec:stadt:3.3.2}

To test whether the pupils were familiar with the English Adv-V word order, we used an English gap-filling task, see \sectref{sec:stadt:3.2}. This English gap-filling task contained 24 items of which eight controlled for Adv-V order. As mentioned in \sectref{sec:stadt:3.2}, we also used a standardised online Anglia placement test, to make sure that all participating pupils had at least the Anglia ‘elementary’ level.

The pupils also filled out a language background questionnaire. The background questionnaire was a standardised questionnaire taken from the University of Amsterdam that we used in adapted form for our study on third- and fourth-year pupils and that we adapted again for this study so that it was suitable for the first-year pupils. Based on this questionnaire, we excluded all simultaneous bilinguals (most of which were English-Dutch and Arabic-Dutch bilinguals) and put aside all pupils who had lived abroad or who had a French language background in their immediate family, to make sure that the learning conditions for L1 Dutch, L2 English and L3 French were the same for all pupils. Of the 43 pupils who showed familiarity with the Adv-V order on the English gap-filling task (see section 3.2), 20 additional pupils were excluded on these grounds. We used the data of the remaining 23 pupils.

  In the next section, we first present the procedure, followed by the results of the experiments described above. Although we will mainly focus on the new first-year results in the following section, we close the section by also comparing the results to the results of the third- and fourth-year pupils (see \sectref{sec:stadt:2.3}) in order to get a clearer idea of the cross-sectional developmental pattern with respect to the influence of the background languages.

\subsection{{Procedure}}
\label{sec:stadt:3.4}

At the time of data collection, the first-year pupils (aged 11–13) were in the second week of secondary school because in the first school week, the students needed time to prepare for taking the test. Three 45-minute periods were dedicated to the study of the basic French vocabulary (verbs, nouns) that the pupils needed to be able to participate in the French test, of which many vocabulary items were cognates. The pupils also studied the necessary vocabulary as a homework assignment. The pupils were also allowed to use their vocabulary list during the test. Since it was their first week at school and their first encounter with French, we aimed at making this first week as playful as possible. For instance, the pupils had to create a colour card on the computer, filling in the French colour terms, and to name animals, although this was not needed for the linguistic tasks. Besides the vocabulary items needed for the linguistic task, they also learned some small clarifying chunks that we used in the tests, such as \textit{À} \textit{Paris,} ‘In Paris’ or \textit{C’est} \textit{un,} ‘It is a’. However, it was crucial that the pupils did not receive any L3 grammar instruction or L3 input of whole sentences to avoid feeding them any relevant information about French syntax.\footnote{ \textrm{The first author (who is also a French teacher in this school) designed the classes for the first week.} }

The L3 French linguistic tasks were followed by the L2 English proficiency tests. As mentioned in \sectref{sec:stadt:3.2}, the English gap-filling task contained 8 sentences that controlled for the knowledge of the English Adv-V order. All pupils with more than three errors were excluded from the test.\footnote{ \textrm{The accuracy minimum of 5/8 was used to make sure that the pupils would have a minimal knowledge of the Adv-V word order to transfer into the L3.} } As also mentioned in \sectref{sec:stadt:3.2}, the standardised online Anglia placement test\footnote{\url{https://www.anglia.org}} was used to make sure that all pupils had at least the Anglia ‘elementary level’ in English. When pupils do not have an elementary level, the online test indicates a level lower than elementary (preliminary, primary, junior or first step).

All in all, five L3 French first-year classes participated in the study, with which two 45-minute sessions were scheduled within the same week in order to collect the data. The L3 French linguistic tasks (i.e. the grammaticality judgement task and the gap-filling task), the L2 English proficiency tests and the background questionnaire were completed during these two 45-minute sessions hours. Since it was important to start with the French test (so as not to bias the pupils with respect to English influence before taking the French test), the learners filled out the L3 French linguistic tasks first, followed by the L2 English proficiency tests and finally the background questionnaire. To complete the GJT, pupils had to indicate for each sentence, whether they judged the sentence correct (c) or incorrect (i). In the GFT, we asked the pupils to fill in the verb in the correct gap.

\subsection{{Data} {analysis} }
\label{sec:stadt:3.5}

In the GJT, possible transfer from the L2 English Adv-V word order was detected by coding answers as incorrect when the pupils marked ungrammatical sentences such as \textit{*Manon} \textit{vraiment} \textit{aime} \textit{les} \textit{biscuits} as correct and grammatical sentences such as \textit{Manon} \textit{aime} \textit{vraiment} \textit{les} \textit{biscuits} (example 1) as incorrect. Possible transfer from the L1 Dutch XVS(O) word order was investigated by marking answers as incorrect when pupils wrongly marked sentences such as \textit{*Aujourd’hui} \textit{mange} \textit{Jean} \textit{une} \textit{pomme} (example 2) as correct or grammatical sentences such as \textit{Aujourd’hui} \textit{Jean} \textit{mange} \textit{une} \textit{pomme} as incorrect. In the GFT, if the pupils placed the verb in the first gap in example \REF{ex:stadt:3} or in the second gap in example \REF{ex:stadt:4} we coded the answer as correct and if the pupils placed the verb in the second gap in example \REF{ex:stadt:3} or in the first gap in example \REF{ex:stadt:4} we coded the answer as incorrect.

To make a valid statistical comparison, we aimed at minimising the differences between the test pairs in both the GJT and the GFT. Therefore, we kept the variances between the items, such as vocabulary and number of words, as minimal as possible. We tested all data for normality of distribution with a Shapiro-Wilk test. Except for the XSV(O)/XVS(O) construction in the GJT, the Shapiro-Wilk test was below 0.05, and thus almost all data deviated significantly from a normal distribution. For this reason, and because the number of pupils was low, we used the non-parametric 2 related samples test to make the results more reliable.

\section{Results}
\label{sec:stadt:4}

In this section, we give an overview of the data. In \tabref{tab:stadt:4}, we present the Adv-V word order errors and V2 errors from both the receptive knowledge task (GJT) and the guided production task (GFT). In the GJT, the percentages are calculated out of a total of 161 (seven items per condition x 23 pupils), both for the Adv-V construction and the XVS(O) construction. In 33.5\% of the cases, the pupils accepted ungrammatical Adv-V sentences based on the acceptable English word order, such as *\textit{Jean} \textit{souvent} \textit{mange} \textit{une} \textit{pomme}, or rejected grammatical V-Adv sentences, such as \textit{Jean} \textit{mange} \textit{souvent} \textit{une} \textit{pomme.} For the sentences examining the potential transfer of the Dutch V2 rule, these pupils misjudged 64.6\% of the relevant items, accepting sentences such as *\textit{Aujourd’hui} \textit{mange} \textit{Jean} \textit{une} \textit{pomme} or rejecting correct sentences such as \textit{Aujourd’hui} \textit{Jean} \textit{mange} \textit{une} \textit{pomme.} In the production task (GFT), the percentages were calculated out of a total of 184 (eight items per condition x 23 pupils). We found the same tendency as in the GJT in both the Adv-V word order construction and the XVS(O) construction: In 72.3\% of the cases concerning the V2 rule, the pupils filled in the wrong gap, creating sentences such as *\textit{Aujourd’hui} \textit{mange} \textit{Jean} \textit{une} \textit{pomme} (Dutch word order), whereas in only 10.9\% of the items targeting the Adv-V construction did they use the ungrammatical English Adv-V word order *\textit{Jean} \textit{souvent} \textit{mange} \textit{une} \textit{pomme} (English word order). In both tests, the difference between the number of errors based on the XVS(O) word order and those on the Adv-V word order is statistically significant. In the GJT (Z = -3.05, \textit{p} = 0.002, \textit{r} = 0.64) and in the GFT (Z = -4.06, \textit{p} < 0.001, \textit{r} = 0.85). Note that the effect sizes of both the GJT and the GFT are large.\footnote{ \textrm{In a previous study \citep{StadtEtAl2018Longitudinal}, we calculated the correlation between the first-year pupils’ proficiency in English (using the standardised Meara vocabulary size test [\citealt{Meara2010}]) and the number of Adv-V errors. We found no significant correlation (}\textrm{\textit{p}} \textrm{= 0.663).}}

%%please move \begin{table} just above \begin{tabular
\begin{table}
\caption{Adv-V and V2 errors in GJT and GFT in first-year pupils}
\label{tab:stadt:4}
\footnotesize
\begin{tabularx}{0.93\textwidth}{p{1.3cm} r p{0.5cm} p{0.5cm} p{0.5cm} p{0.5cm} p{0.0001cm} p{0.7cm }p{0.3cm} p{0.3cm} p{0.3cm} r}
	\lsptoprule
&  &   \multicolumn{4}{c}{Adv-V errors} & &  \multicolumn{4}{c}{V2 errors} & \\
First-year & items & \# & \% & \textit{M} & \textit{SD}&  & \# & \% & \textit{M} & \textit{SD} & \textit{p}-value\\
\cmidrule{3-6}  \cmidrule{8-11}
GJT & 161 & 54/161 & 33.5 & 2.35 & 1.82 & & 104/161 & 64.6 & 4.52 & 1.44 & 0.002\\
GFT & 184 & 20/184 & 10.9\footnotemark{} & 0.87 & 1.52 & & 133/184 & 72.3 & 5.78 & 1.91 & <0.001\\
\lspbottomrule
\end{tabularx}
\end{table}
\footnotetext{ \textrm{In the results of Adv-V errors in the GFT,} \textrm{two pupils} \textrm{fell below two standard deviations of the mean.} \textrm{Without these outliers, the mean falls to 0.48 and the SD falls to 0.75. There are no consequences for the statistical analysis.}}

These results are visually presented in a diagram (\figref{fig:stadt:1}). What catches the eye is the considerable amount of transfer of the L1 Dutch V2 word order into L3 French at the initial stages of acquisition. Let us emphasise that although the first-year pupils make significantly more mistakes based on the Dutch word order than on the English word order, they still make a considerable number of Adv-V errors in the GJT. We will come back to these observations in the discussion.



\begin{figure}
\centering
\begin{tikzpicture}
 \begin{axis}[
%/pgf/number format/1000 sep={},
%width=3.8in,
%height=1.8in,
%at={(0.758in,0.981in)},
% scale only axis,
% clip=false,
separate axis lines,
axis on top,
%major grid style={draw=white},
xmin=0, xmax=6,
xtick={1,2,4,5},
x tick style={draw=none},
xticklabels={Y1 GJT Adv-V, Y1 GJT V2, Y1 GFT Adv-V, Y1 GFT V2},
x tick label style={rotate=30,anchor=east, font = \small, inner sep = 5pt},
%nodes near coords,
%nodes near coords align={vertical},
% enlargelimits=0.05,
ytick={0,25, 50, 75, 100},
ymin=0, ymax=100,
ylabel={Percentage \%},
every axis plot/.append style={
	ybar,
	bar width=0.5cm,
	bar shift=0pt,
	fill
},
	%bar width=5cm,
	%ymajorgrids,% tick align=inside,
	%major grid style={draw=white},
]
\addplot[lsLightBlue!80!black,
fill=lsLightBlue,
nodes near coords,
nodes near coords align={vertical},]
coordinates {(1,33.5)};
\addplot[lsDarkBlue!80!black,
fill=lsDarkBlue,
nodes near coords,
nodes near coords align={vertical},]
coordinates{(2,64.6)};
\addplot[lsLightBlue!80!black,
fill=lsLightBlue,
nodes near coords,
nodes near coords align={vertical},]
coordinates{(4,10.9)};
\addplot[lsDarkBlue!80!black,
fill=lsDarkBlue,
nodes near coords,
nodes near coords align={vertical},]
coordinates{(5,72.3)};
%significance stars
\addplot[black, sharp plot]%
coordinates {(1,78) (2,78)}%
node[above] at (150,77) {**}%
;%
%significance stars 2
\addplot[black, sharp plot]%
coordinates {(4,85) (5,85)}%
node[above] at (450,83) {***}; %

\end{axis}
\end{tikzpicture}
\caption{\label{fig:stadt:1}: Results of Adv-V errors and V2 errors in GJT and GFT for first-year pupils}
\end{figure}



\section{{Discussion} }%5
\label{sec:stadt:5}

\subsection{{Discussion} {of} {the} {results}}%5.1.
\label{sec:stadt:5.1}

In this study, we investigated L1 Dutch and L2 English transfer in secondary school pupils in the initial stages of L3 French acquisition. We aimed at investigating the extent to which L1 and L2 transfer occurs in this specific group of learners. To this end, we investigated to what extent the L1 Dutch and L2 English word order affect L3 French learning in the initial stages of acquisition. We hypothesised that in the initial stages of L3 French acquisition, pupils would make more errors based on the Dutch word order than on the English word order. Our hypothesis is based on \REF{ex:stadt:1} our previous studies (\citealt{StadtEtAl2016, StadtEtAl2018Exposure}) in which we found that third-year mainstream pupils – who are less exposed to English in the daily school context than the third-year bilingual stream pupils – use significantly more often their L1, on \REF{ex:stadt:2} \citet{Hermas2010, Hermas2014Morphosyntax, Hermas2014Relatives}, who found that the L1 plays a strong role at the onset of L3 learning, and \REF{ex:stadt:3} on an adaptation to L3 acquisition of \citegen{SchwartzSprouse1996} Full Access Full Transfer Hypothesis.

  To test transfer at the initial stages of acquisition, we examined to what extent first-year pupils accepted and produced the V2 property (from L1 Dutch) and Adv-V word order (from L2 English) in French. We found support for our hypothesis; that is, in the initial stages of acquisition, pupils do transfer the L1 to a high degree into the L3, which is in line with \citet{Hermas2010, Hermas2014Morphosyntax, Hermas2014Relatives}. Both in judgement (GJT) and in guided production (GFT), first-year pupils applied the L1 Dutch V2 rule in French (and possibly Dutch V-Adv word order, as well). The high amount of L1 transfer compared to the relatively low degree of L2 transfer in the initial stages could be due to the fact that the pupils had not yet received any L3 morphosyntactic input at the time of testing. Therefore, they were~not able to make assumptions about word order in French in comparison with L1 or L2, so that they~resorted to their L1 as a default language.

Regarding the comparison between transfer from the L1 and the L2, we found that the differences between possible negative transfer from L1 Dutch and possible negative transfer from L2 English are significant in both tests (GJT \textit{p} = 0.002 and GFT \textit{p} < 0.001). This finding can be interpreted as evidence against \citet{BardelFalk2007}, who found a preferred role for the L2 in the L3A of absolute beginners, although the period of instruction in the L3 was longer in their study than in ours, which might explain the different results. However, we have to stress that in the GJT, pupils also made a considerable number of errors based on the English word order (in 33.5\% of the cases, pupils accepted the English Adv-V word order, which is ungrammatical in L3 French, or rejected the grammatical V-Adv word order). The relatively high acceptance rate in the GJT could be due to the task. The judgement task might have been difficult for the first-year pupils to complete: Although the pupils were familiar with the vocabulary used in the task, a GJT demands reading skills, focus and morphosyntactic knowledge, which might be hard for first-year pupils who are in the initial stages of acquisition and who have only just learned some words in French. In general, the pupils found the GFT a much easier test to perform.\footnote{ \textrm{After completing all the tests, we always asked the pupils some questions about the tests. We often received the feedback that the GFT was easier than the GJT. Since the two tasks have the same type of sentences, testing the same constructions, the difference in experience could be due to the type of test: judging could be more difficult than completing a sentence with a given verb.} } In the GFT, the pupils almost never placed the verb after the adverb in French, hence making hardly any Adv-V errors. It could be that the activation and/or proficiency level of the L2 is so low that pupils in this stage of acquisition consider the L1 as the only possible language to transfer information from.

A factor that may have influenced the results is the pupils’ proficiency level in English since they come from different primary school backgrounds. However, in a previous study \citep{StadtEtAl2018Longitudinal}, we calculated the correlation between the first-year pupils’ proficiency in English (using the standardised Meara vocabulary size test [\citealt{Meara2010}]) and the number of Adv-V errors. We found no significant correlation (\textit{p} = 0.663, see fn. 10). Secondly, one may wonder whether the tasks were not too difficult for the learners. We do not think that this has been the case. We tried to make the linguistic tasks as easy as possible by including proper names and many cognates, by offering vocabulary lists to learn beforehand, and by offering vocabulary lists during the tasks. A third point that may have influenced the results is that the test sentences were presented only in one order to all pupils, i.e. we did not use different versions of the tasks with different orders in which the sentences were presented. One could also wonder whether the fact that the vocabulary items were translated into Dutch may have influenced the results. However, in \citet{StadtEtAl2016, StadtEtAl2018Exposure} we also provided French-Dutch vocabulary lists, but in these studies, we saw predominantly influence from English. Future research should verify whether these points have influenced the results.

  When we compare the results of the present study to the results of \citet{StadtEtAl2016} with respect to the cross-sectional developmental pattern in the roles of L1 Dutch and L2 English in L3 French, we see an enormous decrease of XVS(O) errors from Y1 to Y3. Although the pupils had not received explicit instruction on L3 French verb placement in this case, the decrease in the number of V2 errors in later stages of L3 learning may also be due to increased L3 proficiency. Still, it is quite interesting to see that the increasing L3 proficiency has no effect on the number of L2 Adv-V errors: When we cross-sectionally compare the first-year GJT data from this study to the third-year GJT data \citep{StadtEtAl2016}, we see that the role of L2 English remains stable despite the fact that L3 proficiency increases (see \citealt{Falk2010}, who found evidence for a substantial role for the L2 in intermediate learners)\footnote{ \textrm{Note that we did find an increase of Adv-V errors from year 1 to year 3 in the guided production task \citep{StadtEtAl2018Bilingual}.}}.

  Another explanation for the changing role of the L1 could be that in the initial stages, the L1 plays the most important role and that in later stages of acquisition – once the pupils have received some L3 input and the L2 has begun to play a more prominent role in their everyday (school) lives – the L2 comes into play so that the L1 is to some extent suppressed by the L2 when the L2 is sufficiently activated. \citet{WestergaardEtAl2016} found positive transfer from L1/L2 Russian (+Adv-V word order) into L3 English (+Adv-V word order) in later stages of L3 English acquisition when learners received supporting evidence from the L1/L2. In our case, it seems possible that in later stages of L3 French acquisition, the learners received ‘misleading’ evidence from the English Adv-V word order and therefore negatively transferred the Adv-V property into L3 French.

The idea that the learner needs a certain level of development and proficiency in the L2 to transfer syntactic structures from the L2 was found earlier by \citet{BardelFalk2007}, \citet{SánchezBardel2017} and \citetv{chapters/sanchez7}. Moreover, the argumentation that the L2 suppresses the L1 when the L2 is sufficiently activated is also found in our previous study \citep{StadtEtAl2016}, in which we found that pupils with the same L3 proficiency showed a different use of the background languages in L3A: The mainstream pupils used the L1 significantly more often than the bilingual stream pupils, most likely because their curricular exposure to the L2 was much lower than in the bilingual stream pupils (in contrast to their more intense curricular exposure to the L1). Furthermore, in the comparison between L1 and L2 transfer, we found that the third-year mainstream pupils used the L1 and the L2 to the same extent, whereas the third-year bilingual stream pupils demonstrated significantly more influence from the L2 than from the L1, which suggests that learners who were not intensely exposed to the L1 at school transferred more from the L2.

For future research, it might also be interesting to look further into the interplay between an increasing L2 input/L2 proficiency and L3 proficiency (\citealt{BardelLindqvist2007, SánchezBardel2017}). It might also be relevant for a subsequent study to test L2 \textit{production} in L3A such as writing and speaking skills as to get a more complete picture of the differences between receptive knowledge and production in L3A. Although the tasks that we used in this study are important in linguistic research because they provide evidence about the grammaticality of utterances that do not normally occur in natural language production (\citealt{SchützeSprouse2014}), it would be interesting to also focus on tasks that concentrate on fluency and meaning such as written or oral narrative tests (\citealt{EllisR2005, EllisR2009}).

\subsection{{Interpretation} {of} {the} {results} {in} {light} {of} {other} {studies} }%5.2
\label{sec:stadt:5.2}
The results of this study show that the pupils transferred from the L1 in the initial stages of L3 acquisition in the constructions that we investigated. In the L3A field of research, some other studies found a strong effect from L1 in the initial stages of acquisition as well (although transfer from the L2 has also be found, as already mentioned in 2.1 and 2.2). Just as in studies conducted by, for example, \citet{NaRanongLeung2009} and \citet{Hermas2010, Hermas2014Morphosyntax, Hermas2014Relatives}, our results suggest L1 transfer in the initial stages. Subsequently, the role of the L2 becomes relatively more important. This may be due to the amount of L2 exposure, hence the degree of activation of the L2, and to awareness.
Transfer from the L1 occurred regardless of the transfer being facilitative (or neutral) (contra the Cumulative Enhancement Model \citep{FlynnEtAl2004}), and without the languages being typologically related, which is contra the Typological Primacy Model \citep{Rothman2010, Rothman2011, Rothman2015}. Moreover, transfer from the L1 occurred although the V2 rule is often recognised as difficult to transfer in the first stages of acquisition (e.g. \citealt{Pienemann1998}). In a model recently introduced – the hierarchical inference framework \citep{PajakEtAl2016} –, development plays a central role. According to this model, the L3 learner slowly changes his or her implicit beliefs about the target language that are based on prior beliefs about other background languages. This is conventionally referred to as ‘interlanguage restructuring’ and results from hypothesis testing. The hypotheses about language are adjusted as the learner receives more input from the L3. \citet{PajakEtAl2016} would explain the decrease of transfer from the L1 by stating that, because of the L3 input in French in combination with an increasing knowledge of English, the learners’ hypotheses about the \textit{usability} of the background languages are adjusted.
In the latest version of the L2 status factor hypothesis (\citealt{BardelSánchez2017}), the particular status of the L2 is related to the degree of metalinguistic knowledge, meaning the extent to which the grammar (L2 and L3) is learned explicitly and thus stored in declarative memory.\footnotemark{} This may be related to our findings on the changing roles of L1 and L2: in later stages of L3A, when pupils have received language instruction (and gained more knowledge about languages) they may use their background languages differently. Future research might further explore the effect of metalinguistic knowledge on L1/L2 transfer in L3A in secondary school pupils (see \citealt{FalkEtAl2015}).
\footnotetext{ \textrm{The L2 status factor hypothesis follows \citeapo{Paradis2004} distinction between procedural (implicit) and declarative (explicit) memory. In foreign language acquisition, information is assumed to be mainly stored in declarative memory. For a detailed discussion, see \citet{BardelSánchez2017}.} }

\section{{Conclusion} {and} {future} {directions}}
\label{sec:stadt:6}

In this study, we examined to what extent first-year secondary school pupils transfer L1 Dutch and L2 English into L3 French in the case of two verb placement constructions: the V2 rule (L1 Dutch) and Adv-V word order (L2 English). Our aim was to further define under which circumstances L1/L2 transfer occurs in the initial stages of L3 learning. The results of this study demonstrate L1 transfer at the first encounter with the L3, arguably because the pupils think, before having received any L3 morphosyntactic input, that Dutch and French share the same word order. \citeapo{Hermas2010} hypothesis on L1 (Arabic) transfer in the initial stages of L3 (English) acquisition would therefore also hold for this particular group of secondary school pupils. A preferred role for the L2 over the L1, which we found in later stages of L3 development when pupils are exposed to L2 English to a greater extent (\citealt{StadtEtAl2016, StadtEtAl2018Exposure}) is not found in this group of learners at the onset of L3 French learning. We argued that this is due to the important L1 transfer in the initial stages but also to the fact that the L2 needs to be sufficiently activated for the L2 to ‘suppress’ the L1.

  In this study, we found task variation between the GJT and the GFT regarding the Adv-V construction. It would be relevant for further work in the field to verify whether a GJT is a suitable task for (young) learners in the initial stages of L3 learning. A suggestion for future research could also be to examine other cases of syntactic transfer from L1 Dutch into L3 French to investigate whether the great amount of L1 transfer that we found in this study is typical for transfer of the V2 property or whether L1 transfer in the initial stages applies to other syntactic constructions as well. In future research, it would be interesting to also look at production data to learn more about L1 versus L2 transfer in the initial stages of acquisition in secondary school pupils (see \citealt{FalkEtAl2015}). Furthermore, it would be interesting to investigate the influence of L1 Dutch and L2 English in L3 French learning in other stages of L3 learning with a longitudinal study to learn more about the developmental patterns of L1/L2 influence in L3 French secondary school pupils.

\section{Acknowledgments}

We are very grateful to the reviewers for their valuable comments on an earlier version of this paper. Many thanks also go to the editors for their help in bringing the paper to its present form.
\sloppy\printbibliography[heading=subbibliography,notkeyword=this]
\end{document}
