\documentclass[output=paper]{../langscibook}
\author{Alonso Guerrero Galván\affiliation{}\orcid{}\lastand Nadiezdha Torres Sánchez\affiliation{}\orcid{}}
\title{El emph{habitus} lingüístico de tres redes indígenas: Otomí, chichimeca y tepehuano del sureste}
\abstract{El presente trabajo es un ejercicio metodológico para describir los espacios de uso de las dos o más lenguas que se hablan en una comunidad multilingüe a partir de la noción de mercado y habitus lingüístico. De tal suerte, se analizan los espacios de uso del español y la lengua indígena (otomí, chichimeca jonaz y tepehuano del sureste) en tres redes comunitarias. El principal interés es, por un lado, comparar tres realidades sociolingüísticas: El Espíritu, comunidad otomí en la que la lengua indígena está entrando en un estado de obsolescencia; Misión de Chichimecas en donde el español parece desplazar a la lengua indígena y Santa María de Ocotán espacio en el que la lengua indígena es ampliamente usada por los miembros de la comunidad; y, por otro lado, proponer una metodología conjunta que permita hacer dicha comparación y empezar a plantear una tipología de comunidades multilingües.
% \keywords{bilingüismo, realidad sociolingüística, mercado lingüístico, \textit{habitus} lingüístico, ámbitos de uso }
}

\IfFileExists{../localcommands.tex}{
  \addbibresource{../localbibliography.bib}
  % add all extra packages you need to load to this file
\usepackage{tabularx}
\usepackage{url}
\urlstyle{same}

\usepackage{listings}
\lstset{basicstyle=\ttfamily,tabsize=2,breaklines=true}


%%%%%%%%%%%%%%%%%%%%%%%%%%%%%%%%%%%%%%%%%%%%%%%%%%%%
%%%                                              %%%
%%%           Examples                           %%%
%%%                                              %%%
%%%%%%%%%%%%%%%%%%%%%%%%%%%%%%%%%%%%%%%%%%%%%%%%%%%%
%% to add additional information to the right of examples, uncomment the following line
% \usepackage{jambox}
%% if you want the source line of examples to be in italics, uncomment the following line
% \renewcommand{\exfont}{\itshape}
\usepackage{langsci-optional}
%\usepackage{langsci-optional}
\usepackage{langsci-gb4e}
% \usepackage{langsci-lgr}
\usepackage{pgfplots,pgfplotstable}
\usetikzlibrary{fit}
%\usetikzlibrary{positioning}

\definecolor{lsDOIGray}{cmyk}{0,0,0,0.45}

\usepackage{xassoccnt}
\newcounter{realpage}
\DeclareAssociatedCounters{page}{realpage}
\AtBeginDocument{%
  \stepcounter{realpage}
}

\usepackage{hhline}

\usepackage{rotating}

\usepackage{comment}

\usepackage[linguistics,edges]{forest}

% \usepackage{./langsci/styles/jambox}








  \input{../localcommands}
  \input{../localhyphenation}
  \togglepaper[1]%%chapternumber
}{}

\begin{document}
\maketitle 
\shorttitlerunninghead{El \textup{habitus} lingüístico de tres redes indígenas}%%use this for an abridged title in the page headers


 \section{Introducción}


La identidad lingüística es algo que se va moldeando con la práctica social del lenguaje y el uso de una o varias lenguas o dialectos, es decir, no se trata de algo fijo, pues en un contexto determinado es negociable. El hablante tiene a su disposición la elección de ciertas formas lingüísticas a las que le asigna un valor dependiendo de diversos factores tales como el interlocutor, el espacio de interacción, el estigma social que puede tener la lengua en uso, el contenido del mensaje o las implicaciones sociales que puede representar enunciarlo o no.

En este sentido, consideramos a la identidad lingüística como una construcción individual, la cual se forja a partir de la negociación dentro de la comunidad o comunidades de habla en las que se participa (\citealt{Niño-MurciaRothman2008}). Esta interacción tiene su locus en los dominios de uso que conforman un determinado \textit{habitus} lingüístico, el cual es producto de las condiciones sociales e históricas de cada comunidad \citep{Bourdieu1990}\footnote{Bloommaert, Collins y \citet[210]{Slembrouck2005} estudian las diferentes competencias multilingües a partir de un análisis crítico a las posturas de Goffman y Bourdieu y advierten que "[...] in Goffmans case, due to a lack of attention to how the social order structures and 'fomats' situations and practices; in Bordieu's case, due to an over- generalization of the case of the unified symbolic market", sugieren que "we need to address space and differences and relations between spaces as aspects of communication" pero sobre todo abogan por pensar en el espacio desde una perspectiva semiótica y escalar (2005: 212, 213). En nuestro caso, hacemos uso del concepto de mercado y \textit{habitus} lingüístico como un mecanismo que permita describir las decisiones que toma el hablante en la selección de una o más lenguas en un dominio de uso determinado (i.e. las fiestas tradicionales, la casa, las juntas, la clínica, etc.), sin profundizar en las diversas competencias comunicativas o estrategias discursivas que sigan en cada uno de estos espacios. Enfatizando las diferencias y semejanzas en la realidad sociolingüística de cada comunidad en las que se hablan dos o más lenguas.} . El presente trabajo analiza los espacios de usos del español y la lengua indígena (otomí, chichimeca jonaz y tepehuano del sureste) en tres redes comunitarias: i) El Espíritu, comunidad otomí en la que la lengua indígena está entrando en un estado de obsolescencia; ii) Misión de Chichimecas, en donde el español parece desplazar a la lengua indígena, y iii) Santa María de Ocotán, espacio en el que la lengua indígena es ampliamente usada por los miembros de la comunidad.

En resumen, nuestro principal interés es comparar tres realidades sociolingüísticas a partir de una misma metodología, de tal suerte que, para determinar los distintos dominios de uso de las lenguas, se analizan los resultados obtenidos tanto en el trabajo etnográfico, como en el empleo de un mismo cuestionario sociolingüístico (Guerrero \citealt{Galván2009}), y se implementa la noción de \textit{habitus} lingüístico para describirlos y esquematizarlos.


 \section{Justificación}


La descripción sociolingüística de las diferentes comunidades en las que se hablan dos o más lenguas, en su mayoría, se ha hecho con base en el término de diglosia, por lo tanto, es común que al describir las comunidades bilingües éstas también se describan como diglósicas y de alguna manera se homogenicen las diferentes características propias de una comunidad de habla. Sin embargo, el termino de diglosia expuesto por \citet{Ferguson1959} ha tenido un sin número de modificaciones con el fin de adecuarlo a la realidad que cada investigador observa en la comunidad bajo estudio (véase \citep{Zimmermann2010}. Esto, refleja por un lado el poco dinamismo del término, es decir, que para poder mostrar las diferencias entre las comunidades es necesario modificarlo al grado que ya no se puede relacionar con su sentido inicial; y, por otro lado, si bien dichas adecuaciones muestran la diversidad de realidades sociolingüísticas de las diferentes comunidades, no permite la comparación entre ellas en aras de tener una tipología de comunidades multilingües.

  No obstante, si se considera que cada comunidad lingüística o de práctica construye una serie de disposiciones o esquemas generadores de conductas lingüísticas o \textit{habitus} \citep[114]{Bourdieu1990}, que determina que el uso de una o más lenguas es más o menos ventajoso para determinada coyuntura situacional y discursiva, como lo afirma Alonso \citet[221]{Benito2004} Alonso \citet[6]{Benito2004}:

\begin{quote}
Los actos particulares de habla [...] no se producen como actos racionalizados, individualizados y calculadores, sino como exteriorización práctica de un \textit{habitus} que aquí es un \textit{habitus} lingüístico, definido por un conjunto relacionado de disposiciones adquiridas, esquemas de percepción y de apreciación de la realidad, así como de actuación en ella, incalculados en un contexto social y una situación histórica determinada. El \textit{habitus} es simultáneamente productor de prácticas, mediador entre las relaciones socialmente objetivas y los comportamientos individuales, producto, a su vez también, de la interiorización de las condiciones objetivas y de las estrategias de adaptación de los actores de un campo.
\end{quote}

Es posible hacer uso del concepto de \textit{habitus} para la descripción de los usos diferenciados de las lenguas habladas en una comunidad. Así, Bourdieu lo considera algo implantado en los individuos de manera preconsciente; por lo que para poder describir en profundidad los diferentes \textit{habitus} es necesario contar con un registro etnográfico que nos permita develar la lógica comunitaria, las formas de organización, los ciclos rituales y las exegesis culturales que los configuran simbólicamente. En este trabajo, para este fin, nos valemos de las notas etnografías hechas por los autores en distintas temporadas de campo en las comunidades bajo estudio, así como de las etnografías realizadas y publicadas por distintos investigadores de la región, tal y como se expone en los apartados de cada comunidad.

De esta forma, a partir de la noción de \textit{habitus}, se plantea la descripción de tres comunidades, cuya selección obedece al hecho de que, en principio, son totalmente ajenas unas a otras, pues no mantienen lazos comerciales, religiosos o parentales. Se ubican en regiones muy diferentes y han enfrentado procesos históricamente muy distintos en cuanto a su incorporación al Estado Nacional y la implantación del español y su uso en las comunidades. Además, las lenguas indígenas estudiadas pertenecen a ramas y familias distintas, de tal forma que el otomí y el chichimeca pertenecen a la familia otopame, pero el primero es de la rama central u otomiana y el segundo de la norteña o pameana, mientras que el tepehuano del sureste es de la rama pimana de la familia yutoazteca. Finalmente, y como se explica más adelante, la situación lingüística que ejemplifican estas comunidades ilustra un \textit{continuum} que va de la obsolescencia, al desplazamiento y de éste al mantenimiento de la lengua indígena, con lo que se pondrá a prueba la eficacia del \textit{habitus} como una noción explicativa de los usos de las lenguas en los distintos ámbitos, y, además, se da cuenta de los factores a los que estas comunidades se enfrentan y cómo han generado conductas lingüísticas tan distintas en algunos aspectos y tan parecidas en otros.

Es importante mencionar que, con el fin de ampliar la visión ética de la etnografía y contar con la participación directa de las comunidades (visión émica), se pensó en un instrumento que nos permitiera recolectar datos más homogéneos y que pudieran ser comparables, por lo que se eligió la “Encuesta para peritaje lingüístico” (Guerrero \citealt{Galván2009}), que se describe en el apartado de metodología. Trabajamos bajo la hipótesis de que instrumentos como esta encuesta, registrada de manera oral\footnote{Siguiendo lo expuesto por Silva-Corvalán y \citet[65]{Enrique-Arias2001} para quien “la encuesta oral es la más aconsejable porque permite al encuestador investigar más a fondo las respuestas de los informantes y no ata a estos a dos o tres posibilidades preestablecidas".}, nos permiten abordar la realidad desde una perspectiva cercana a la comunidad de estudio, arrojando datos de gran valor descriptivo y comparativo.

Así pues, nuestra justificación es meramente metodológica y documental, pues consideramos que la realidad sociolingüística de México es compleja y aún no se encuentra del todo descrita, por lo que es necesario empezar a implementar metodologías unificadas que ayuden a describir dichas realidades\footnote{Para tener una idea general de los trabajos sobre la variación en lenguas indígenas, su contacto con el español y bilingüismo en las comunidades indígenas en México véase \citet{Cifuentes1998}; Flores \citet{Farfán1998}; Guerrero y San \citet{Giacomo2014}; \citet{Hekking1995}; Hill y \citet{Hill1986}; \citet{Levy1990}; \citet{Lastra2003}; \citet{Pfeiler1988}; Smith \citet{Stark2007}; \citet{Suárez1977}; \citet{Villavicencio2006} y Zimmerman (1987; 2010), por mencionar algunos.}.


 \section{Metodología}


Los ámbitos de uso lingüísticos conforman el marco de las interacciones entre hablantes, en este sentido sintetizan los campos en que se desarrolla el \textit{habitus} lingüístico de la comunidad \citep[109]{Bourdieu1990}. Así, identificamos tres distintos campos en los que se han desarrollado \textit{habitus} relacionados con el uso de las lenguas indígenas y el español: i) la familia, se trata de un campo nuclear en donde las lenguas suelen transmitirse y los individuos comparten intereses más cercanos (consanguíneos, políticos o rituales) y también, suele haber una mayor solidaridad con el interlocutor. El ámbito donde prototípicamente se da este juego de negociaciones es la casa; ii) la comunidad, que tiene ámbitos altamente simbólicos y culturalmente determinados, como la organización ritual (fiestas) o la forma de organización interna (juntas), y otros que los ponen en contacto directo con los intereses de otras comunidades, como la iglesia, la escuela y el mercado que también pueden ubicarse dentro de iii) el campo extracomunitario, en donde entran en juego intereses municipales, regionales y estatales.

  Ahora bien, para intentar delinear cómo se estructuran estos diferentes campos y los ámbitos de uso que los comprenden, se exploran los resultados arrojados por la “Encuesta para peritaje lingüístico”, diseñada para el proyecto de Normatividad de Variación en Lenguas Otopames de la Dirección de Lingüística del INAH. Este instrumento busca establecer una metodología específica de encuesta oral para la obtención de datos sociolingüísticos en situaciones de contacto y para el estudio del cambio lingüístico. Contempla los siguientes aspectos: (a) datos sociodemográficos, (b) datos socioeconómicos, (c) adquisición y competencia lingüística, (d) uso de las lenguas y ámbitos lingüísticos, (e) actitudes y creencias, (f) identidad y cultura y (g) variación lingüística.

Para este trabajo revisamos los resultados obtenidos en el inciso (d) uso de las lenguas y ámbitos lingüísticos. Este apartado contiene dos tipos de preguntas, el primero se interesa por identificar a los interlocutores en cada una de las lenguas en relación con la familia nuclear y extensa, parentesco ritual, amigos y extraños, intentando hacer una gradación en términos de una mayor o menor solidaridad y, también, se inquiere si hay una diferencia en el uso de la lengua con respecto a los hijos, por ser mayores o menores, como se esquematiza en la siguiente tabla.

\begin{table}
\caption{\label{tab:guerrero}1. Tipo de interlocutor \citep[305]{Guerrero2016}}


\begin{tabularx}{\textwidth}{XXXXXXXX}

\lsptoprule
{¿Qué lengua habla con ...?} & \multicolumn{1}{c}{} & {(\textit{parentesco})} & {\textit{(nuclear)}} & {\textit{(extensa)}} & {\textit{(ritual)}} &  & \multicolumn{1}{c}{}\\
&  & {Familia} & {mamá} & {abuela} & {compadres} & {amigos} & {extraño}\\
&  &  & {papá} & {abuelo} & {comadres} &  & \\
&  &  & {hermanos} & {Tíos} &  &  & \\
&  &  & {hermanas} & {Tías} &  &  & \\
&  & {Pareja} & {hijos, hijas hij@ mayor} & {primas primos} &  &  & \\
&  & {Suegro} & {hij@ menor} & {nuera} &  &  & \\
&  & {Suegra} & {niet@} & {yerno} &  &  & \\
&  &  &  &  &  &  & \\
%\hhline%%replace by cmidrule{~~------}
{\textit{familiaridad} (solidaridad)} &  & {+} &  &  &  &  & {{}-}\\
%\hhline%%replace by cmidrule{~~------}
\lspbottomrule
\end{tabularx}
\end{table}

El segundo tipo de pregunta indaga sobre los ámbitos de uso de las lenguas y quiénes podrían ser los interlocutores en ellos. Se trata de explorar las relaciones sociales en el mismo eje de poder y solidaridad que se expuso en la Tabla 1. Lo anterior con el fin de observar si hay un uso determinado de una lengua por la preferencia del hablante o bien si es impuesto por el interlocutor, así como determinar si hay complementariedad de ámbitos de uso o traslape.

\begin{table}
\caption{\label{tab:guerrero}2. Dominios de uso de las lenguas \citep[305]{Guerrero2016}}


\begin{tabularx}{\textwidth}{XXXXXXXX}

\lsptoprule
{¿Qué lengua utiliza cuando...?} &  & {\textit{(ámbito)}} & {(\textit{parentesco})} & {\textit{(ritual)}} &  &  & \multicolumn{1}{c}{}\\
&  & {casa} & {familia} & {compadre} & {amigos} & {extraño} & \\
&  & {calle} &  &  &  &  & \\
&  & {trabajo} &  &  & {compañero} &  & {jefe}\\
&  & {mercado} &  &  &  &  & {comerciante}\\
&  & {ciudad} &  &  &  &  & {autoridades}\\
&  & {comunidad} &  &  &  &  & {delegados}\\
&  & {escuela} &  &  &  &  & {profesores}\\
&  & {iglesia} &  &  &  &  & {sacerdote}\\
&  & {clínica} &  &  &  &  & {médico}\\
&  & {fiestas} &  &  &  &  & {curandero}\\
&  & {juntas} &  &  &  &  & {secretarios}\\
&  & {futbol} &  &  & {equipo} & {equipo contrario} & {árbitro}\\
%\hhline%%replace by cmidrule{~~~-----}
{\textit{familiaridad} (solidaridad)} &  &  & {(+solidaridad)}

{{}-/+} &  &  &  & {{}-/+}

{(+ poder)}\\
%\hhline%%replace by cmidrule{~~~-----}
\lspbottomrule
\end{tabularx}
\end{table}

Los datos que aquí se exponen pertenecen a este segundo grupo de preguntas, nos enfocamos en la identificación jerárquica de ámbitos lingüísticos con el fin de obtener un esquema de las estrategias de uso de las lenguas en un determinado espacio o situación comunicativa, para así determinar el \textit{habitus} lingüístico de cada lengua en los tres contextos sociolingüísticos aquí estudiados. Los ámbitos de uso que se encuentran en la encuesta fueron definidos a partir de los posibles interlocutores y los espacios físicos y simbólicos que pueden existir en la realidad mexicana. Aunque no pretende ser exhaustiva, busca explorar los campos reconocidos más explícitos de interacción entre los hablantes de una comunidad dada (la familia, la comunidad y el exterior de la comunidad).

Ahora bien, Alonso \citet[217]{Benito2004}(2004: 2)(2004: 2), argumenta que la configuración particular del mercado lingüístico, así como los campos que abarca, dependerá de las condiciones históricas y sociolingüísticas de cada comunidad, y, por lo tanto, "los discursos no son otra cosa que las jugadas prácticas con la que los sujetos que intervienen en un mercado lingüístico, tratando de aumentar sus beneficios simbólicos, adaptándose a las leyes de formación de los valores y a la vez poniendo en juego su capital lingüístico, social y culturalmente codificado".

De esta manera, para esquematizar los \textit{habitus} lingüísticos de cada comunidad y los mercados en que participan (cómo interactúan los campos con los ámbitos de uso de cierta lengua), es necesario representar las distintas fuerzas que participan de la interacción, a saber: (i) en un campo familiar o doméstico, que tiene como escenario la casa, se implica un nivel personal, relacionado con la familia consanguínea y se tiene como interlocutores (actores) a los padres, tíos, padrinos, etcétera; (ii) en un campo comunitario, donde se sitúan las fiestas o las juntas se tiene como interlocutores a los delgados o a los especialistas rituales; y (iii) un campo más amplio como el municipio, la región y el Estado, se tiene interlocutores que generalmente representa la otredad, personas externas a la comunidad, como los patrones, los extraños o los desconocidos. En este sentido, los resultados se presentan siguiendo el esquema de campos que se presenta en la Figura 1, el cual está estructurado dependiendo de la comunidad de estudio y representa el \textit{habitus} lingüístico relatado por los colaboradores en cuanto a la selección y uso de una o varias lenguas en cada campo.




\begin{figure}
\caption{\label{fig:}1. Esquema de los campos y habitus lingüístico \citep[309]{Guerrero2016}.}
%%\includegraphics[width=\textwidth]{figures/guerrero-img001.jpg}
\end{figure}

Así, debido a que históricamente las tres comunidades hablan una lengua indígena, nos proponemos indagar cuáles son los campos en que se desarrollan \textit{habitus} que llevan a sus hablantes a cambiar de lengua, en este caso al español, o en los que se opta por hablar indistintamente en una u otra lengua, es decir de manera bilingüe. Esta selección depende del valor de intercambio que tiene cada lengua en uso, el conjunto de intercambios de productos lingüísticos conforma un mercado lingüístico específico. Estos mercados operan en los campos ya mencionados (mercado doméstico, comunitario, municipal, regional, estatal, etcétera), yendo de lo micro a lo macro-social.



 \subsection{El muestreo en redes}



Debido a la imposibilidad de encuestar a la totalidad de la población de cada comunidad, por las cuestiones económicas y de capital humano que implica, es necesario llevar a cabo un muestreo de la población estudiada (selección de los hablantes). Hay distintas maneras de seleccionar una muestra dependiendo de los objetivos de la investigación\footnote{“Hay básicamente dos métodos de selección de una muestra de hablantes: un método \textit{survey} en el que la selección se hace siguiendo técnicas estrictas de muestreo al azar; y un método de selección intencionada en el que: (a) las características de los hablantes han sido predeterminadas y éstos se seleccionan más o menos al azar hasta completar el número deseado de individuos en cada categoría social; o (b) los hablantes seleccionados constituyen un grupo social compacto, ya sea porque son miembros de una red social, porque viven en la misma manzana o vecindario, porque tienen un dominio común de trabajo (por ejemplo una escuela) etc. Cualquiera que sea el método que el investigador emplee, deberá decidir qué factores extralingüísticos incluirá en el análisis” (\citealt{Silva-CorvalánEnrique-Arias2001}: 51).}, en este caso se hizo un muestreo retomando la noción de 'el amigo de un amigo', esto es, se obtuvieron los datos a partir del trabajo con redes de relaciones sociales, es decir, que el investigador estableció una serie de relaciones de colaboración con algunos miembros de la comunidad, quienes a su vez presentaron al investigador con otras personas con las que estaban vinculadas y así sucesivamente, hasta tener el muestreo y hacer una postestratificación en cuanto a las variables sociales de género, edad, educación formal, lugar de residencia, entre otros. Es importante resaltar que la red como instrumento analítico permite analizar los vínculos de interpretaciones sobre la conducta social que tienen los actores implicados en la red. Para registrar una estructura social de este tipo tenemos que determinar su localización o anclaje (los mercados en los que participa), ver la posición que ocupan los actores (central o periférica), su grado de autonomía con respecto a los demás (su accesibilidad), y el número de vínculos que mantiene con otros actores (su rango). El total de vínculos que existen dentro de la red determinan su densidad (véase Requena  (1989: 137–38)). Un ejemplo de este tipo de redes se presenta en la Figura 2, para Santa María de Ocotán. En ella se observan las distintas relaciones de amistad, laborales y de contacto. Así como los diferentes núcleos de relación como la familia, el trabajo o individuos independientes.


\begin{figure}
\caption{\label{fig:}2. Red de colaboradores de Santa María de Ocotán (Torres \citealt{Sánchez2018})}
\includegraphics[width=\textwidth]{figures/guerrero-img002.png}
\end{figure}

Debido a limitaciones de espacio no nos es posible detallar aquí todos estos aspectos, únicamente se presenta un resumen que esboza los vínculos históricos y sociolingüísticos de cada comunidad, según lo manifestaron en la encuesta los actores de las redes estudiadas y fue corroborado por la etnografía.


 \section{Redes indígenas estudiadas}


Como se mencionó en los aparados anteriores, la encuesta se realizó en tres redes de distintas comunidades indígenas con diversos grados de vitalidad lingüística. Se trata de la comunidad de El Espíritu, Municipio de Alfajayucan (Hidalgo), en donde el \textit{hñähñu} u otomí prácticamente ha sido desplazado por el español; la comunidad de Misión de Chichimecas, San Luis de la Paz (Guanajuato), última localidad en donde se habla la lengua \textit{úza}’ o chichimeca jonaz, por lo que se considera que está en un proceso de desplazamiento; y Santa María de Ocotán, Mezquital (Durango), en donde la lengua con mayor uso es el \textit{o’dam} o tepehuano del sureste.



 \subsection{Red otomí de El Espíritu}



El otomí o \textit{hñähñu} es una lengua otopame, de la rama otomiana (junto con el mazahua, el matlazinca y el ocuilteco), que se habla en ocho estados de la República Mexicana (México, Hidalgo, Querétaro, Guanajuato, Michoacán, Tlaxcala, Puebla y Veracruz). Generalmente se habla de su fragmentación en tres grandes grupos de variantes dialectales, i) las variantes orientales o de la Sierra Oriental, que abarca los estados de Hidalgo, Puebla, Veracruz y Tlaxcala; ii) las variantes de occidentales del norte y el Valle del Mezquital, que tiene como centro el estado de Hidalgo y se extiende al norte del Estado de México, Querétaro y el sureste de Guanajuato; y iii) las variantes del sur de los estados de Querétaro, México y Michoacán (\citealt{Soustelle1990}; \citealt{Lastra2006}; Guerrero \citealt{Galván2013}).

Esta lengua se ha mantenido en contacto intenso con el español desde el siglo XVI, principalmente en la región del Valle del Mezquital, donde las variantes han tenido cambios más profundos y se encuentran en franco desplazamiento lingüístico. Este último es el caso del otomí de la comunidad de El Espíritu, fundada como una visita franciscana del vecino convento de San Martín Alfajayucan que se encuentra a menos de un kilómetro de distancia, hecho que llevó a un contacto cultural intenso debido a la presencia de los frailes. Sin embargo, en relación con el contacto lingüístico, este se dio de manera intensa hasta el siglo XIX, trayendo consigo el inicio de un proceso de estigmatización y represión lingüística del otomí en los espacios públicos de la comunidad tales como la escuela y la iglesia.

Si bien la lengua ya sólo es hablada por personas mayores de 60 años, la población conserva una parte de su identidad otomí en la celebración del Carnaval, el cual reúne de manera sincrética el culto a las potencias de la naturaleza, el santoral católico y el ciclo agrícola. El carnaval de El Espíritu se articula con los de otras tres comunidades:  i) Xamange; ii) San Antonio Corrales, y iii) Boxtho. Los comparsas o \textit{xithás} (abuelos) visitan la iglesia y desfilan bailando por pueblo del Alfajayucan. Su atuendo representa a los antepasados de la comunidad, seres inframundanos que fertilizan las tierras con sus batallas rituales. En El Espíritu, el momento liminal del ritual es marcado por la “barrida” del templo, pues esto ahuyentará a los malos aires, por lo que se realiza en Año Nuevo y antes del Carnaval.

No es posible saber el número exacto de hablantes de otomí en el Espíritu\footnote{Fuentes como \textit{mexico.pueblosdeamerica.com} dan un total de 387 habitantes, 201 mujeres y 186 hombres, con una ratio de fecundidad de la población femenina de 2.77 hijos por mujer y un total de 85 viviendas (consultado el 15 de junio de 2015).}, pero según el censo de 2010, el 33.93 \% de los habitantes de la comunidad de 3 años y más hablan una lengua indígena, el 96.82 \% de ellos son bilingües (lengua indígena-español) y no se registró población que no hablara español. Si bien se trata de una comunidad pequeña (134 viviendas) fundada por familias otomíes, la población en hogares indígenas de 5 años y más representa sólo el 61.57 \%; es decir que el 38.43 \% de los hogares no se consideran indígenas. A nivel municipal los hablantes de lengua indígena representan tan sólo un 18.35\% de la población (véase Tabla 3).

\begin{table}
\caption{\label{tab:guerrero}3. Población de 3 años y más en El Espíritu. Fuente: \citet{INEGI2010}} \textit{mexico.pueblosdeamerica.com}
\begin{tabularx}{\textwidth}{XXXXX}
\lsptoprule
{\textsc{El} \textsc{Espíritu} } & {\textsc{mujeres}} & {\textsc{hombres}} & {\textsc{total}} & {\textsc{alcances} \textsc{y} \textsc{porcentajes}}\\
{Población} & {287} & {270} & {557} & {(RM/H 1.081)}\\
{hablante de lengua indígena} & {98} & {91} & {189} & {33.93 \% (15.76 \%)}\\
{no habla español} & {0} & {0} & {0} & {0}\\
{habla español y lengua indígena} & {94} & {89} & {183} & {96.82 \%}\\
{en hogares indígenas} & {{}-{}-{}-} & {{}-{}-{}-} & {343}  & {61.57 \% (población de 5 años y más)}\\
{grado de escolaridad} & {7.16} & {6.88} & {7.04}  & {Adultos}\\
{analfabetismo} & {(7.46\%)} & {(4.84\%)} & {{}-{}-{}-} & {6.2\%}\\
{viviendas} & {{}-{}-{}-} & {{}-{}-{}-} & {134} & {Comunidad}\\
{Alfajayucan} & {8819} & {8304} & {17123} & {Municipio}\\
{hablante de lengua indígena} & {1559} & {1584} & {3143} & {Municipio}\\
{hablante de lengua indígena} & {187465} & {182084} & {369549} & {Estado}\\
\lspbottomrule
\end{tabularx}
\end{table}

Dentro de esta primera comunidad la encuesta se realizó a una red familiar de 24 miembros, todos vecinos de El Espíritu, 11 mujeres y 13 hombres, lo que es cercano al 4\% de la población femenina y el 5\% de la masculina.


 \subsubsection{Ámbitos de uso en El Espíritu.}


Los resultados de la encuesta advierten, en primer lugar, que no hay un espacio en el que la lengua otomí tenga una preferencia de uso, de tal suerte que en los espacios que está presente con mayores porcentajes, siempre está junto con el español. De igual forma, si observamos el uso del español podemos comprobar que esta lengua es la que los colaboradores dicen usar en la mayoría de los espacios destacando incluso que, en un ámbito como la casa en el que se podría esperar que predominara el otomí, es el español el que obtiene el 67\% frente al 25\% de uso de ambas lenguas y tan solo un 8\% del otomí (véase Tabla 4).

\begin{table}
\caption{\label{tab:guerrero}4. Ámbitos de uso de El Espíritu}
\begin{tabularx}{\textwidth}{XXXXXXXXX}
\lsptoprule
{\textsc{casa}} & {\textsc{fiestas}} & {\textsc{juntas}} & {\textsc{trabajo}} & {\textsc{mercado}} & {\textsc{iglesia}} & {\textsc{escuela}} & {\textsc{clínica}} & {\textsc{ciudad}}\\
{\textsc{es} \textsc{67} \textsc{\%} \textsc{(16/24)}} & {\textsc{bl} \textsc{46} \textsc{\%} \textsc{(11/24)}} & {\textsc{bl} \textsc{71} \textsc{\%} \textsc{(17/24)}} & {\textsc{es} \textsc{62} \textsc{\%} \textsc{(15/24)}} & {\textsc{es} \textsc{67} \textsc{\%} \textsc{(16/24)}} & {\textsc{bl} \textsc{58} \textsc{\%} \textsc{(14/24)}} & {\textsc{es} \textsc{83} \textsc{\%} \textsc{(20/24)}} & {\textsc{es} \textsc{87} \textsc{\%} \textsc{(21/24)}} & {\textsc{es} \textsc{83} \textsc{\%} \textsc{(20/24)}}\\
{\textsc{bl} \textsc{25} \textsc{\%} \textsc{(6/24)}} & {\textsc{es} \textsc{46} \textsc{\%}  \textsc{(11/24)}} & {\textsc{es} \textsc{29} \textsc{\%} \textsc{(7/24)}} & {\textsc{bl} \textsc{21\%} \textsc{(5/24)}} & {\textsc{bl} \textsc{29} \textsc{\%} \textsc{(7/24)}} & {\textsc{es} \textsc{42} \textsc{\%} \textsc{(10/24)}} & {\textsc{bl} \textsc{12} \textsc{\%}  \textsc{(3/24)}} & {\textsc{bl} \textsc{12} \textsc{\%}  \textsc{(3/24)}} & {\textsc{bl} \textsc{12} \textsc{\%}  \textsc{(3/24)}}\\
{\textsc{ot} \textsc{8} \textsc{\%} \textsc{(2/24)}} & {\textsc{ot} \textsc{8} \textsc{\%} \textsc{(2/24)}} &  & {\textsc{ot} \textsc{8} \textsc{\%} \textsc{(1/24)}} & {\textsc{ot} \textsc{8} \textsc{\%} \textsc{(1/24)}} &  &  &  & \\
\lspbottomrule
\end{tabularx}
\end{table}

La Figura 3 (Fuente: INEGI2010) \footnote{El Censo de 2010 fue revisado por medio de la “Consulta interactiva de datos” y los “Datos abiertos” del INEGI (véase http://www.inegi.org.mx/lib/olap/consulta/general\_ver4/MDXQueryDatos.asp?c=27781 consultado el 01 de mayo de 2020). El porcentaje de hablantes de lengua indígena añadido entre paréntesis lo extrajimos de mexico.pueblosdeamerica.com (consultado el 27 de julio de 2015).} representa los distintos ámbitos de uso en relación con los campos de la comunidad, el municipio, la región y el Estado, nos muestra una preferencia por el español en los niveles regionales, el uso del bilingüismo en los ámbitos comunitarios de la fiesta, la iglesia y las juntas, que en este caso también son espacios de interacción con la vecina cabecera municipal. Como se mencionó en el párrafo anterior llama la atención el uso predominante del español en un espacio privado como el de la casa.


\begin{figure}
\includegraphics[width=\textwidth]{figures/guerrero-img003.jpg}
\caption{\label{fig:}3. Esquema del habitus lingüístico en El Espíritu (elaborada por los autores).}
\end{figure}



 \subsection{Red chichimeca de Misión de Chichimecas}



El \textit{úza}’ o chichimeca jonaz es una lengua de la familia otopame, de la rama pameana (junto con el pame de norte y el del sur). Los hablantes de esta lengua se mantuvieron reacios al contacto con los españoles durante la época colonial, participando activamente en la llamada “guerra chichimeca” que se prolongó hasta mediados del siglo XVII cuando los españoles implementaron la “paz por compra” y destinaron una renta a los chichimecas para que no siguieran atacando sus recuas de mulas que circulaban con mercancías por el Camino Real de Tierra Adentro. El grupo que sobrevivió de estos indómitos guerreros fueron denominados chichimeca jonaz o \textit{eza’r} como ellos se autodenominan.

A pesar de su tendencia al aislamiento y a mantenerse como un grupo relativamente pequeño, los \textit{eza’r} han tenido que irse abriendo al contacto, sobre todo desde los inicios del siglo XX. Su lengua está experimentando cambios importantes, como por ejemplo el paso de /y/ por /i/ y el de /ts/ por /s/, que probablemente se han ido acelerado por su contacto y convergencia con el español \citep{GuerreroEtAl2017}.

La comunidad de Misión de Chichimecas está situada a tres kilómetros de la cabecera municipal de San Luis de Paz, Guanajuato. Esta cercanía ha hecho que la situación comunitaria empiece con un proceso de desplazamiento de la lengua indígena por el español, por ser esta última la lengua mayoritaria y de prestigio a nivel nacional.

En los años setenta la comunidad fue dividida por libramiento carretero hacia el municipio de Victoria, lo que separó a la Misión en Misión de Abajo y Misión de Arriba. Misión de Abajo, fue el primer asentamiento \textit{eza’r} y espacio en el que se colocaron las primeras escuelas y servicios públicos tales como luz y drenaje; debido a la cercanía con la cabecera municipal empezó a ser una zona de avecindados mestizos lo que cambió la dinámica del uso del \textit{úza}’. Es importante destacar que también se encuentran mestizos avecindados en Misión de Arriba, pero en menor porcentaje, por lo que la mayoría de la población es chichimeca. Ambas parcialidades celebran la fiesta de la Virgen de Guadalupe, a quien consideran patrona y ofrendan un gran \textit{chimal} o \textit{chich’á}\footnote{El chimal o \textit{chich’á} es una estructura de madera y metal de varios metros de altura, tejida artísticamente con la planta del mismo nombre, perteneciente a la familia de los agaves como el sotol y el mezcal. Esta ofrenda es cargada en hombros por los \textit{eza’r} y acompañada por danzas de guerreros chichimecas desde la Capilla de Cerrito en Misión de Abajo hasta el Santuario de Guadalupe que dista de unos tres kilómetros.}.

También es difícil determinar el número exacto de hablantes de lengua indígena en Misión de Chichimecas\footnote{En \textit{mexico.pueblosdeamerica.com} dan un total de 4262 habitantes, 2139 mujeres y 2123 hombres, con una tasa de fecundidad de la población femenina de 4.11 hijos por mujer y un total de 587 viviendas (consultado el 15 de junio de 2015).}, el censo de 2010 registra 2011 habitantes entre los 3 años y más que hablan una lengua indígena, lo que representa el 30.76\% de la población total registrada; se presenta un monolingüismo (no habla español) del 2.73\% y un bilingüismo (lengua indígena-español) del 95.82\%. A nivel municipal se cuentan 2273 hablantes de lengua indígena, los cuales representan el 2.22\% de los habitantes del municipio. En Misión se registran 6716 habitantes, por lo que si todos los hablantes de lengua indígena registrados para el municipio en el censo vivieran en esta comunidad representarían el 33.84\% de la población. No obstante, en Misión los hogares indígenas representan el 61.41\% de la población de 5 años y más, en una localidad con 1157 viviendas (véase Tabla 5).

\begin{table}
\caption{\label{tab:guerrero}5. Población de Misión de Chichimecas}
Fuente: \citet{INEGI2010}

\begin{tabularx}{\textwidth}{XXXXX}
\lsptoprule
{\textsc{Misión} \textsc{de} \textsc{chichimecas}} & {\textsc{mujeres}} & {\textsc{hombres}} & {\textsc{total}} & {\textsc{alcances} \textsc{y} \textsc{porcentajes}}\\
{Población} & {3386} & {3330} & {6716} & {(RM/H 1.008)}\\
{hablante de lengua indígena} & {969} & {1042} & {2011} & {30.76 \%}\\
{no habla español} & {34} & {21} & {55} & {2.73 \%}\\
{habla español y lengua indígena} & {936} & {991} & {1927} & {95.82 \%}\\
{en hogares indígenas} & {{}-{}-{}-} & {{}-{}-{}-} & {3488} & {61.41 \% (población de 5 años y más)}\\
{grado de escolaridad} & {3.34} & {3.86} & {3.59}  & {Adultos}\\
{analfabetismo} & {(21.83\%)} & {(13.94\%)} & {{}-{}-{}-} & {17.9\%}\\
{viviendas} & {{}-{}-{}-} & {{}-{}-{}-} & {1157} & {Comunidad}\\
{Población de San Luis de la Paz} & {54241} & {47902} & {102143} & {Municipio}\\
{hablante de lengua indígena} & {1089} & {1184} & {2273} & {Municipio}\\
{hablante de lengua indígena} & {7026} & {8178} & {15204} & {Estado}\\
\lspbottomrule
\end{tabularx}
\end{table}

Los datos de esta segunda comunidad se obtuvieron de un total de 11 colaboradores, 8 hombres y 3 mujeres, lo que representaría el 0.2\% de la población masculina y 0.8\% femenina radicada en Misión que, como mencionamos en el párrafo anterior, no necesariamente es hablante \textit{úza’}.


 \subsubsection{Ámbitos de uso Misión de Chichimecas}


Para el caso de Misión de Chichimecas podemos observar en la Tabla 6 que, de los nueve ámbitos, en tres el chichimeca tiene preferencia de uso: la casa, el mercado y la escuela; y en cuatro se usan las dos lenguas: fiesta, junta, trabajo e iglesia y tan sólo en dos, la clínica y la ciudad, se destaca el uso del español. A diferencia de El Espíritu podemos observar que en la comunidad de Misión de Chichimecas la lengua indígena sigue estando en uso sin que el español esté presente.

\begin{table}
\caption{\label{tab:guerrero}6. Ámbitos de uso Misión de Chichimecas}


\begin{tabularx}{\textwidth}{XXXXXXXXX}
\lsptoprule
{\textsc{casa}} & {\textsc{fiestas}} & {\textsc{juntas}} & {\textsc{trabajo}} & {\textsc{mercado}} & {\textsc{iglesia}} & {\textsc{escuela}} & {\textsc{clínica}} & {\textsc{ciudad}}\\
{\textsc{ch} \textsc{54} \textsc{\%} \textsc{(6/11)}} & {\textsc{bl} \textsc{45} \textsc{\%} \textsc{(5/11)}} & {\textsc{bl} \textsc{54} \textsc{\%} \textsc{(6/11)}} & {\textsc{bl} \textsc{36} \textsc{\%} \textsc{(4/11)}} & {\textsc{ch} \textsc{45} \textsc{\%} \textsc{(5/11)}} & {\textsc{bl} \textsc{36} \textsc{\%} \textsc{(4/11)}} & {\textsc{ch} \textsc{27} \textsc{\%} \textsc{(3/11)}} & {\textsc{es} \textsc{54} \textsc{\%} \textsc{(6/11)}} & {\textsc{es} \textsc{54} \textsc{\%} \textsc{(6/11)}}\\
{\textsc{es} \textsc{18} \textsc{\%} \textsc{(2/11)}} & {\textsc{ch} \textsc{36} \textsc{\%} \textsc{(4/11)}} & {\textsc{ch} \textsc{27} \textsc{\%} \textsc{(3/11)}} & {\textsc{ch} \textsc{27} \textsc{\%} \textsc{(3/11)}} & {\textsc{bl} \textsc{27} \textsc{\%} \textsc{(3/11)}} & {\textsc{es} \textsc{36} \textsc{\%} \textsc{(4/11)}} & {\textsc{es} \textsc{18} \textsc{\%} \textsc{(2/11)}} & {\textsc{bl} \textsc{27} \textsc{\%} \textsc{(3/11)}} & {\textsc{bl} \textsc{27} \textsc{\%} \textsc{(3/11)}}\\
{\textsc{bl} \textsc{27} \textsc{\%} \textsc{(3/11)}} & {\textsc{es} \textsc{18} \textsc{\%} \textsc{(2/11)}} & {\textsc{es} \textsc{18} \textsc{\%} \textsc{(2/11)}} & {\textsc{es} \textsc{18} \textsc{\%} \textsc{(2/11)}} & {\textsc{es} \textsc{18} \textsc{\%} \textsc{(2/11)}} & {\textsc{ch} \textsc{9} \textsc{\%} \textsc{(1/11)}} & {\textsc{bl} \textsc{9} \textsc{\%} \textsc{(1/11)}} &  & \\
\lspbottomrule
\end{tabularx}
\end{table}

En la Figura 4, se muestra como el chichimeca es usado preferentemente en tres niveles distintos, a nivel particular en la casa, a nivel comunitario en las fiestas y en la escuela, y el mercado que representan un nivel de interacción entre la comunidad y el municipio. En cambio, en los ámbitos con una mayor injerencia de la municipalidad tales como la clínica y la propia ciudad se prefiere el uso del español.



\begin{figure}
%%\includegraphics[width=\textwidth]{figures/guerrero-img004.jpg}
\caption{\label{fig:}4. Esquema del habitus lingüístico en Misión de Chichimecas (elaboración de los autores).}
\end{figure}



 \subsection{Red tepehuana de Santa María de Ocotán}



El \textit{o’dam} o tepehuano del sureste es una lengua de la familia yutoazteca, de la rama pimana (junto con el tepehuano de norte, el pima y el pápago), pertenece culturalmente a la región conocida como Gran Nayar (Reyes \citealt{Valdez2006a}), donde ha mantenido un contacto histórico con coras, huicholes y mexicaneros. De manera contraria, con respecto al contacto con el español, este se mantuvo con un bajo dominio de ámbitos.

Si bien se considera la incursión de Nuño de Beltrán en el año de 1531 como una primera presencia española en la zona, fue hasta ocho años después al fundarse San Francisco de Mezquital y sobretodo en el año de 1600, que inició el contacto con la lengua española, pues fue cuando los frailes franciscanos entraron a la sierra a través de las visitas parroquiales, es decir, que acudían a las rancherías periódicamente, pero no se establecía un religioso de manera permanente. Sin embargo, estas primeras incursiones se vieron interrumpidas por la conocida “rebelión tepehuana” a inicios del siglo XVII (Reyes \citealt{Valdez2006a}) y al menos, para Santa María de Ocotán, se tiene el registro de un capítulo provincial de 1806 en el que se especifica la presencia permanente de un religioso en la comunidad (De la Torre \citealt{Curiel2006}: 160), hecho que implicó un contacto más permanente con hispanohablantes, pero no necesariamente fuerte.

Una de las características culturales más importante de este grupo es el ciclo ceremonial en el que se encuentran tanto las fiestas de origen católico y las de tradición prehispánica. Esta última es conocida como xiotalh o \textit{mitote}\footnote{El mitote o \textit{xiotalh} está relacionado con los cambios estacionales y “son ceremonias que duran cinco días (aunque hay mitotes familiares de tres días); su principal característica es un baile nocturno que se realiza el último día de la ceremonia alrededor de una fogata y un músico que, colocado en el centro de la plaza, percute la cuerda de un arco sobre un tecomate que le sirve de caja de resonancia” (Reyes \citealt{Valdez2006b}: 17).} la cual se realiza de manera comunal o familiar (Reyes \citealt{Valdez2006b,2006a2006b}).

Actualmente, los tepehuanos suelen migrar para trabajar en los cultivos de tomate, frijol y manzana a los estados de Sinaloa, Nayarit y Chihuahua, o bien migran a la ciudad de Durango de manera temporal o definitiva, pero en la mayoría de los casos regresando al menos una vez al año a la comunidad a la que pertenecen en la sierra.

Como en los otros dos casos, no es posible determinar el número exacto de hablantes de lengua indígena en Santa María de Ocotán, el censo de 2010 registra 474 habitantes entre los 3 años y más, y 542 personas de 5 años o más en hogares indígenas. Es la comunidad que reporta un mayor porcentaje de monolingüismo (no habla español) del 25.68\% y un menor bilingüismo (lengua indígena-español) del 96.82\%. A nivel municipal se cuentan 23724 hablantes de lengua indígena; es decir, el 79.10\% de los habitantes del municipio. El estado de Durango alcanza los 32917 hablantes de lengua indígena, pero también hay presencia de otros grupos como wixarika y rarámuri. Santa María es una comunidad pequeña con 91 viviendas censadas, pero el 100 \% de la población de 5 años y más vive en un hogar indígena (véase Tabla 7).

\begin{table}
\caption{\label{tab:guerrero}7. Población de Santa María Ocotán }
Fuente: \citet{INEGI2010}; \textit{mexico.pueblosdeamerica.com}
\begin{tabularx}{\textwidth}{XXXXX}
\lsptoprule
{\textsc{Sta.} \textsc{María} \textsc{de} \textsc{Ocotán}} & {\textsc{mujeres}} & {\textsc{hombres}} & {\textsc{total}} & {\textsc{alcances} \textsc{y} \textsc{porcentajes}}\\
{población} & {258} & {256} & {514} & {(RM/H 1.008)}\\
{hablante de lengua indígena} & {231} & {243} & {474} & {92.21 \% (80.54 \%)}\\
{no habla español} & {89} & {43} & {132} & {25.68 \%}\\
{habla español y lengua indígena} & {142} & {200} & {342} & {96.82 \%}\\
{en hogares indígenas} & {{}-{}-{}-} & {{}-{}-{}-} & {542}  & {100 \% (población de 5 años y más)}\\
{grado de escolaridad} & {4.22} & {6.73} & {5.37}  & {Adultos}\\
{analfabetismo} & {(24.42\%)} & {(6.25\%)} & {{}-{}-{}-} & {15.37\%}\\
{viviendas} & {{}-{}-{}-} & {{}-{}-{}-} & {91} & {Comunidad}\\
{Población del Mezquital} & {29989} & {14793} & {29989} & {Municipio}\\
{hablante de lengua indígena} & {12062} & {11680} & {23724} & {Municipio}\\
{hablante de lengua indígena} & {16393} & {16524} & {32917} & {Estado}\\
\lspbottomrule
\end{tabularx}
\end{table}

Los datos de esta comunidad se obtuvieron de un total de 18 colaboradores pertenecientes a distintas redes familiares de la comunidad, 7 hombres y 12 mujeres, lo que implica cerca del 3\% de la población masculina y del 5\% de la femenina.


 \subsubsection{Ámbitos de uso en Santa María de Ocotán.}


En Santa María de Ocotán encontramos tres espacios en los que el tepehuano tiene preferencia de uso: la casa, la fiesta y la iglesia. Sin embargo, como se puede observar en la Tabla 8, destaca el hecho que en ninguno de los nueve ámbitos el español sea el que predomine, pues en los seis restantes es el uso de las dos lenguas el que obtiene un mayor porcentaje incluyendo espacios como la ciudad, la cual es un contexto más hispanohablante.

\begin{table}
\caption{\label{tab:guerrero}8. Ámbitos de uso de Santa María de Ocotán}
\begin{tabularx}{\textwidth}{XXXXXXXXX}
\lsptoprule
{\textsc{casa}} & {\textsc{fiestas}} & {\textsc{juntas}} & {\textsc{trabajo}} & {\textsc{mercado}} & {\textsc{iglesia}} & {\textsc{escuela}} & {\textsc{clínica}} & {\textsc{ciudad}}\\
{\textsc{te} \textsc{66.6\%} \textsc{(12/18)}} & {\textsc{te} \textsc{94.4} \textsc{\%} \textsc{(17/18)}} & {\textsc{bl} \textsc{58.8} \textsc{\%} \textsc{(10/17)}} & {\textsc{bl} \textsc{71.4} \textsc{\%} \textsc{(10/14)}} & {\textsc{bl} \textsc{55} \textsc{\%} \textsc{(10/18)}} & {\textsc{te} \textsc{68.7} \textsc{\%} \textsc{(11/16)}} & {\textsc{bl} \textsc{92.3} \textsc{\%} \textsc{(12/13)}} & {\textsc{bl} \textsc{55} \textsc{\%} \textsc{(10/18)}} & {\textsc{bl} \textsc{66} \textsc{\%} \textsc{(12/18)}}\\
{\textsc{bl} \textsc{33.3} \textsc{\%} \textsc{(6/18)}} & {\textsc{bl} \textsc{5.8} \textsc{\%} \textsc{(1/18)}} & {\textsc{te} \textsc{41.1} \textsc{\%} \textsc{(7/17)}} & {\textsc{es} \textsc{21.4} \textsc{\%} \textsc{(3/14)}} & {\textsc{es} \textsc{44} \textsc{\%} \textsc{(8/18)}} & {\textsc{bl} 25 \% (4/16)} & {\textsc{te} \textsc{7.6\%} \textsc{(1/13)}} & {\textsc{te} \textsc{28} \textsc{\%} \textsc{(5/18)}} & {\textsc{es} \textsc{33} \textsc{\%} \textsc{(6/18)}}\\
&  &  & {\textsc{te} \textsc{7.1} \textsc{\%} \textsc{(1/14)}} &  & {\textsc{es} \textsc{6.2} \textsc{\%} \textsc{(1/16)}} &  & {\textsc{es} 16 \% (3/18)} & \\
\lspbottomrule
\end{tabularx}
\end{table}

Asimismo, en la Figura 5 observamos como el tepehuano es usado de manera preferencial en los ámbitos privados como la casa y comunitarios como la iglesia y las fiestas, mientras que en todos los demás ámbitos el bilingüismo es la norma extendiéndose al espacio estatal, hecho que no se presenta en las dos comunidades anteriores.




\begin{figure}
\caption{\label{fig:}5. Esquema del habitus lingüístico de Santa María de Ocotán (elaboración de los autores).}
\includegraphics[width=\textwidth]{figures/guerrero-img005.jpg}
\end{figure}


 \section{Resultados comparativos}


Como se mencionó al inicio de este trabajo, uno de los principales objetivos de usar tanto, un mismo instrumento de elicitación como la noción de \textit{habitus lingüístico} para describir el uso diferenciado de las lenguas habladas era con fines comparativos. Así, el analizar distintas comunidades que tiene sus propias dinámicas de uso, según sea su realidad sociolingüística, nos permite observar ciertos patrones que se cumplen o bien que se rompen. En la sección que sigue compararemos los ámbitos en los que se refleja la presencia de la lengua indígena, de ambas lenguas y del español, destacando que la forma en que estos ámbitos se van jerarquizando muestra dónde se usa qué lengua y cómo se prefiere el uso de una sobre otra.



 \subsection{Ámbitos de la lengua indígena.}



En la Tabla 9 muestra que, de los espacios en los que hay presencia de uso de la lengua indígena, en ninguno de éstos el \textit{hñähñu} obtiene un mayor porcentaje. No obstante, es importante resaltar que en los espacios de la fiesta, la casa y el mercado es donde el otomí obtiene sus porcentajes más altos con un 8\%. Esto, puede relacionarse, a que en El Espíritu el espacio privilegiado para la reproducción cultural, y que se ha mantenido como pilar identitario, es la fiesta, particularmente la del Carnaval.

\begin{table}
\caption{\label{tab:guerrero}9. Ámbitos de El Espíritu con presencia del otomí (OT)}


\begin{tabularx}{\textwidth}{XXXX}
\lsptoprule
\textsc{fiestas} & {\textsc{casa}} & {\textsc{trabajo}} & {\textsc{mercado}}\\
{\textsc{ot} \textsc{8} \textsc{\%} \textsc{(2/24)}} & {\textsc{ot} \textsc{8} \textsc{\%} \textsc{(2/24)}} & {\textsc{ot} \textsc{4.1} \textsc{\%} \textsc{(1/24)}} & {\textsc{ot} \textsc{8} \textsc{\%} \textsc{(1/24)}}\\
{\textsc{es} \textsc{46} \textsc{\%}  \textsc{(11/24)}} & {\textsc{es} \textsc{67} \textsc{\%} \textsc{(16/24)}} & {\textsc{es} \textsc{62} \textsc{\%} \textsc{(15/24)}} & {\textsc{es} \textsc{67} \textsc{\%} \textsc{(16/24)}}\\
{\textsc{bl} \textsc{46} \textsc{\%} \textsc{(11/24)}} & {\textsc{bl} \textsc{25} \textsc{\%} \textsc{(6/24)}} & {\textsc{bl} \textsc{21\%} \textsc{(5/24)}} & {\textsc{bl} \textsc{29} \textsc{\%} \textsc{(7/24)}}\\
\lspbottomrule
\end{tabularx}
\end{table}
Por su parte, en Misión de Chichimecas la Tabla 10 nos muestra que, a diferencia de El Espíritu, tanto en la casa como en el mercado el chichimeca es la lengua que obtiene un mayor porcentaje con un 54\% y 45\%, respectivamente. Sin embargo, a pesar de no obtener los porcentajes más altos la fiesta tradicional (36\%) también es un espacio de uso de chichimeca, con un papel muy importante para en la reproducción cultural e identitaria del grupo.

\begin{table}
\caption{\label{tab:guerrero}10. Ámbitos de Misión de Chichimecas en que se prefiere el chichimeca (CH)}
\begin{tabularx}{\textwidth}{XXXX}
\lsptoprule
{\textsc{casa}} & {\textsc{mercado}} & {\textsc{ciudad}} & {\textsc{fiestas}}\\
{\textsc{ch} \textsc{54} \textsc{\%} \textsc{(6/11)}} & {\textsc{ch} \textsc{45} \textsc{\%} \textsc{(5/11)}} & {\textsc{ch} \textsc{27} \textsc{\%} \textsc{(3/11)}} & {\textsc{ch} \textsc{36} \textsc{\%} \textsc{(4/11)}}\\
{\textsc{es} \textsc{18} \textsc{\%} \textsc{(2/11)}} & {\textsc{es} \textsc{18} \textsc{\%} \textsc{(2/11)}} & {\textsc{es} \textsc{45} \textsc{\%} \textsc{(5/11)}} & {\textsc{bl} \textsc{45} \textsc{\%} \textsc{(5/11)}}\\
{\textsc{bl} \textsc{27} \textsc{\%} \textsc{(3/11)}} & {\textsc{bl} \textsc{27} \textsc{\%} \textsc{(3/11)}} &  & \\
\lspbottomrule
\end{tabularx}
\end{table}

Finalmente, en Santa María Ocotán, a diferencia de las otras dos comunidades arriba mencionadas, el uso de \textit{o'dam} obtiene altos porcentajes en un mayor número de espacios tales como las fiestas con 94.4\% (17/18), seguido por la casa con 66\% (12/18) y la iglesia con 68.7\% (11/16) (véase Tabla 11). Destaca nuevamente que tanto las fiestas como la Iglesia son dos ámbitos en los que, de manera muy similar a los casos anteriores, tienen una relación con la tradición y cosmovisión tepehuana.

\begin{table}
\caption{\label{tab:guerrero}11. Dominios de Santa María de Ocotán en que se prefiere el tepehuano (TE)}
\begin{tabularx}{\textwidth}{XXXX}
\lsptoprule
{\textsc{fiestas}} & {\textsc{casa}} & {\textsc{iglesia}} & {\textsc{juntas}}\\
{\textsc{te} \textsc{94.4} \textsc{\%} \textsc{(12/18)}} & {\textsc{te} \textsc{66.6} \textsc{\%} \textsc{(12/18)}} & {\textsc{te} \textsc{68.7} \textsc{\%} \textsc{(11/16)}} & {\textsc{te} \textsc{41.4} \textsc{\%} \textsc{(7/17)}}\\
{\textsc{bl} \textsc{5.8\%} \textsc{(6/18)}} & {\textsc{bl} \textsc{33.3} \textsc{\%} \textsc{(6/18)}} & {\textsc{bl} 25 \% (3/16)} & {\textsc{bl} \textsc{58.8} \textsc{\%} \textsc{(10/17)}}\\
&  & {\textsc{es} \textsc{6.2} \textsc{\%} \textsc{(1/16)}} & \\
\lspbottomrule
\end{tabularx}
\end{table}



 \subsection{Ámbitos del uso bilingüe}



A diferencia del uso del \textit{hñähñu}, la Tabla 12 muestra que en El Espíritu el uso bilingüe tiene mayores porcentajes en las fiestas (46\%), la iglesia (58\%) y las juntas (71\%), nuevamente se observa que son espacios que representan dinámicas comunitarias. Si bien, el hecho de que la lengua indígena no aparezca de manera solitaria en ningún espacio refleja el estado de obsolescencia en el que se encuentra. Sin embargo, los usos bilingües pueden tomarse como una buena señal para futuros trabajos de revitalización y mantenimiento del \textit{hñähñu.}

\begin{table}
\caption{\label{tab:guerrero}12. Ámbitos de El Espíritu en que se prefiere el bilingüismo (BL)}
\begin{tabularx}{\textwidth}{XXXX}
\lsptoprule
\textsc{fiestas} & {\textsc{iglesia}} & {\textsc{mercado}} & {\textsc{juntas}}\\
{\textsc{bl} \textsc{46} \textsc{\%} \textsc{(11/24)}} & {\textsc{bl} \textsc{58} \textsc{\%} \textsc{(14/24)}} & {\textsc{bl} \textsc{29} \textsc{\%} \textsc{(7/24)}} & {\textsc{bl} \textsc{71} \textsc{\%} \textsc{(17/24)}}\\
{\textsc{es} \textsc{46} \textsc{\%} \textsc{(11/24)}} & {\textsc{es} \textsc{42} \textsc{\%} \textsc{(10/24)}} & {\textsc{es} \textsc{67} \textsc{\%} \textsc{(16/24)}} & {\textsc{es} \textsc{29} \textsc{\%} \textsc{(7/24)}}\\
{\textsc{ot} \textsc{8} \textsc{\%} \textsc{(2/24)}} &  & {\textsc{ot} \textsc{8} \textsc{\%} \textsc{(1/24)}} & \\
\lspbottomrule
\end{tabularx}
\end{table}

La Tabla 13 exhibe que el uso dominante del bilingüismo entre los chichimecas, a diferencia de lo que sucede en El Espíritu, esté indicando un aumento en el uso del español en dominios que en otro momento fueron típicamente chichimecas, como las juntas, donde se registra un 54\%, y las fiestas con un 45\%.

  En Misión, las juntas o asambleas son el máximo órgano regulador de la comunidad, en ellas se eligen a los delegados municipales por sufragio universal y se discuten los asuntos trascendentales y de interés público, pero también es donde se negocia discursivamente con el mundo hispanohablante y la sociedad nacional, por lo que no es extraño que el uso del español alcance un 18\%, frente al 27\% del \textit{uza’} o chichimeca.

\begin{table}
\caption{\label{tab:guerrero}13. Ámbitos de Misión de Chichimecas en que se prefiere el bilingüismo (BL)}


\begin{tabularx}{\textwidth}{XXXX}
\lsptoprule
{\textsc{juntas}} & {\textsc{fiestas}} & {\textsc{trabajo}} & {\textsc{iglesia}}\\
{\textsc{bl} \textsc{54} \textsc{\%} \textsc{(6/11)}} & {\textsc{bl} \textsc{45} \textsc{\%} \textsc{(5/11)}} & {\textsc{bl} \textsc{36} \textsc{\%} \textsc{(4/11)}} & {\textsc{bl} \textsc{36} \textsc{\%} \textsc{(4/11)}}\\
{\textsc{ch} \textsc{27} \textsc{\%} \textsc{(3/11)}} & {\textsc{ch} \textsc{36} \textsc{\%} \textsc{(4/11)}} & {\textsc{ch} \textsc{27} \textsc{\%} \textsc{(3/11)}} & {\textsc{es} \textsc{36} \textsc{\%} \textsc{(4/11)}}\\
{\textsc{es} \textsc{18} \textsc{\%} \textsc{(2/11)}} &  & {\textsc{es} \textsc{18} \textsc{\%} \textsc{(2/11)}} & {\textsc{ch} \textsc{9} \textsc{\%} \textsc{(1/11)}}\\
\lspbottomrule
\end{tabularx}
\end{table}

Por último, en San María de Ocotán con excepción de los ámbitos mencionados en la sección anterior, el resto de los espacios requieren un uso bilingüe. La Tabla 14, advierte que muchos de ellos se pueden asociar campos que están más alejados de la vida comunitaria empezando por el desplazamiento a la ciudad de Durango con un 66\%, la clínica que se encuentra en la comunidad con un 55\% y el mercado 55\% que solo se encuentra en la ciudad. Es importante mencionar, que el uso bilingüe en la clínica se debe a que, si bien el doctor es monolingüe en español, la enfermera que lo asiste es hablante de \textit{o'dam} y la mayoría de los colaboradores afirmaron que prefieren ir cuando se encuentra ella.

Particularmente en el ámbito de las juntas que, si bien es un ámbito muy relacionado con la vida de la comunidad, sucede algo similar que en el caso de Misión de Chichimecas. Es decir, en él se discuten tanto los temas relacionado con la comunidad como los temas externos, razón por la cual hay una presencia de las dos lenguas (\textit{o'dam} y español). Es importante destacar que fue en este espacio en el único en el que los colaboradores aseguraban mezclar las lenguas.

Asimismo, es importante destacar los porcentajes encontrados en el ámbito de la escuela, pues nuevamente es el uso bilingüe el que obtiene un mayor porcentaje (92.3\%), el cual muestra que, al igual que en otras situaciones de contacto la escuela, es uno de los primeros espacios en los que los colaboradores tienen un primer contacto con el español, y donde el uso de esta lengua se va haciendo cada vez más habitual. Sin embargo, incluso en este ámbito el uso del español no aparece sin la presencia del tepehuano, pero sí el uso del tepehuano lo que muestra el amplio uso de la lengua indígena.

\begin{table}
\caption{\label{tab:guerrero}14. Ámbitos de Santa María de Ocotán en que se prefiere el bilingüismo (BL)}
\begin{tabularx}{\textwidth}{XXXXX}
\lsptoprule
{\textsc{escuela}} & {\textsc{ciudad}} & {\textsc{juntas}} & {\textsc{clínica}} & {\textsc{mercado}}\\
{\textsc{bl} \textsc{92.3} \textsc{(12/13)}} & {\textsc{bl} \textsc{66} \textsc{\%} \textsc{(12/18)}} & {\textsc{bl} \textsc{55} \textsc{\%} \textsc{(10/18)}} & {\textsc{bl} \textsc{55} \textsc{\%} \textsc{(10/18)}} & {\textsc{bl} \textsc{55} \textsc{\%} \textsc{(10/18)}}\\
{\textsc{te} \textsc{7.6} \textsc{(1/13)}} & {\textsc{es} \textsc{33} \textsc{\%} \textsc{(6/18)}} & {\textsc{te} \textsc{39} \textsc{\%} \textsc{(7/18)}} & {\textsc{te} \textsc{28} \textsc{\%} \textsc{(5/18)}} & {\textsc{es} \textsc{44} \textsc{\%} \textsc{(8/18)}}\\
&  & {\textsc{es} 16 \% (3/18)} & {\textsc{es} 16 \% (3/18)} & \\
\lspbottomrule
\end{tabularx}
\end{table}



 \subsection{Ámbitos de uso del español.}



En El Espíritu el español tiene una fuerte presencia y, aunado a lo expuesto en los apartados anteriores, nos da pie a sugerir que los miembros de esta comunidad se encuentran en un proceso de desplazamiento del otomí en favor del español. La Tabla 15, muestra que la clínica con un 87\%, seguida por la cuidad y la escuela, ambos con un 83\% (20/24), muestran el dominio del español. Estos tres ámbitos comparten el hecho de ser espacios con una alta presencia de hispanohablantes tanto en la atención como en la solicitud del servicio además de ser los más distantes a la comunidad.

\begin{table}
\caption{\label{tab:guerrero}15. Dominios El Espíritu en que se prefiere el español (ES)}
\begin{tabularx}{\textwidth}{XXXX}
\lsptoprule
\textsc{clínica} & {\textsc{ciudad}} & {\textsc{escuela}} & {\textsc{juntas}}\\
{\textsc{es} \textsc{87} \textsc{\%} \textsc{(21/24)}} & {\textsc{es} \textsc{83} \textsc{\%} \textsc{(20/24)}} & {\textsc{es} \textsc{83} \textsc{\%} \textsc{(20/24)}} & {\textsc{es} \textsc{71} \textsc{\%} \textsc{(17/24)}}\\
{\textsc{bl} \textsc{12} \textsc{\%} \textsc{(3/24)}} & {\textsc{bl} \textsc{12} \textsc{\%} \textsc{(3/24)}} & {\textsc{bl} \textsc{12} \textsc{\%} \textsc{(3/24)}} & {\textsc{bl} \textsc{29} \textsc{\%} \textsc{(7/24)}}\\
\lspbottomrule
\end{tabularx}
\end{table}

En Misión de Chichimecas, la Tabla 16 también exhibe el papel del español en la comunidad, siendo la clínica y la ciudad las que se destacan con un mayor porcentaje, ambas con un 54\% y la Iglesia tiene un porcentaje de 36\% al igual que el uso bilingüe. A diferencia de la comunidad de El Espíritu, entre los chichimecas se puede observar que la lengua \textit{úza'}, ya sea de manera solitaria o en conjunto con el español sigue teniendo un valor de uso.

\begin{table}
\caption{\label{tab:guerrero}16. Dominios de Misión de Chichimecas en que se prefiere el español (ES)}
\begin{tabularx}{\textwidth}{XXXX}
\lsptoprule
{\textsc{clínica}} & {\textsc{iglesia}} & {\textsc{ciudad}} & {\textsc{juntas}}\\
{\textsc{es} \textsc{54} \textsc{\%} \textsc{(6/11)}} & {\textsc{es} \textsc{36} \textsc{\%} \textsc{(4/11)}} & {\textsc{es} \textsc{54} \textsc{\%} \textsc{(6/11)}} & {\textsc{es} \textsc{18} \textsc{\%} \textsc{(2/11)}}\\
{\textsc{bl} \textsc{27} \textsc{\%} \textsc{(3/11)}} & {\textsc{bl} \textsc{36} \textsc{\%} \textsc{(4/11)}} & {\textsc{bl} \textsc{27} \textsc{\%} \textsc{(3/11)}} & {\textsc{ch} \textsc{27} \textsc{\%} \textsc{(3/11)}}\\
& {\textsc{ch} \textsc{9} \textsc{\%} \textsc{(1/11)}} &  & {\textsc{bl} \textsc{54} \textsc{\%} \textsc{(6/11)}}\\
\lspbottomrule
\end{tabularx}
\end{table}

A diferencia de las otras dos comunidades, Santa María Ocotán se destaca porque en ninguno de los ámbitos de uso el español tiene el porcentaje más alto, en relación con el uso de la lengua indígena o bilingüe. No obstante, la Tabla 17 indica que los espacios en los que se usa el español con porcentajes menores son aquellos en los que se da una interacción con el mundo hispanohablante y que se encuentran fuera de la comunidad. Ejemplos de esto son el mercado, con un 44\% que se encuentra en la ciudad o en la cabecera municipal y las migraciones que hacen ya sea a la ciudad con un 33\% o a diferentes estados para trabajar con un 21.4\%.

\begin{table}
\caption{\label{tab:guerrero}17. Dominios de Santa María de Ocotán con presencia del español (ES)}
\begin{tabularx}{\textwidth}{XXXX}
\lsptoprule
{\textsc{mercado}} & {\textsc{ciudad}} & {\textsc{clínica}} & {\textsc{trabajo}}\\
{\textsc{es} \textsc{44} \textsc{\%} \textsc{(8/18)}} & {\textsc{es} \textsc{33} \textsc{\%} \textsc{(6/18)}} & {\textsc{es} 16 \% (3/18)} & {\textsc{es} \textsc{11} \textsc{\%} \textsc{(2/18)}}\\
{\textsc{bl} \textsc{55} \textsc{\%} \textsc{(10/18)}} & {\textsc{bl} \textsc{66} \textsc{\%} \textsc{(12/18)}} & {\textsc{bl} \textsc{55} \textsc{\%} \textsc{(10/18)}} & {\textsc{te} \textsc{5} \textsc{\%} \textsc{(1/18)}}\\
&  & {\textsc{te} \textsc{28} \textsc{\%} \textsc{(5/18)}} & {\textsc{bl} \textsc{50} \textsc{\%} \textsc{(9/18)}}\\
\lspbottomrule
\end{tabularx}
\end{table}

En resumen, podemos observar que existe una relación del uso de la lengua indígena con ámbitos que están relacionados a la vida comunitaria tales como las fiestas y el hogar; la presencia del español y de la lengua indígena en la comunidad de El Espíritu implica que no hay un desplazamiento total de la lengua, mientras que en Misión de Chichimecas esta co-aparición de las lenguas puede sugerir que el español empiece a abarcar espacios comunitarios en los que se esperaría el chichimeca. Por su parte en Santa María de Ocotán, el uso bilingüe parece implicar un uso más equilibrado de las dos lenguas y éste podría estar sujeto al interlocutor. Finalmente, en relación con el español son los ámbitos que están fuera de la comunidad o bien en los que se tiene una interacción con hispanohablantes. Destaca el hecho que, mientras que en El Espíritu el español tiene los porcentajes de uso más altos, en Santa María de Ocotán presenta los más bajos, mostrándonos la estrecha relación que tienen la realidad sociolingüística de la comunidad con los usos de las lenguas.


 \section{Reflexiones finales}


El presente estudio, plantea el uso de la noción de mercado y \textit{habitus lingüístico} con el fin de describir los ámbitos de uso de las lenguas en una comunidad multilingüe, ya que permite esquematizar las distintas decisiones que toman los hablantes de una comunidad en relación con el uso de las lenguas en cuestión y comparar para encontrar patrones de usos en sus diferencias y semejanzas. Asimismo, nos muestra la estrecha relación que existe entre los usos y las distintas realidades sociolingüísticas. Esto, se mostró en el análisis de tres comunidades mexicanas que forman un \textit{contínuum} que va de la obsolescencia (El Espíritu), el inicio de desplazamiento (Misión de Chichimecas) y la vitalidad de la lengua indígena (Santa María de Ocotán). Por lo tanto, observamos tres mercados lingüísticos diferentes en los que el español y la lengua indígena tienen valores de cambio distintos.

Así, la comunidad de El Espíritu refleja un menor uso de la lengua indígena pero no su total desplazamiento, pues pese a que el español está presente en todos los ámbitos, el uso bilingüe en algunos de ellos implica la permanencia del otomí y su legitimidad en ciertos actos de habla; por su parte la comunidad de Misión de Chichimecas representaría un estado intermedio frente a las otras dos comunidades, pues en ella el chichimeca sigue teniendo una fuerte presencia a pesar que el español empieza a ser usado en contextos en los que se esperaría la lengua indígena. Finalmente, la comunidad de Santa María de Ocotán sería el extremo contrario de El Espíritu, pues los miembros de esta comunidad prefieren el uso de la lengua indígena, ya sea en solitario o en co-aparición con el español. La presencia del español es muy baja y se especializa en contextos en los que hay presencia de hispanohablantes.

  Asimismo, se exhibió que la fiesta, la casa y la iglesia (para el caso del tepehuano del sur) son los ámbitos que los hablantes mencionan usar más la lengua indígena. Estos, son espacios en los que se destacan los vínculos de interpretación de la cosmovisión y la costumbre de los grupos indígenas estudiados. Mientras que, la junta o asambleas parece ser un espacio en el que el uso de las dos lenguas es común en las tres comunidades, esto parece tener relación con el hecho de que es en estos ámbitos en los que se tocan tanto temas relacionados a festividades y gobierno tradicional, como con proyectos gubernamentales en los que es necesario el uso del español, o bien el uso de las dos lenguas mezcladas por la ausencia de términos necesarios\footnote{\citet[922]{Zimmermann2010} expone esta situación para el otomí de los años ochenta en el que “situaciones sobre todo públicas se puede observar la alternancia de códigos (code switching) y la transferencia del español al otomí”}.

En relación con la presencia del español son los ámbitos que están más alejados de la vida comunitaria, o que incluso están fuera de ella como el caso de la ciudad y el mercado, en los que los hablantes prefieren su uso. Asimismo, también destacan los espacios en los que hay interacción con hispanohablantes\footnote{\citet[928]{Zimmermann2010}, expone el caso de la comunidad zapoteca de Rincón Juárez, caracterizada por tener un bilingüismo colectivo pues a finales del siglo XX pues el 80\% de la comunidad era bilingüe; menciona que el uso del español está especializado para contextos formales e institucionales, tales como la “escuela, la iglesia, el centro de salud, ocasionalmente el mercado y las asambleas municipales [...] Estos dominios son a la vez los espacios intracomunitarios de encuentro con personas provenientes de fuera de la comunidad. Mientras que, en la escuela, el factor determinante es el sistema y la orientación educativa, en el centro de salud, la iglesia y el mercado es la falta de conocimiento del zapoteco por parte de los interlocutores”.}.

Queda pendiente en el análisis etnográfico de estos ámbitos en estas tres comunidades, específicamente en los usos bilingües, para observar la selección de recursos lingüísticos de las lenguas en contacto que hacen los colaboradores en relación con los interlocutores.

  Finalmente, es importante destacar la implementación de instrumentos que nos permitan hacer estudios comparativos en las diferentes comunidades bilingües del país, con el fin de poder tener un mayor conocimiento de su realidad sociolingüística y de las decisiones que toman los hablantes con base en esta última.


\begin{verbatim}%%move bib entries to  localbibliography.bib
@book{AlonsoBenito2004,
	address = {In Pierre Bourdieu : las herramientas del sociólogo, 215–254. Navarra},
	author = {Alonso Benito, Luis Enrique},
	note = {(17 August, 2020).},
	publisher = {Universidad de Navarra},
	title = {Pierre Bourdieu, el lenguaje y la comunicación: {{D}}el análisis de los mercados lingüísticos a la denuncia de la degradación mediática},
	year = {2004}
}


@article{BlommaertSlembrouck2005,
	author = {Blommaert, Jan, James Collins and Stef Slembrouck},
	journal = {Language & Communication (Multilingualism and Diasporic Populations)},
	doi = {10.1016/j.langcom.2005.05.002},
	number = {3},
	pages = {197–216},
	title = {Spaces of multilingualism},
	volume = {25},
	year = {2005}
}


@book{Bourdieu1990,
	address = {1a edición en francés 1984. México},
	author = {Bourdieu, Pierre},
	publisher = {Grijalbo},
	title = {Sociología y cultura},
	year = {1990}
}


@book{Cifuentes1998,
	address = {Multilingüismo a través de la historia. México},
	author = {Cifuentes, Bárbara},
	publisher = {Centro de Investigaciones y Estudios Superiores en Antropología Social & Instituto Nacional Indigenista},
	sortname = {Cifuentes, Barbara},
	title = {Letras sobre voces},
	year = {1998}
}


@incollection{DelaTorreCuriel2006,
	address = {México},
	author = {De la Torre Curiel, José Refugio},
	booktitle = {{La} sierra tepehuana. {{A}}sentamientos y movimientos de población},
	editor = {Chantal Cramaussel and Ortelli Ortelli},
	pages = {147–162},
	publisher = {El Colegio de Michoacán y la Universidad Juárez del Estado de Durango},
	sortname = {De la Torre Curiel, Jose Refugio},
	title = {{La} presencia franciscana en las misiones del sur de la sierra tepehuana},
	year = {2006}
}


@misc{Ferguson1959,
	author = {Ferguson, Charles},
	note = {Word (15). 325–340.},
	title = {Diglossia},
	year = {1959}
}


@incollection{FloresFarfán1998,
	address = {Amsterdam-Atlanta},
	author = {Flores Farfán, José Antonio},
	booktitle = {Foro hispánico. {{S}}ociolingüística: {{L}}enguas en contacto},
	editor = {Pieter Muysken},
	note = { (17 August, 2020).},
	pages = {75–86},
	publisher = {Rodopi},
	sortname = {Flores Farfan, Jose Antonio},
	title = {Hablar cuatrapeado: {{{E}}}n torno al español de los indígenas mexicanos},
	year = {1998}
}


@book{GuerreroGalván2009,
	address = {Instrumento generado dentro del proyecto Variación y normatividad en las lenguas otopames: Cambio fonológico en el contexto de la sistematización ortográfica 2009-2012. Manuscrito. México},
	author = {Guerrero Galván, Alonso},
	note = {(https://linguistica.inah.gob.mx/index.php/pro/27-normatividad-y-variacion-en-lenguas-otopames).},
	publisher = {Instituto Nacional de Antropología e Historia, ms},
	sortname = {Guerrero Galvan, Alonso},
	title = {Encuesta para peritaje lingüístico},
	year = {2009}
}


@book{GuerreroGalván2013,
	address = {México},
	author = {Guerrero Galván, Alonso},
	publisher = {El Colegio de México Tesis de Doctorado},
	sortname = {Guerrero Galvan, Alonso},
	title = {Fonología histórica del otomí, siglos {XVI} al {XIX}},
	year = {2013}
}


@incollection{GuerreroGalvánGiacomo2014,
	address = {México},
	author = {Guerrero Galván, Alonso and Marcela San Giacomo},
	booktitle = {Historia sociolingüística de México. {{{E}}}spacio, contacto y discurso político},
	editor = {Rebeca Barriga Villanueva and Pedro Martín Butragueño},
	pages = {1457–1524},
	publisher = {El Colegio de México},
	sortname = {Guerrero Galvan, Alonso and Marcela San Giacomo},
	title = {El llamado español indígena en el contexto del bilingüismo},
	year = {2014}
}


@book{Hekking1995,
	address = {Amsterdam},
	author = {Hekking, Ewald},
	note = {(17 August, 2020).},
	publisher = {IFOTT},
	title = {El otomí de {Santiago} Mexquititlán: {{D}}esplazamiento linguístico, préstamos y cambios gramaticales},
	year = {1995}
}


@book{HillHill1986,
	address = {Arizona},
	author = {Hill, Jane H and Kenneth Hill},
	publisher = {The University of Arizona Press},
	title = {Speaking mexicano: {{D}}ynamics of syncretic language in {Central} {Mexico}},
	year = {1986}
}


@book{Lastra2003,
	address = {México},
	author = {Lastra, Yolanda},
	publisher = {El Colegio de México, Centro de Estudios Lingüísticos y Literarios},
	title = {Sociolingüística para hispanoamericanos: {{{U}}}na introducción},
	year = {2003}
}


@book{Lastra2006,
	address = {1. ed. México, D.F},
	author = {Lastra, Yolanda},
	publisher = {Universidad Nacional Autónoma de México, Instituto de Investigaciones Antropológicas},
	title = {Los otomíes: {{S}}u lengua y su historia},
	year = {2006}
}


@incollection{Levy1990,
	address = {México},
	author = {Levy, Paulette},
	booktitle = {Estudios de lingüística de España y México},
	editor = {Violeta Demonte and Beatriz Garza Cuarón},
	pages = {551–559},
	publisher = {El Colegio de México},
	title = {Un caso de interferencia sintáctica del español en totonaco},
	year = {1990}
}


@incollection{Niño-MurciaRothman2008,
	address = {Amsterdam},
	author = {Niño-Murcia, Mercedes and Jason Rothman},
	booktitle = {{Spanish}-contact bilingualism and identity},
	editor = {Mercedes Niño-Murcia and Jason Rothman},
	note = { https://doi.org/10.1075/sibil.37.03nin (17 August, 2020).},
	pages = {11–32},
	publisher = {John Benjamins Publishing Company},
	sortname = {Nino-Murcia, Mercedes and Jason Rothman},
	title = {{Spanish}-contact bilingualism and identity},
	volume = {37},
	year = {2008}
}


@article{Pfeiler1988,
	author = {Pfeiler, Bárbara},
	journal = {Estudios de cultura maya},
	pages = {423–444},
	sortname = {Pfeiler, Barbara},
	title = {Yucatán: {{{E}}}l uso de dos lenguas en contacto},
	volume = {17},
	year = {1988}
}


@article{RequenaSantos1989,
	author = {Requena Santos, Félix},
	journal = {Reis},
	pages = {137–152},
	sortname = {Requena Santos, Felix},
	title = {El concepto de red social},
	volume = {48},
	year = {1989}
}


@book{ReyesValdez2006,
	address = {Serie Estudios Monográficos). 1. ed. México, D.F},
	author = {Reyes Valdez, Jorge Antonio},
	publisher = {Instituto Nacional de Antropología e Historia},
	title = {Los que están benditos: {{{E}}}l mitote comunal de los tepehuanes de {Santa} María de Ocotán, Durango (Colección {E}tnografía de Los Pueblos Indígenas de México},
	year = {2006a}
}


@book{ReyesValdez2006,
	address = {1. ed. México, D.F},
	author = {Reyes Valdez, Jorge Antonio},
	publisher = {Comisión Nacional para el Desarrollo de los Pueblos Indígenas},
	title = {Tepehuanes del sur (Pueblos Indígenas Del México Contemporáneo)},
	year = {2006b}
}


@book{Silva-CorvalánEnrique-Arias2001,
	address = {Washington, DC},
	author = {Silva-Corvalán, Carmen and Andrés Enrique-Arias},
	publisher = {Georgetown University Press},
	sortname = {Silva-Corvalan, Carmen and Andres Enrique-Arias},
	title = {Sociolingüística y pragmática del español (Georgetown Studies in {Spanish} Linguistics)},
	year = {2001}
}


@book{SmithStark2007,
	address = {UniverSOS},
	author = {Smith Stark, Thomas},
	note = {Servei de Publicacions (4). 9–39.},
	publisher = {revista de lenguas indígenas y universos culturales},
	title = {Los préstamos entre el español y el zapoteco de {San} Baltasar Chichicapan},
	year = {2007}
}


@book{Soustelle1990,
	address = {1. ed. México, D.F},
	author = {Soustelle, Jacques},
	publisher = {Fondo de Cultura Económica},
	title = {{La} familia otomí-pame del México {Central} (Sección de obras de historia)},
	year = {1990}
}


@book{Suárez1977,
	address = {Anuario de Letras},
	author = {Suárez, Jorge},
	note = {Universidad Nacional Autónoma de México (15). 115–164.},
	publisher = {Lingüística y filología},
	sortname = {Suarez, Jorge},
	title = {{La} influencia del español en la estructura gramatical del náhuatl},
	year = {1977}
}


@book{TorresSánchez2018,
	address = {México},
	author = {Torres Sánchez, Nadiezdha},
	publisher = {El Colegio de México Tesis de Doctorado},
	sortname = {Torres Sanchez, Nadiezdha},
	title = {Aquí hablamos tepehuano y allá español : {{{U}}}n estudio de la situación de bilingüismo incipiente entre español y tepehuano del sureste (o’dam) en {Santa} María de Ocotán y Durango},
	year = {2018}
}


@article{Villavicencio2006,
	author = {Villavicencio, Frida},
	journal = {Predicación nominal en purépecha y español. Tópicos del Seminario},
	pages = {159–195},
	title = {Estructuras gramaticales en contacto},
	volume = {15},
	year = {2006}
}


@book{Zimmermann1987,
	address = {Préstamos gramaticalmente relevantes del español al otomí},
	author = {Zimmermann, Klaus},
	note = {Anuario de Lingüística Hispánica. Servicio de Publicaciones (3). 223–253.},
	publisher = {una aportación a la teoría del contacto entre lenguas},
	year = {1987}
}


@incollection{Zimmermann2010,
	address = {México},
	author = {Zimmermann, Klaus},
	booktitle = {Historia Sociolingüística de México},
	editor = {Rebeca Barriga Villanueva and Pedro Martín Butragueño},
	pages = {881–955},
	publisher = {El Colegio de México},
	title = {Diglosia y otros usos diferenciados de lenguas y variedades en el México del siglo {XX}: {{{E}}}ntre el desplazamiento y la revitalización de las lenguas indomexicanas},
	volume = {2},
	year = {2010}
}

\end{verbatim}
\sloppy\printbibliography[heading=subbibliography,notkeyword=this]
\end{document}
