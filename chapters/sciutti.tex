\documentclass[output=paper,modfonts,nonflat,newtxmath]{langsci/langscibook}
\author{Sandro Sciutti\affiliation{University of Genoa}}
\title{The acquisition of clitic pronouns in complex infinitival clauses by German-speaking learners of Italian as an L3: The role of proficiency in target and background language(s)}
\abstract{This paper reports the findings of a study on the acquisition of clitic pronouns in complex infinitival clauses by 20 German-speaking learners of Italian as a third language, German being a language devoid of clitics. The subjects have been grouped on the basis of their proficiency in the target language (intermediate or advanced) and their knowledge of a Romance background language (French or Spanish). Those learners who had knowledge of French were further subdivided into low and high proficiency with respect to second language (L2) French. The same was not possible for learners with a background in Spanish, because of the absence of subjects with low proficiency in L2 Spanish. Performances in three experimental tests – an elicited production, a grammaticality judgment/correction task and a written translation from German into Italian – have been comparatively analysed to determine whether the acquisition of Italian clitics in complex clauses containing an infinitive is in any way affected by 1) proficiency in L3 Italian; 2) the specific Romance L2 (French vs. Spanish) and 3) proficiency in L2 French. In particular, the analysis has focused on three main aspects: 1) overall clitic production and adoption of avoidance strategies; 2) production of specific clitic categories (primarily partitive and locative clitics); 3) clitic placement.}
%{Keywords}: {clitic} {pronouns,} {L3} {acquisition,} {infinitival} {clauses,} {proficiency,} {Romance} {background} {languages}


\shorttitlerunninghead{L3 Italian acquisition of clitics by German-speaking learners}
\begin{document}
\maketitle
\newpage
\section{Introduction}%S. 1
\label{sec:sciutti:1}

Complement clitic pronouns have been repeatedly shown as one of the trickiest grammatical features to acquire among all learner populations (e.g., \citealt{BellettiGuasti2015}). This fact is supported by a wide array of empirical research carried out on all kinds of subjects: normally developing children, children with Specific Language Impairment, bilingual children, early and adult second language (L2) learners. When it comes to Italian, this general difficulty is compounded by the fact that clitic placement is fairly complex in infinitival clauses, since clitics can occur either in an enclitic position in relation to the infinitive or in a proclitic position with regard to the finite verb governing the infinitive, as a result of a climbing movement. For this reason, L2 acquisition of Italian clitics – as is the focus of this article – can prove rather demanding.

This study was carried out within the framework of the wide-ranging research field of multilingualism and, more specifically, of additional or Third Language Acquisition (TLA or L3A), an L3 being a non-native language acquired after one or more L2s, in line with the definition given by \citet[97]{Hammarberg2010}. Some parallel terms used in the literature are those of Ln, subsequent language, additional language, multilingual acquisition and multiple language acquisition. From this viewpoint, Italian as a non-native language, particularly abroad, is mostly learnt as an L3, after learners have already been confronted with other non-native language learning experiences: it is more often than not the case of English, but also of another Romance language like Spanish or French.

As far as clitic acquisition in non-native Italian is concerned, the potential role of another non-native language that has clitics remains to be further explored, as does the effect of proficiency level in that other non-native language \citep[238--239]{Giannini2008}. French and Spanish as “bridge” languages between a first language (L1) and L3 Italian may provide interesting matter for comparison in this sense, although both are equipped with clitics, they differ in terms of both repertoire and placement in complex infinitival clauses. In terms of level of proficiency in the background Romance language, the consequences of its variation on the acquisition of Italian as an L3 must be thoroughly investigated, in order to ascertain whether a sound mastery in an L2 results in a higher degree of crosslinguistic influence between L2 and L3.

\section{Crosslinguistic information}%S. 2
\label{sec:sciutti:2}

As the current study looks at German-speaking learners of L3 Italian with prior knowledge of another Romance language (Spanish or French), it is useful to provide some information about pronouns in the languages involved in the research, in terms of both repertoire and syntax within infinitival clauses.

\subsection{{Repertoire} {of} {clitic} {pronouns} {in} {Italian,} {French} {and} {Spanish}}%2.1.
\label{sec:sciutti:2.1}

Like all Romance languages, Italian, French and Spanish possess accusative, dative and reflexive clitic pronouns. In addition, both Italian and French have partitive and locative clitics, unlike Spanish. Partitive clitics – like the pronoun \textit{ne} in Italian – are typically used to replace a noun embedded in a determiner phrase (DP) containing an expression of quantity, as in the following example:


\ea \label{ex:sciutti:1}
\begin{xlist}
    \ex \label{ex:sciutti:1a}
    \gll Ho incontrato dieci ragazz-i.\\
        have.\textsc{1sg} met ten boy-\textsc{m.pl}\\
    \glt  ‘I have met ten boys.’

    \ex \label{ex:sciutti:1b}
    \gll Ne ho incontrat-i dieci.\\
        Thereof have.\textsc{1sg} met-\textsc{m.pl} dieci \\
    \glt ‘I met ten of them.’

\end{xlist}
\z


Locative clitics, for their part, represent a prepositional phrase with the function of a locative complement, like the Italian \textit{ci} in the example below:

\ea \label{ex:sciutti:2}
\begin{xlist}
    \ex \label{ex:sciutti:2a}
    \gll Ho abitato due anni a Firenze.\\
        have.\textsc{1sg} lived two years at Florence\\
    \glt  ‘I lived two years in Florence.’

    \ex \label{ex:sciutti:2b}
    \gll Ci ho abitato due anni.\\
        There have.\textsc{1sg} lived two years\\
    \glt ‘I lived there two years.’
\end{xlist}
\z



\subsection{Pronominal syntax in infinitival clauses} %{2.2.}
\label{sec:sciutti:2.2}

\subsubsection{Italian} %{2.2.1.}

The infinitival clauses dealt with in the study are of different kinds and are characterised by varied rules as far as clitic placement is concerned. They include:

\begin{itemize}
\item Clauses in which the infinitive is governed by either a preposition or a predicate adjective. In these clauses clitics do not climb, as restructuring does not obtain:

\ea \label{ex:sciutti:3}
    \gll Ho dimenticato di restituir=lo a Giacomo.\\
        have.\textsc{1sg} forgotten of give\_back=it.\textsc{m.sg.acc} to Giacomo\\
    \glt  ‘I have forgotten to give it back to Giacomo.’
\ex \label{ex:sciutti:4}
    \gll Non è facile trovar=ne uno economico.\\
         not is easy find=thereof one cheap\\
    \glt  ‘It is not easy to find a cheap one.’
\z

\item Clauses in which the infinitive is governed by a modal, motion or aspectual verb. In these clauses clitics may climb as a result of restructuring. Unlike modal verbs, motion and aspectual verbs are followed by a preposition:

\ea \label{ex:sciutti:5}
    \begin{xlist}
    \ex \label{ex:sciutti:5a}
    \gll Vorrei prender=ne un’ altra.\\
         would\_like.\textsc{1sg} take=thereof an.\textsc{f.sg} other.\textsc{f.sg}\\
    \glt  ‘I would like to have another one.’

    \ex \label{ex:sciutti:5b}
    \gll Ne vorrei prendere un’ altra.\\
         thereof would\_like.\textsc{1sg} take an.\textsc{f.sg} other.\textsc{f.sg}\\
    \glt  ‘I would like to have another one.’
    \end{xlist}
\ex \label{ex:sciutti:6}
    \begin{xlist}
    \ex \label{ex:sciutti:6a}
    \gll I miei genitori sono andati ad abitar=ci.\\
         the my parents are gone to live=there\\
    \glt  ‘My parents moved there.’

    \ex \label{ex:sciutti:6b}
    \gll I miei genitori ci sono andati ad abitare.\\
         the my  parents there are  gone  to live\\
    \glt  ‘My parents moved there.’
    \end{xlist}
\ex \label{ex:sciutti:7}
    \begin{xlist}
    \ex \label{ex:sciutti:7a}
    \gll Ho iniziato a legger=lo una settimana fa.\\
         have.\textsc{1sg} started to read=it.\textsc{m.sg.acc} one  week  ago\\
    \glt  ‘I started reading it a week ago.’
    \ex \label{ex:sciutti:7b}
    \gll L’ ho iniziato a leggere una settimana     fa.\\
         \textsc{m.sg.acc} have.\textsc{1sg} started to read one week ago\\
    \glt  ‘I started reading it a week ago.’
    \end{xlist}
\z

\item Causative and perceptive clauses, in which the infinitive is governed by a causative (\textit{fare}, \textit{lasciare}) or a perceptive verb (\textit{vedere}, \textit{sentire}) and is not preceded by a preposition. In these clauses clitics obligatorily climb, as restructuring occurs:

\ea \label{ex:sciutti:8}
    \gll Lo faccio lavare in lavanderia.\\
        it.\textsc{m.sg}   make.\textsc{1sg} wash in laundry\\
    \glt  ‘I’ll have it washed in the laundry.’
\ex \label{ex:sciutti:9}
    \gll Non la vedo arrivare.\\
       not it.\textsc{f.sg.acc}   see.\textsc{1sg} arrive\\
    \glt  ‘I cannot see it approaching.’
\z
\end{itemize}


\subsubsection{German}%{2.2.2.}

In German infinitival clauses, pronouns tend to occur to the left, while the infinitive is found at the end of the clause, preceded or not by the preposition \textit{zu}, as shown in the examples \REF{ex:sciutti:10} to \REF{ex:sciutti:13} below. If the infinitive is not preceded by \textit{zu} (e.g. when the infinitive is governed by a modal, a causative or a perceptive verb) and the tense of the finite verb governing the infinitive is compound, the past participle of that verb is replaced by an infinitive, which occupies the last clausal position, i.e. following the infinitive governed by the finite verb (as in \ref{ex:sciutti:14} and \ref{ex:sciutti:15}):

\ea \label{ex:sciutti:10}
    \gll Ich habe vergessen, ihn anzurufen.\\
        I have forgotten, him.\textsc{m.sg.acc}  phone\\
    \glt  ‘I have forgotten to phone him.’
\ex \label{ex:sciutti:11}
    \gll Ich  will ihm dieses Buch schenken.\\
        I want him.\textsc{m.sg.dat} this  book  give \\
    \glt  ‘I would like to give him this book.’
\ex \label{ex:sciutti:12}
    \gll Ich gehe mich waschen.\\
        I go myself.\textsc{acc} wash \\
    \glt  ‘I’ll go and wash myself.’
\ex \label{ex:sciutti:13}
    \gll Ich habe angefangen, es zu lesen.\\
        I have started it.\textsc{n.sg.acc} to read \\
    \glt  ‘I have started reading it.’
\ex \label{ex:sciutti:14}
    \gll Sie haben mich nicht hinein-gehen lassen.\\
         they have me.\textsc{acc}   not   therein-go let\\
    \glt  ‘They did not let me in.’
\ex \label{ex:sciutti:15}
    \gll Ich habe ihn hinaus-gehen sehen.\\
         I have him.\textsc{m.sg.acc} out-go see\\
    \glt  ‘I saw him go out.’
\z

\subsubsection{French} %{2.2.3.}

In French infinitival clauses, clitics occur between the finite verb and the infinitive. This holds true both for clauses which could restructure in Italian (\ref{ex:sciutti:17} to \ref{ex:sciutti:19}) and for those which would not \REF{ex:sciutti:16}:

\ea \label{ex:sciutti:16}
    \gll J’ ai oublié de l’ appeler.\\
         I have forgotten of him.\textsc{m.sg.acc} call \\
    \glt  ‘I have forgotten to call him.’
\ex \label{ex:sciutti:17}
    \gll Je veux lui offrir ce livre.\\
         I want him.\textsc{m.sg.dat} give   this.\textsc{m.sg} book \\
    \glt  ‘I would like to give him this book.’
\ex \label{ex:sciutti:18}
    \gll Je vais me laver.\\
         I go myself.\textsc{acc} wash\\
    \glt  ‘I’ll go and wash myself.’
\ex \label{ex:sciutti:19}
    \gll J’ ai commencé à le lire.\\
         I have started to it.\textsc{m.sg.acc}    read\\
    \glt  ‘I started reading it.’
\z

Causative and perceptive clauses, however, are characterised by clitic climbing, just as in Italian:

\ea \label{ex:sciutti:20}
    \gll Ils ne m’ ont pas laissé entrer.\\
         they not me.\textsc{acc} have not let enter \\
    \glt  ‘They did not let me in.’
\ex \label{ex:sciutti:21}
    \gll Je l’ ai vu sortir.\\
         I him.\textsc{m.sg.acc} have seen go\_out \\
    \glt  ‘I saw him go out.’
\z

\subsubsection{Spanish} %\textbf{2.2.4.}

In Spanish infinitival clauses, clitic placement always mirrors the one found in Italian:%\todo{not sure sentence in ex. 21 is grammatical, better: me he olvidado de llamarlo}

\ea \label{ex:sciutti:22}
    \gll He olvidado de llamar=lo.\\
         have.\textsc{1sg} forgotten of call=him.\textsc{m.sg.acc} \\
    \glt  ‘I saw him go out.’
\ex \label{ex:sciutti:23}
    \begin{xlist}
    \ex \label{ex:sciutti:23a}
    \gll Quiero regalar=le este libro.\\
         want.\textsc{1sg} give=\textsc{sg.dat} this  book\\
    \glt  ‘I would like to give him/her this book.’

    \ex \label{ex:sciutti:23b}
    \gll Le quiero regalar este libro.\\
         \textsc{sg.dat} want.\textsc{1sg} give this book\\
    \glt  ‘I would like to give him/her this book.’
    \end{xlist}
\ex \label{ex:sciutti:24}
    \begin{xlist}
    \ex \label{ex:sciutti:24a}
    \gll Voy a lavar=me.\\
         go.\textsc{1sg} to wash=myself.\textsc{acc}\\
    \glt  ‘I’ll go and wash myself.’

    \ex \label{ex:sciutti:24b}
    \gll Me voy a lavar.\\
         myself.\textsc{acc} go.\textsc{1sg} to wash\\
    \glt  ‘I’ll go and wash myself.’
    \end{xlist}
\ex \label{ex:sciutti:25}
    \begin{xlist}
    \ex \label{ex:sciutti:25a}
    \gll He empezado a leer=lo.\\
         have.\textsc{1sg} started to read=it.\textsc{acc.m.sg}\\
    \glt  ‘I have started reading it.’

    \ex \label{ex:sciutti:25b}
    \gll Lo he empezado a leer.\\
         it.\textsc{acc.m.sg} have.\textsc{1sg} started to read\\
    \glt  ‘I have started reading it.’
    \end{xlist}
\ex \label{ex:sciutti:26}
    \gll No me dejaron entrar.\\
         not me.\textsc{acc} let.\textsc{3pl.pret} enter \\
    \glt  ‘They did not let me in.’
\ex \label{ex:sciutti:27}
    \gll Lo he visto salir.\\
         him.\textsc{m.sg.acc} have.{1SG} seen go\_out \\
    \glt  ‘I saw him go out.’
\z


\section{{Previous} {research} {on} {L2} {Italian} {clitic} acquisition} %3
\label{sec:sciutti:3}

Several studies have been devoted to the acquisition of pronouns, most notably clitics, in L2 Italian. \citet{Berretta1986} carried out what can be considered in many respects a pioneering work, followed after several years by research by \citet{LeoniniBelletti2004, GianniniCancila2006, Santoro2007, Giannini2008} and \citet{Maffei2009}. These studies share some common findings.

To start with, it can be said that the L2 acquisition of Italian clitic pronouns~– and more in general that of the clitics of the Romance languages – involves a rather slow and difficult process \citet{BruhnMontrul1996, White1996, DuffieldWhite1999, DuffieldEtAl2002, Santoro2007}. Indeed, in her study, \citet{Giannini2008} found out that the omission rate is high even in advanced learning stages, above all in elicited productions. This finding may reflect an avoidance strategy, which may also encompass the replacement of clitics with lexical DPs or, to a much lesser degree, with strong pronouns \citet{LeoniniBelletti2004}. Similarly, \citet[191]{ChiniEtAl2003}, in comparing the use of clitics as textual anaphoric devices by L1 and L2 Italian speakers, noticed that they are much more resorted to by native speakers than by L2 Italian learners with different L1s. Sheer omission seems to be the most widespread alternative to clitic production in the acquisition of Italian as an L2, its rate ranging from 9\% to 20\% at most \citet{BellettiGuasti2015}. The use of a lexical DP in a context in which a native speaker would use a clitic is a feature distinguishing L2 from L1 acquisition: whereas it is a strategy quite often employed by L2 learners, it is very rarely found in native language acquisition (\citealt{BellettiGuasti2015}). For instance, \citet{LeoniniBelletti2004} report that their L2 subjects produced a lexical DP in 40\% of cases, against 7.7\% in the control groups. Likewise, the German-speaking subjects investigated by \citet{Leonini2006} used a lexical DP in 52\% of cases on average. More in detail, the recourse to this strategy decreased as proficiency in Italian increased: lexical DPs were produced in 69\% of cases by subjects with intermediate proficiency in Italian, in 49\% of cases by advanced learners and in 32\% by near-natives. As for the use of strong pronouns in place of clitics, it is something which occurs very rarely in the L2 acquisition of Italian (\citealt{BellettiGuasti2015}).

The relatively slow acquisition of clitics in comparison with other grammar structures is possibly due to the structural complexity of the cliticisation phenomenon and may, therefore, point to a difficulty with processing on the part of the learners. This is maintained by \citet{BellettiGuasti2015}, who pinpoint the special and complex morphosyntax of clitics as one of the reasons why they tend to be avoided in the early L2 productions. \citet{BottariEtAl2000} identify a series of possible reasons accounting for the difficulty in acquiring clitics (i.e. phonological saliency, argument structure, control, morphological paradigms and syntactic representation), in particular in the field of chain formation. Another reason might be the markedness of clitics from a typological point of view, as highlighted by \citet[329]{Berretta1986}.

As to the mistakes made by L2 learners, in most cases, they seem to concern more the morphological features of clitics – number, gender and case – than the syntactic ones (i.e. those related to their placement with respect to the verb). In other words, the former appear to be acquired more slowly and less accurately than the latter. However, some scholars counsel caution in this regard as placement errors amongst L2 learners do not seem to be utterly negligible, as reported with French clitics \citep{White1996, HulkMuller2000, BellettiHamann2004, GranfeldtSchlyter2004, HamannBelletti2006}.

\citet{HamannBelletti2006} suggest that misplacement errors found in the L2 acquisition of French clitics may be due to an initial interpretation of complement clitics as weak pronouns, instantiated as nominative phonological clitics in French. The smaller presence of weak pronouns in Italian might account for the reported lack of misplacement errors in Italian \citep{BellettiGuasti2015}. However, as rare as they may be, misplacement errors in the L2 acquisition of Italian clitics are not totally lacking. For instance, in her research on the acquisition of personal pronouns in L2 Italian, based on a sample of ADIL2 corpus made up of 823 informants with 27 different L1s, \citet[116]{Maffei2009} mentions cliticisation with imperatives, infinitives, compound tenses and modal verbs amongst the placement errors:

\ea \label{ex:sciutti:28}
    \gll Spero che tu non avrai più paura di ti risvegliare.\\
         hope.\textsc{1sg} that you not have.\textsc{fut.2sg} more fear of yourself.\textsc{acc} wake\\
    \glt  ‘I hope that like this one you will be no more afraid of waking up.’
\z

It must be pointed out that some studies on the acquisition of clitics in L2 Italian focused on less complex clauses than, say, the infinitival ones, such as those with a single finite verb conjugated in a simple or compound tense, exhibiting only proclisis (\citealt{LeoniniBelletti2004, Leonini2006}). Turning the attention more closely to instances of pronominal enclisis or to those subject to clitic climbing phenomena, misplacement errors are not entirely negligible. For instance, in the research carried out by \citet{GianniniCancila2006} on nine English-speaking subjects, such phrasal contexts are characterised by the highest rate of mistakes and omissions by the learners. The authors put this result down to the high computational load required by clauses stemming from a twofold syntactic movement. In the same vein, in the corpus collected by \citet{Berretta1986}, the otherwise very rare syntactic mistakes to be found are linked with complex infinitival clauses subject to restructuring:

\ea[*]{\label{ex:sciutti:29}
    \gll Vuoi=mi dare il libro, per piacere?\\
         want.\textsc{2sg}=me\textsc{dat} give the  book,   for    pleasure\\
    \glt  ‘Would you please give me the book?’}
\ex [*]{\label{ex:sciutti:30}
    \gll Voglio far dir=gli=la.\\
         want.\textsc{1sg} make say=\textsc{sg.dat}=it\textsc{f.sg.acc}\\
    \glt  ‘I want to get him/her to say it.’}
\z


Similar results are found in the research carried out by \citet{BennatiMatteini2006} on eighteen L2 Italian learners with different L1s (English, German and Spanish): compared with the almost total lack of placement errors in clauses with a finite verb conjugated in a compound tense, the syntactic mistake rate is far higher in complex infinitival clauses subject to restructuring, such as causative clauses and clauses made up of an infinitive governed by a modal or a motion verb. Similar data can be found in the longitudinal study carried out by \citet{Ferrari2006} on two Italian/German bilingual children (see also \citealt{BernardiniTimofte2017} and \citealt{BernardiniWeijer2017} for comparable results). Here no syntactic mistakes are reported in clauses made up of a single finite verb (exhibiting pronominal proclisis), in complex infinitival clauses not subject to restructuring and in imperative clauses (both showing pronominal enclisis); however, clitics are often placed – in 63\% and 27\% of the instances found in the two children’s productions – between the finite verb and the infinitive in complex infinitival clauses containing a modal verb (where restructuring obtains):

\ea[*]{\label{ex:sciutti:31}
    \gll Anche tu vuoi lo mangiare?\\
         also you want.\textsc{2sg}   it.\textsc{m.sg.acc} eat?\\
    \glt  ‘Do you want to eat it, too?’}
\ex [*]{\label{ex:sciutti:32}
    \gll Voglio mi mettere.\\
         want.\textsc{1sg} myself.\textsc{acc} put\\
    \glt  ‘I want to put myself.’}
\z

Finally, \citet[58]{Corino2012} too points out that, in the corpus she analysed, the clitic placement errors are mostly found in connection with compound tenses, infinitives and imperatives:

\ea[*]{ \label{ex:sciutti:33}
    \gll  Mi sembra-va che lui fosse in grado di si         liberare.\\
        me.\textsc{dat}   seemed-\textsc{ipfv}   that   he was in degree  of himself.\textsc{acc}  free\\
    \glt  ‘It seemed to me that he was able to free himself.’}
\z

As for the role of the learners’ prior linguistic knowledge in the acquisition of Italian clitics, whereas that of the L1 has already been investigated in some studies (\citealt{Berretta1986, LeoniniBelletti2004, BennatiMatteini2006, GianniniCancila2006,Giannini2008, Maffei2009}), that of a Romance L2 is still partially unexplored, thus providing the opportunity for further research. However, at least two findings are worth reporting.

The first stems from the research carried out by \citet{LeoniniBelletti2004} on twenty-six L2 Italian learners with different L1s (German, French, Polish, Dutch, Russian, Greek, Albanian and Bosnian). A German-speaking subject in this study (referred to as subject no. 15), reveals, in acquiring Italian clitics, a very different behaviour from the rest of the German-speaking subjects, even though he was exposed to Italian – at first in his home country and then in Italy – not as long as the other subjects with the same L1: his clitic production rate amounts to 87\%, compared with the average of German-speaking subjects (22\%). Since this datum cannot be put down to a longer exposure to Italian, the authors suggest that it may be due to the learner’s advanced knowledge of L2 Spanish, acquired in fifteen years of school education.

The second finding is found in the research carried out by \citet{Corino2012} on German-speaking learners’ L2 Italian. For some productions, the author acknowledges a possible influence of L2 French, as in the use of \textit{lui} instead of \textit{gli} (\citeyear[48]{Corino2012}):

\ea[*]{ \label{ex:sciutti:34}
    \gll  Quest’ altro lui faceva paura.\\
            this.\textsc{m.sg} other.\textsc{m.sg}  him.\textsc{m.sg.dat} made.\textsc{3sg.impfv} fear \\
    \glt  ‘This other scared him.’}
\ex [*]{ \label{ex:sciutti:35}
    \gll  La ragazza lui ha detto: che hai         fatto!\\
    the  girl     him.\textsc{m.sg.dat}  has   said   what have.\textsc{2sg}   done\\
    \glt  ‘The girl told him: what have you done!’}
\z

The role of an L2 is, therefore, a research field still in need of further empirical data in order to confirm or downplay the importance that a Romance L2 plays in acquiring Italian clitics.

\section {The {current} study} %{4.}
\label{sec:sciutti:4}

\subsection{Methodology}%{4.1.}
\label{sec:sciutti:4.1}

\subsubsection{Research {design} {and} goals} %{4.1.2}

The study aims primarily at investigating the role of two Romance L2s – French and Spanish – in the acquisition of clitic pronouns in infinitival clauses by Ger\-man-speaking learners of Italian. Since clitic pronouns do not exist in German, it may well be that an L2 equipped with clitics, as is the case of French and Spanish, exerts an influence on the acquisition of a similar syntactic category in L3 Italian, be it for the better or the worse. As illustrated in \sectref{sec:sciutti:2} above, despite both being equipped with clitics, French and Spanish also display substantial crosslinguistic differences as regards both repertoire and syntax. In terms of a comparison with Italian, there are syntactic similarities between Spanish and Italian, while between French and Italian the similarities have to do with the repertoire of the clitics instantiated.

A further goal of the study is probing the role of proficiency in both L3 Italian and L2 French, in an attempt to ascertain whether a varying degree of mastery of L3 Italian or L2 French affects the occurrence of crosslinguistic influence arising from the similarities and differences between L2 and L3.

The study looks into both the production and placement of clitics. As far as the former is concerned, two main questions arise, as follows:

\begin{enumerate}
\item As proficiency in either a Romance L2 (French)\footnote{The effect of varying proficiency in L2 Spanish could not be tested owing to a lack of informants with a suitable profile.} or in L3 Italian increases, is the high rate of avoidance strategies – mostly omissions or replacements with lexical DPs – found in previous studies (\citealt[191]{ChiniEtAl2003}; \citealt{LeoniniBelletti2004, Leonini2006, Giannini2008, Maffei2009}) reduced (with a parallel increase in overall clitic production)?

\item Does prior knowledge of L2 French – as well as a varying degree of proficiency in it – affect the learners’ production rates and grammaticality judgments of both partitive and locative clitics (instantiated in French but not in Spanish)?
\end{enumerate}

The question linked to clitic placement is whether prior knowledge of L2 Spanish reduce the occurrence of mistakes in infinitival clauses. This is because, as shown earlier on, clitics are differently placed in French and Italian, whereas in Spanish their placement mirrors the one obtaining in Italian.

\subsubsection{Informants} %4.1.2.

The informants were twenty German-speaking learners of Italian enrolled at the University of Hamburg, aged between 20 and 47 years old (on average 25.8). Fifteen of them were attending an intermediate course, whereas five of them were attending an advanced course. In order to collect their personal and language background data, the students were asked to fill out a questionnaire, which included, among other things, a self-evaluation of their level of proficiency in L2 French or Spanish. The informants with intermediate proficiency in L3 Italian have therefore been divided into the following three groups on based on their self-rated proficiency in L2 French or Spanish:

\begin{description}
\item [Group 1:] five informants with high proficiency in L2 French
\item [Group 2:] five informants with high proficiency in L2 Spanish
\item [Group 3:] five informants with low proficiency in L2 French
\end{description}

No group of informants with low proficiency in L2 Spanish was formed due to a lack of learners with such a profile. As far as the informants with advanced proficiency in L3 Italian are concerned, these groups were not divided into subgroups according to their prior knowledge of a Romance L2, as the actual focus of the research lay on learners having an intermediate proficiency in L3 Italian. This additional group mainly served the purpose of providing information about the development over time of clitic acquisition in non-native Italian, as proficiency in this language increases. Finally, the control group was made up of 21 Italian-speaking subjects, five of whom with knowledge of L2 German.

\subsubsection{Tasks} % \textbf{4.1.3.}

The informants attending an intermediate Italian course carried out the following four tasks:

\begin{itemize}
\item An oral elicitation task based on a series of pictures each accompanied by a question
\item A written translation task from German into Italian
\item An oral grammaticality judgment task (GJT1) made up of 70 Italian sentences
\item An oral grammaticality judgment task (GJT2) made up of 48 Italian sentences
\end{itemize}

The two GJTs were administered in two separate sessions and had a different focus on Italian clitics. The first one tested mainly clitic placement in infinitival clauses, whereas the second one also aimed at gathering information on the learners’ competence related to partitive and locative clitics. The informants attending an advanced Italian course carried out all the tasks with the sole exception of the GJT2.

The use of both written and oral tests was meant to vary and balance the nature of the data collected. As pointed out by \citet{BialystokRyan1985}, \citet{Montrul2009} and \citet{EllisR2005} amongst others, in L2 acquisition research written tasks performed in absence of time constraints are liable to lead to data linked to learners’ explicit or declarative knowledge, whereas oral tasks subject to time constraints are bound to give rise to data shedding light on learners’ implicit or procedural knowledge. Also, it was deemed necessary to include oral tasks besides the translation from L1 German into L3 Italian because the latter is liable to give rise to a slightly higher degree of crosslinguistic influence from the subjects’ L1 as compared to the elicitation and the GJTs, as noted by \citet{BennatiDiDomenico2008} and \citet{Tytus2019}, amongst others.\footnote{One suggestion to keep this potentially disturbing factor at bay in future research could be that of having the subjects translate from their L2 into the L3.}

\subsubsubsection{Elicitation task} % {4.1.3.1.}

The elicitation task consisted of a PowerPoint presentation containing twenty-four pictures (twelve experimental items and twelve fillers\footnote{For example, the image shown in the experiment for (\ref{ex:sciutti:Q}--\ref{ex:sciutti:R}) is available at \url{https://www.flickr.com/photos/sunumer/3337961554/}.}), each accompanied by a question, which the subjects could read and hear simultaneously. Every picture was visible on the screen, along with its question, until the subject gave his/her answer. Immediately after that, the next picture appeared on the screen. Before the actual test began, each subject was asked to read the instructions for the task in his/her L1 and went through a training session made up of six items. For each item, the subjects were instructed to repeat a part of the question – coloured in red – and not to repeat another part of the question, which was consistently crossed out. The part not to be repeated in the answer was the subject of the sentence in the mock test and in the test fillers (which did not require the production of any complement clitics), while it corresponded to a complement of the infinitival clause for ten experimental items of the actual test. Two experimental items aimed at eliciting reflexive clitics and so the subjects did not have to repeat once more the subject of the sentence – there was no complement which could be replaced by a clitic in this case. Thus the test was designed in such a way as to prompt the production of twelve clitics on the whole. All test items were randomised, so that their order was different for each subject. The following is an example of an experimental item of the test:

\ea \label{ex:sciutti:Q}
\gll Questa è una {coppia di fidanzati}. Che cosa vuole regalare il ragazzo alla ragazza?\\
this is a couple what thing wants {give as present}  the boy to.the girl\\
\glt `This is a couple. What does the boy want to give the girl?'
\ex Expected answers:\label{ex:sciutti:R}
\ea
\gll Le vuole regalare un fiore.\\
her want give a flower \\
\ex
\gll Vuole regalar=le un fiore.\\
want give=her a flower \\
\glt He wants to give her a flower.
\z
\z

\subsubsubsection{Translation task}% {4.1.3.2.}

The translation task consisted of ten short paragraphs in German to be translated into Italian. In order to maximise the production of clitics in the Italian translation, the German text was equipped with footnotes containing suggestions for the lexical items deemed less accessible to intermediate learners and essential for the translation of the experimental infinitival clauses.\footnote{The lexical items whose translation into Italian were provided in the footnotes were also chosen by making reference to the word list corresponding to the B1 level of proficiency contained in \citet{SpinelliParizzi2010}.} For the same reason, the subjects were allowed to resort to a bilingual dictionary, as the focus of the task was essentially grammatical. Moreover, the task was introduced by a chart showing an example of infinitival clause governed by a causative verb and its translation into Italian, so as to minimise the risk that such clauses, which the subjects may not have been familiar with, were skipped altogether in the translation. The sequence of the paragraphs within the German text was randomised.

\subsubsubsection{Grammaticality judgment tasks (GJTs)} % {4.1.3.3.}

The two GJTs consisted of Powerpoint Presentations made up of Italian sentences which the subjects could read on a computer screen and hear simultaneously. Every sentence was visible for seventeen seconds, after which the next sentence appeared on the screen. Before the actual test began, each subject was asked to read the instructions for the task in his/her L1 and went through a training session made up of five items. For each item, the subjects were instructed to say whether the sentence was grammatically correct or not. Additionally, if they found that the sentence was not grammatically correct, they had to identify the mistake and correct it orally. The subjects’ responses were recorded and then transcribed in writing for the sake of data analysis. All test items were randomised.

\subsubsection{Data analysis} % {4.1.4.}

In order to provide answers to the aforementioned research questions, the data resulting from the subjects’ performances in the experimental tasks were compared as follows:

\begin{itemize}
\item The group with advanced proficiency in Italian was compared with the group having intermediate proficiency in the same language, irrespective of any L2s known by the subjects. A \textit{t}-test was carried out to ascertain the existence of any statistically significant differences between the results of the two groups.

\item Among the intermediate-level learners of Italian, the subgroup with high proficiency in L2 French was compared with the group with high proficiency in L2 Spanish on the one hand and with the group with low proficiency in L2 French on the other. A one-way ANOVA test, followed by a Bonferroni comparison, was carried out to ascertain the existence of any statistically significant differences between the results of the two groups in each case. Given the limited number of subjects in each subgroup and of tokens in each test, though, statistically significant differences – if any – were typically found in omission rates rather than in production ones in both the elicitation and the translation tasks. For the same reason, no statistically significant difference usually emerged from the comparisons involving the production of lexical DPs or the use of other avoidance strategies.\footnote{Considering the small datasets – only five participants per group – it must be borne in mind that in some cases inferential statistics may be of limited relevance only.}

\item Still within the group with intermediate proficiency in Italian, the two subgroups with high proficiency in L2 French and in L2 Spanish, taken as a whole, were compared with that having a low proficiency in L2 French. A \textit{t}{}-test was carried out to ascertain the existence of any statistically significant differences.
\end{itemize}

When analysing the data of the elicitation and translation tasks, one consistent finding emerged from the corpus: in some cases the subjects produced non-target or weak pronouns (i.e. ones which did not exhibit all the features of either clitic or strong pronouns in Italian, but typically a mixture of the two). For this reason, when discussing the subjects’ pronominal production, the kind of pronouns produced will also be described, whether they were clitic, strong or weak pronouns.


\subsection{Results} % {4.2.}
\label{sec:sciutti:4.2}

\subsubsection{Pronominal production} %{4.2.1.}

\subsubsubsection{Overall {pronominal} production} % {4.2.1.1.}

As far as overall pronominal production is concerned, \tabref{tab:sciutti:1} recapitulates the results of the elicitation and translation tasks.%\todo{unclear: other avoid.? Please check column labels}

\begin{table}
\caption{\label{tab:sciutti:1} Overall pronominal production in the elicitation and translation tasks}
\fittable{
	\footnotesize
\begin{tabularx}{\textwidth}{p{1.3cm} YYYY p{0.2cm} YYYY}
\lsptoprule
 & \multicolumn{4}{c}{ Elicitation Task} & & \multicolumn{4}{c}{ Translation Task}\\
\cmidrule{2-5} \cmidrule{7-10}
Group & Produc\-tion & Omission & Lexical DP & Other Avoid. & & Produc\-tion & Omission & Lexical DP & Other Avoid.\\
\midrule
High L2 \newline French & 25/50 (50\%) & 9/50 (18\%) & 13/50 (26\%) & 3/50 (6\%) & & 92/110 (84\%) & 17/110 (15\%) & 1/110 (1\%) & 0/110 (0\%)\\
\tablevspace
High L2 \newline Spanish & 19/50 (38\%) & 20/50 (40\%) & 8/50 (16\%) & 3/50 (6\%) & & 75/110 (68\%) & 34/110 (31\%) & 1/110 (1\%) & 0/110 (0\%)\\
\tablevspace
Low L2 \newline French & 7/50 (14\%) & 39/50 (78\%) & 0/50 (0\%) & 4/50 (8\%) & & 62/110 (56\%) & 43/110 (39\%) & 1/110 (1\%) & 4/110 (4\%)\\
\tablevspace
Intermediate \newline L3 Italian  & 51/150 (34\%) & 68/150 (45\%) & 21/150 (14\%) & 10/150 (7\%) & & 229/330 (69\%) & 94/330 (29\%) & 3/330 (1\%) & 4/330 (1\%)\\
\tablevspace
Advanced \newline L3 Italian & 40/50 (80\%) & 3/50 (6\%) & 5/50 (10\%) & 2/50 (4\%) & & 92/110 (84\%) & 17/110 (15\%) & 0/110 (0\%) & 1/110 (1\%)\\
\tablevspace
L1 Italian \newline Control & 95/100 (95\%) & 0/100 (0\%) & 3/100 (3\%) & 2/100 (2\%) & & 109/110 (99\%) & 1/110 (1\%) & 0/110 (0\%) & 0/110 (0\%)\\
\lspbottomrule
\end{tabularx}
}
\end{table}
As can be seen, in both tasks the group with an advanced proficiency in Italian shows a higher production rate and a lower omission rate than the group with an intermediate proficiency in Italian. Within the latter, the subgroup with high proficiency in L2 French produces more pronouns and omits fewer than that with high proficiency in L2 Spanish and that with low proficiency in L2 French. As far as production rates are concerned, in both tasks there are statistically significant differences between the two groups with high proficiency in either French or Spanish as an L2, considered as a whole, and those of the group with low proficiency in L2 French ($t = 2.5078$; $p = 0.0267$ in the elicitation task; $t = 3.3365$; $p = 0.0059$ in the translation task). On the contrary, there are no statistically significant differences between the production rates of the three subgroups of subjects with an intermediate proficiency in Italian in either task. There is also a statistically significant difference between the groups with an intermediate proficiency in Italian, considered as a whole, and that with an advanced proficiency in Italian, but only in the elicitation task ($t = -3.2697$; $p = 0.01$). As for omission rates, in both tasks there are statistically significant differences between the omission rates of the two groups with high proficiency in either French or Spanish as an L2, considered as a whole, and those of the group with low proficiency in L2 French ($t = -6.0777$; $p = 0.0000$ in the elicitation task; $t = -2.8441$; $p = 0.0176$ in the translation task). There are also significant differences between the groups with high proficiency in either French or Spanish as an L2 ($F = 70.73$; $p = 0.012$ in the elicitation task; $F = 36.76$; $p = 0.024$ in the translation task) as well as between the group with low proficiency in L2 French and that with high proficiency in L2 French ($F = 70.73$; $p = 0.000$ in the elicitation task; $F = 36.76$; $p = 0.000$ in the translation task). Moreover, there is a statistically significant difference between the group with an intermediate proficiency in Italian and that with an advanced proficiency in Italian, but only in the elicitation task ($t = 2.8912$; $p = 0.0097$).

\subsubsubsection{Kind of pronominal production} % {4.2.1.2.}

As to the kind of pronominal production, one finding worth reporting is that within the group of subjects with intermediate proficiency in Italian only those proficient in French as an L2 sometimes produce strong pronouns placing them in a position which cannot be occupied by strong pronouns in Italian, as in \REF{ex:sciutti:36} here below:

\ea[*]{ \label{ex:sciutti:36}
    \gll   Ho intenzione di lei dare un libro.\\
            have.\textsc{1sg} intention  of  her.\textsc{f.sg.dat}  give  a  book \\
    \glt  ‘I am going to give her a book.’}
\z

This is done in 2 cases out of 25 (i.e. 8\%) by one subject with high proficiency in French in the elicitation task and in 5 cases out of 62 (i.e. 8\%) by subjects with low proficiency in French in the translation task.

\subsubsubsection{Production {of} {partitive} pronouns} % {4.2.1.3.}

\tabref{tab:sciutti:2} highlights the production and omission rates of partitive pronouns in the elicitation and translation tasks. These data thus represent a sub-set of the dataset presented in \tabref{tab:sciutti:1}:


\begin{table}
\caption{\label{tab:sciutti:2} Production of partitive pronouns in the elicitation and translation tasks}
\fittable{
\footnotesize
\begin{tabularx}{\textwidth}{p{1.3cm} YYYY p{0.1cm} YYYY}
	\lsptoprule
	& \multicolumn{4}{c}{Elicitation Task} & & \multicolumn{4}{c}{Translation Task}\\
	\cmidrule{2-5} \cmidrule{7-10}
	Group & Produc\-tion & Omission & Lexical DP & Other Avoid. & & Produc\-tion & Omission & Lexical DP & Other Avoid.\\
	\midrule
High L2 \newline French & 4/10 (40\%) & 0/10 (0\%) & 6/10 (60\%) & 0/10 (0\%) & & 14/30 (47\%) & 16/30 (53\%) & 0/30 (0\%) & 0/30 (0\%)\\
\tablevspace
High L2 \newline Spanish & 3/10 (30\%) & 4/10 (40\%) & 3/10 (30\%) & 0/10 (0\%) & & 0/30 (0\%) & 30/30 (100\%) & 0/30 (0\%) & 0/30 (0\%)\\
\tablevspace
Low L2 \newline French & 0/10 (0\%) & 10/10 (100\%) & 0/10 (0\%) & 0/10 (0\%) & & 2/30 (6\%) & 28/30 (94\%) & 0/30 (0\%) & 0/30 (0\%)\\
\tablevspace
Intermediate \newline L3 Italian & 7/30 (23\%) & 14/30 (47\%) & 9/30 (30\%) & 0/30 (0\%) & & 16/90 (18\%) & 74/90 (82\%) & 0/90 (0\%) & 0/90 (0\%)\\
\tablevspace
Advanced \newline L3 Italian & 8/10 (80\%) & 1/10 (10\%) & 1/10 (10\%) & 0/10 (0\%) & & 14/30 (47\%) & 16/30 (53\%) & 0/30 (0\%) & 0/30 (0\%)\\
\tablevspace
L1 Italian Control & 18/20 (90\%) & 0/20 (0\%) & 2/20 (10\%) & 0/20 (0\%) & & 30/30 (100\%) & 0/30 (0\%) & 0/30 (0\%) & 0/30 (0\%)\\
\lspbottomrule
\end{tabularx}
}
\end{table}

As the table shows, in both tasks the group with advanced proficiency in Italian displays a higher production rate and a lower omission rate than the group with intermediate proficiency in Italian. Within the latter, the subgroup with high proficiency in French produces more partitive clitics and omits fewer than the subgroup with high proficiency in Spanish and the subgroup with low proficiency in French. As far as production rates are concerned, in the elicitation task there is a statistically significant difference between the group with intermediate proficiency in Italian and that with advanced proficiency in Italian ($t = -3.1153$; $p = 0.0060$). On the contrary, in the translation task, there are no statistically significant differences. As to omission rates, in both tasks there is a statistically significant difference between the subgroup with high proficiency in L2 French and that with high proficiency in L2 Spanish within the group of subjects with intermediate proficiency in Italian ($F = 41.63$; $p = 0.004$ in the elicitation task; $F = 28.60$; $p = 0.038$ in the translation task). In the elicitation task, there is also a statistically significant difference between the subgroup with low proficiency in L2 French and that with high proficiency in L2 French ($F = 41.63$; $p = 0.000$).


\subsubsubsection{Production of locative pronouns} % {4.2.1.4.}

\tabref{tab:sciutti:3} highlights the production and omission rates of locative pronouns in the elicitation and translation task. These data thus represent a sub-set of the dataset presented in \tabref{tab:sciutti:1}.

\begin{table}
\caption{\label{tab:sciutti:3} Production of locative pronouns in the elicitation and the translation task}
\fittable{
	\footnotesize
\begin{tabularx}{\textwidth}{p{1.3cm} YYYY p{0.1cm} YYYY}
	\lsptoprule
	& \multicolumn{4}{c}{ Elicitation Task} & & \multicolumn{4}{c}{ Translation Task}\\
	\cmidrule{2-5} \cmidrule{7-10}
	Group & Produc\-tion & Omission & Lexical DP & Other Avoid. & & Produc\-tion & Omission & Lexical DP & Other Avoid.\\
	\midrule
High L2 \newline French & 5/10 (50\%) & 3/10 (30\%) & 2/10 (20\%) & 0/10 (0\%) & & 17/20 (85\%) & 1/20 (5\%) & 0/20 (0\%) & 2/20 (10\%)\\
\tablevspace
High L2 \newline Spanish & 0/10 (0\%) & 9/10 (90\%) & 1/10 (10\%) & 0/10 (0\%) & & 12/20 (60\%) & 2/20 (10\%) & 1/20 (5\%) & 5/20 (25\%)\\
\tablevspace
Low L2 \newline French & 0/10 (0\%) & 10/10 (100\%) & 0/10 (0\%) & 0/10 (0\%) & & 1/20 (5\%) & 9/20 (45\%) & 1/20 (5\%) & 9/20 (45\%)\\
\tablevspace
Intermediate \newline L3 Italian & 5/30 (17\%) & 22/30 (73\%) & 3/30 (10\%) & 0/30 (0\%) & & 30/60 (50\%) & 12/60 (20\%) & 2/60 (3\%) & 16/60 (27\%)\\
\tablevspace
Advanced \newline L3 Italian & 7/10 (70\%) & 3/10 (30\%) & 0/10 (0\%) & 0/10 (0\%) & & 17/20 (85\%) & 2/20 (10\%) & 0/20 (0\%) & 1/20 (5\%)\\
\tablevspace
L1 Italian Control & 12/20 (60\%) & 8/20 (40\%) & 0/20 (0\%) & 0/20 (0\%) & & 16/20 (80\%) & 3/20 (15\%) & 0/20 (0\%) & 1/20 (5\%)\\
\lspbottomrule
\end{tabularx}
}
\end{table}
\footnotetext{The term \textit{pro-forms} refers to the use of a locative adverb such as the Italian equivalent of \textit{there} where a locative clitic was expected to be produced.}

As can be noticed, in both tasks the group with an advanced proficiency in Italian shows a higher production rate and a lower omission rate than the group with an intermediate proficiency in Italian. Within the latter, the subgroup with high proficiency in L2 French produces more locative clitics and omits fewer than those with high proficiency in L2 Spanish and low proficiency in L2 French. As far as production rates are concerned, there are no statistically significant differences in either task. As to omission rates, in both tests there are statistically significant differences between the subgroup with low proficiency in L2 French and that with high proficiency in L2 French within the group of subjects with intermediate proficiency in Italian ($F = 11.90$; $p = 0.010$ in the elicitation task; $F = 7.51$; $p = 0.002$ in the translation task). Moreover, in the elicitation task there is a statistically significant difference between the subgroup with high proficiency in L2 French and that with high proficiency in L2 Spanish within the group of subjects with intermediate proficiency in Italian ($F = 11.90$; $p = 0.041$). In the translation task, there is also a statistically significant difference between the subgroup with high proficiency in L2 French and those with high proficiency in L2 Spanish and low proficiency in L2 French, if considered as a whole ($t = -2.2422$; $p = 0.0433$).

\subsubsection{Detection of omissions of partitive and locative pronouns} % {4.2.2.}

As far as partitive and locative pronouns are concerned, the findings of the elicitation and translation tasks seem to be corroborated by those stemming from the second GJT. \tabref{tab:sciutti:4} recapitulates the judgments given to the three ungrammatical items in which partitive clitics were omitted.

\begin{table}
\caption{\label{tab:sciutti:4}Judgments on items including an omission of a partitive clitic (ungrammatical)}
\fittable{\begin{tabular}{l r r r}
\lsptoprule
 & Grammatical & \multicolumn{2}{c}{ Ungrammatical}\\
 \cmidrule{3-4}
Group &   & Right correction & Wrong correction \\
\midrule
{High L2 French} & 9/15 (60\%) & 4/15 (27\%) & 2/15 (13\%)\\
%\tablevspace
{High L2 Spanish} & 12/15 (80\%) & 1/15 (7\%) & 2/15 (13\%)\\
%\tablevspace
{Low L2 French} & 12/14 (86\%) & 0/14 (0\%) & 2/14 (14\%)\\
%\tablevspace
{Intermediate L3 Italian} & 33/44 (75\%) & 5/44 (11\%) & 6/44 (14\%)\\
%$\tablevspace
{L1 Italian Control} & 1/27 (4\%) & 26/27 (96\%) & 0/27 (0\%)\\
\lspbottomrule
\end{tabular}}
\end{table}

These data mirror those obtained with the elicitation and translation tasks. Among the subjects with an intermediate proficiency in Italian the acceptance rate of the subgroup with high proficiency in L2 French is lower than those of the subgroup with high proficiency in L2 Spanish and of the subgroup with low proficiency in L2 French. At the same time, the subgroup with high proficiency in L2 French  judges the omissions of partitive clitics as ungrammatical – providing the right correction – to a greater extent than those with high proficiency in L2 Spanish and a low proficiency in L2 French, who never provide the right correction. In any case, there are no statistically significant differences in the GJT between the single experimental groups.

\tabref{tab:sciutti:5} recapitulates the judgments given to the three items in which locative clitics were omitted.

\begin{table}
\caption{\label{tab:sciutti:5} Judgments on items including an omission of a locative clitic (ungrammatical)}
\fittable{\begin{tabular}{l r r r }
	\lsptoprule
	& Grammatical & \multicolumn{2}{c}{ Ungrammatical}\\
	\cmidrule{3-4}
	Group &   & Right correction & Wrong correction \\
	\midrule
High L2 French & 8/14 (57\%) & 4/14 (29\%) & 2/14 (14\%)\\
High L2 Spanish & 11/15 (73\%) & 2/15 (13.5\%) & 2/15 (13.5\%)\\
Low L2 French & 13/15 (87\%) & 0/15 (0\%) & 2/15 (13\%)\\
Intermediate L3 Italian (Total) & 32/44 (73\%) & 6/44 (13.5\%) & 6/44 (13.5\%)\\
L1 Italian Control & 3/27 (11\%) & 24/27 (89\%) & 0/27 (0\%)\\
\lspbottomrule
\end{tabular}}
\end{table}

These data again mirror – partially, at least – those obtained with the elicitation and translation tasks. Within the group of subjects with an intermediate proficiency in Italian the acceptance rate of the subgroup with high proficiency in L2 French is lower than that of the subgroup with high proficiency in L2 Spanish and of the subgroup with low proficiency in L2 French. At the same time, the subgroup with high proficiency in L2 French  judges the omissions of locative clitics as ungrammatical – providing the right correction – to a greater extent than the subgroup with high proficiency in L2 Spanish and the subgroup with low proficiency in L2 French, whose subjects never provide the right correction. In any case, there are no statistically significant differences between the single experimental groups in this grammaticality judgment task.

\subsubsection{Clitic placement} % {4.2.3.}

As far as pronominal placement is concerned, a comparison between the elicitation and the translation tasks highlights that within the group of subjects with intermediate proficiency in Italian the only instances of pronouns placed before an infinitive – an ungrammatical option in Italian – are found in the subgroups with high and low proficiency in L2 French. \tabref{tab:sciutti:6} shows these findings.

\begin{table}
\caption{\label{tab:sciutti:6}Pronouns placed before an infinitive in the elicitation and translation tasks}
\begin{tabular}{lrr}
\lsptoprule
Group & Elicitation task & Translation task\\
\midrule
High L2 Fr & 1/30 (3\%) & 7/105 (7\%)\\
High L2 Sp & 0/18 (0\%) & 0/83 (0\%)\\
Low L2 Fr & 1/4 (25\%) & 16/54 (30\%)\\
Intermediate L3 Italian (Total) & 2/52 (4\%) & 23/242 (9.5\%)\\
Advanced L3 Italian & 0/47 (0\%) & 0/107 (0\%)\\
L1 Italian Control & 0/107 (0\%) & 0/121 (0\%)\\
\lspbottomrule
\end{tabular}
\end{table}

Indeed, the subgroup with high proficiency in L2 French places pronouns before an infinitive in 3\% of cases (i.e. 1 out of 30) in the elicitation task and 7\% of cases (i.e. 7 out of 105) in the translation task. The subgroup with low proficiency in L2 French, for its part, places pronouns before an infinitive in 25\% of cases (i.e. 1 out of 4) in the elicitation task and 30\% of cases (i.e. 16 out of 54) in the translation task. On the contrary, no instances of pronouns placed before an infinitive are found in the productions of the subgroup with high proficiency in L2 Spanish nor in the group with advanced proficiency in Italian. As far as statistics are concerned, the low production rates in the elicitation task do not make it possible to establish whether the differences between the experimental groups are significant or not. In the translation task, within the group of subjects with intermediate proficiency in Italian there is a significant difference between the subgroups with low and high proficiency in L2 French with respect to the placement of the clitic directly before the infinitive, but only in clauses not licensing restructuring ($F = 6.87$; $p = 0.020$). Moreover, there is also a significant difference in all clause types between the subgroup with low proficiency in L2 French and that with high proficiency in L2 Spanish ($F = 6.87$; $p = 0.003$ in clauses not licensing restructuring; $F = 6.94$; $p = 0.011$ in optionally restructuring clauses and in obligatorily restructuring clauses governed by a causative verb).

These results seem to be substantiated by those obtained with the first GJT. \tabref{tab:sciutti:7} recapitulates the judgments given to the fourteen items in which a clitic was placed before an infinitive.\footnote{Since some subjects did not express any judgment over some of the items, the total number of items per group is variable.}

\begin{table}
\caption{\label{tab:sciutti:7} {Judgments} {on} {items} {including} {a} {clitic} {placed} {before} {an} {infinitive}}
\fittable{\begin{tabular}{l r r r }
	\lsptoprule
	& Grammatical & \multicolumn{2}{c}{ Ungrammatical}\\
	\cmidrule{3-4}
	Group &   & Right correction & Wrong correction \\
	\midrule
High L2 French & 27/60 (45\%) & 24/60 (40\%) & 9/60 (15\%)\\
High L2 Spanish & 17/65 (26\%) & 33/65 (51\%) & 15/65 (23\%)\\
Low L2 French & 43/55 (78\%) & 4/55 (7.5\%) & 8/55 (14.5\%)\\
Intermediate L3 Italian (Total) & 87/180 (48\%) & 61/180 (34\%) & 32/180 (18\%)\\
Advanced L3 Italian & 10/67 (15\%) & 41/67 (61\%) & 16/67 (24\%)\\
L1 Italian Control & 0/168 (0\%) & 165/168 (98.5\%) & 3/168 (1.5\%)\\
\lspbottomrule
\end{tabular}}
\end{table}

These data mirror those obtained with the elicitation and translation tasks. As can be observed, within the group of subjects with intermediate proficiency in Italian, the subgroup with low proficiency in L2 French accepts these ungrammatical items to a greater extent than the other two subgroups. For its part, the subgroup with high proficiency in L2 French judges these items as grammatical to a greater extent than does the group with high proficiency in L2 Spanish. The rate of acceptance of the group with advanced proficiency in L3 Italian is only a 15\% (i.e. 10 cases out of 67), against 48\% (i.e. 87 out of 180) for the overall group with an intermediate proficiency in Italian. An ANOVA test followed by a Bonferroni comparison reveals that there is a significant difference in the acceptance rates between the subgroup with low proficiency in L2 French and that with high proficiency in L2 Spanish ($F = 17.44$; $p = 0.008$), in line with the finding of the translation task. Turning to the ungrammatical judgments provided with the right correction, a similar picture emerges. Indeed, within the group of subjects with intermediate proficiency in Italian, the subgroup with low proficiency in L2 French both judges these items as ungrammatical and provides the right correction to a lesser extent than the other two subgroups. For its part, the subgroup with high proficiency in L2 French does so to a lesser extent than that with high proficiency in L2 Spanish. Finally, the rate of exact answers provided by the group with advanced proficiency in Italian is higher than that of the group with intermediate proficiency in Italian.

\section{Discussion} %{5.}
\label{sec:sciutti:5}

\subsection{Pronominal production} %{5.1}
\label{sec:sciutti:5.1}

\subsubsection{Overall {pronominal} production} %{5.1.1}

As a reminder, the research question pertaining to overall pronominal production was the following: As proficiency in either French L2 or L3 Italian increases, does the high rate of avoidance strategies – mostly omissions or replacements with lexical DPs – reduce (with a parallel increase in overall clitic production)?

The findings pertaining to overall pronominal production might point to the fact that high proficiency in a Romance L2 increases pronominal production and reduces instances of omissions, substitutions or avoidances in L3 Italian to a greater extent than does low proficiency in a Romance L2. German learners seem all the more able to transfer their knowledge about the existence of a clitic pronominal series from one Romance language to another as their proficiency in a background Romance language increases. However, a wider corpus would be needed to support this claim. The data available only allow a comparison between learners with a high and a low proficiency in L2 French, but nothing can be said about proficiency in L2 Spanish, for example. Additionally, pronominal production is boosted, with a parallel decrease in omissions, substitutions and avoidances, as proficiency in Italian increases: advanced learners of Italian have been shown to produce more pronouns – and fewer instances of omissions, substitutions and avoidances – than intermediate ones in both the elicitation and the translation tasks.\footnote{Only in the translation task are the avoidance rates the same, equalling 1\%.}

\subsubsection{Kind of pronominal production} % {5.1.2}

As to the kind of pronominal production, within the group of subjects with an intermediate proficiency in Italian, only those proficient in French as an L2 sometimes produced strong pronouns, placing them in a position which cannot be occupied by strong pronouns in Italian. There were no such instances in the subgroup of subjects with high proficiency in L2 Spanish, nor in the group with an advanced proficiency in Italian. \citet{HamannBelletti2006} discuss a similar error found by \citet[355]{GranfeldtSchlyter2004} and made by an L2 speaker, here reported for comparison:

\ea[*]{\label{ex:sciutti:37}
    \gll    Il a lui assis.\\
            he has him.\textsc{m.sg.acc} sat\\
    \glt  ‘He sat him down.’}
\z

This kind of error is said to be typical of the productions of German-speaking learners of French \citep{Herschensohn2004}. Consider these examples of productions of German-speaking learners of Italian taken from the VALICO corpus \citep[48, 55]{Corino2012}:


\ea[*]{\label{ex:sciutti:38}
    \gll Quest’ altro lui faceva paura. \\
         this  other   him-\textsc{m.sg.dat}  made.\textsc{3sg.impfv}   fear   \\
    \glt  ‘This other scared him.’}
\ex [*]{ \label{ex:sciutti:39}
    \gll La ragazza lui ha detto: che hai fatto!\\
    the girl him.\textsc{m.s.dat} has  said: what have.\textsc{2.SG} done \\
    \glt  ‘The girl told him: what have you done!’}
\z


\citet{HamannBelletti2006} claim that this error derives from a misanalysis of complement clitics as weak pronouns, instantiated in German, much as the error consisting of placing a clitic in a thematic position after a finite verb, reported by \citet[355]{GranfeldtSchlyter2004} as produced by a Swedish-speaking learner of French:

\ea[*]{\label{ex:sciutti:40}
    \gll Elle croit la.\\
    She believes her.\textsc{f.sg.acc}\\
    \glt  ‘She believes her.’}
\z

According to \citet{HamannBelletti2006}, such a misanalysis might be reinforced by the fact that weak pronouns are also instantiated in French, although not with the function of complements. Indeed, subject pronouns in French are in fact weak pronouns. In the cases found in this experiment, the same principle might be at work. In other words, it may well be that German-speaking learners of Italian first tend to assimilate clitics to weak pronouns, instantiated in German, thus producing non-target pronouns of the same kind as the one in \REF{ex:sciutti:36}. The fact that such errors are only found among the subjects proficient in French as an L2 might indicate that such a misinterpretation is again reinforced by the existence of weak pronouns in French in the shape of subject pronouns. In any case, it seems that a high proficiency in L2 Spanish as well as advanced proficiency in Italian reduce the likelihood of occurrence of such non-target pronouns.

\subsubsection{Partitive and locative clitics} % 5.1.3}

The research question related to partitive and locative clitics was the following: Does prior knowledge of L2 French – as well as a varying degree of proficiency in it – affect the learners’ production rates and grammaticality judgments of both partitive and locative clitics (instantiated in French but not in Spanish)?

On the whole, the findings related to partitive and locative clitics may be an indication of the fact that a good prior knowledge of French increases production of both partitive and locative clitics and reduces instances of their omissions in L3 Italian. German learners seem all the more able to transfer their knowledge about the existence of partitive and locative clitics from French to Italian as their proficiency in French increases. Indeed, the subjects with high proficiency in L2 French have produced the most – and omitted the least – partitive and locative clitics within the group with an intermediate proficiency in Italian. The instantiation of partitive and locative clitics in French has apparently had the effect of boosting the production of partitive and locative clitics in Italian, while limiting instances of omissions. The fact that the subjects with high proficiency in L2 Spanish have produced more – and omitted fewer – partitive and locative clitics than the subjects with low proficiency in L2 French may indicate that the former find themselves in a more advanced stage in the acquisition path of partitive and locative clitics. Production of both partitive and locative clitics increases, with a parallel decrease in omissions, also as proficiency in Italian increases. As a matter of fact, advanced learners of Italian have produced more partitive and locative clitics – with fewer instances of omissions – than intermediate ones in both the elicitation and the translation task.

\subsection{Pronominal placement} % {5.2}
\label{sec:sciutti:5.2}

The research question regarding clitic placement was the following: does prior knowledge of L2 Spanish reduce the occurrence of mistakes in infinitival clauses?

As far as pronominal placement is concerned, it seems that the tendency to place and accept clitics before an infinitive is favoured by prior knowledge of L2 French. Cases of placement errors similar to those found in the experiment (i.e. with clitics placed in an intermediate position between two verbs) have been reported by several authors as typical of L2 learners, especially those with a Germanic L1 (\citealt{GundelTarone1983, ConnorsNuckle1986, Zobl1992, TowellHawkins1994, GrondinWhite1996,  HulkMuller2000, Herschensohn2004, Ferrari2006, Maffei2009, Corino2012}). Such errors are part of the following sequence of acquisition of pronominal placement identified by \citet{TowellHawkins1994} and \citet{Herschensohn2004} for English-speaking learners of French and by \citet{Schlyter1997} for Swedish-speaking learners of French:

\begin{enumerate}
\item Postverbal position:
	 \ea[*]{\label{ex:sciutti:41}
	 \gll Je vois lui\\
	 	I see him.\textsc{sg.acc}\\} 
 		\z

\item Omission of the object:
			\ea[*]{\label{ex:sciutti:42}
			\gll Je ai vu $\emptyset$\\
				I have seen $\emptyset$\\}
			\z

\item Intermediate position:
		\ea[*]{\label{ex:sciutti:43}
		\gll Je ai le vu\\
		I have him.\textsc{sg.acc} seen \\}
		\z

\item Pre-finite position, target-like:%\todo{why is this starred?}
				\ea[*]{ \label{ex:sciutti:44}
				\gll Je l' ai vu \\
				I him.\textsc{sg.acc} have seen \\
				\glt {`I have seen him.'} }
				\z
\end{enumerate}

Even though in the previous examples the intermediate position of the clitic involves an auxiliary verb and a past participle of a lexical verb, clitics found between a finite verb and the infinitive in the corpus of this study might also belong to the third stage of the acquisition process. The same sequence could also hold for German-speaking learners of Italian, as suggested by \citet{HamannBelletti2006}. The presence of clitics in an intermediate position, between the finite verb and the infinitive, in those subjects proficient in L2 French could be accounted for by a similarity between German and French which apparently reinforces the general tendency of German-speaking learners to place clitics in an intermediate position, as typical of a stage of their acquisition process.\footnote{There is a striking similarity with what was suggested above when discussing the instances of strong pronouns placed in a position which cannot be occupied by strong pronouns.} Indeed, at a superficial level, clitic placement in French complex clauses containing an infinitive which do not restructure and in those which optionally do resembles the word order found in German, since the pronouns occur in a higher position in syntax than the infinitive in both languages, as the following example shows:

\ea \label{ex:sciutti:45}
\begin{xlist}
    \ex \label{ex:sciutti:45a}
    \gll Je vais me laver.\\
        I go myself.\textsc{acc} wash\\
    \glt  ‘I’ll go and wash myself.’

    \ex \label{ex:sciutti:45b}
    \gll Ich gehe mich waschen.\\
        I go myself.\textsc{acc} wash\\
    \glt ‘I’ll go and wash myself.’
\end{xlist}
\z

As in the case discussed above of strong pronouns filling a position which cannot be occupied by strong pronouns in Italian, this similarity between French and German word order may induce a misanalysis of Italian word order contributing to an extension of the third stage in the sequence mentioned above for the learners with prior knowledge of French. In other words, for these learners the process of acquisition of clitic placement may be temporarily slowed down. As in the case of partitive and locative clitics discussed above, the fact that the subjects with high proficiency in L2 French have placed fewer clitics before an infinitive than those with low proficiency in L2 French may be explained by a greater degree of metalinguistic awareness that usually goes hand in hand with high proficiency in an L2 – especially in one closely related to the target language. There are several studies in support of this claim, all of which point to heightened metalinguistic awareness and enhanced metacognitive skills in learners with high proficiency in one or more L2s. For instance, \citet{Fouser2001} carried out an introspective study on two English-speaking learners of Korean who had prior advanced knowledge of Japanese as a non-native language and found that their reflection on their learning process and their awareness of the relationship between Korean and Japanese facilitated their task. Another study worth mentioning is the one carried out by \citet{Jessner1999}, who analysed qualitative data stemming from think-aloud protocol sessions. She reports that Italian/German bilingual learners of English as a non-native language consciously reflect on and compare their prior knowledge of two languages in searching for a word in an L3. Thus the boost to metalinguistic awareness and metacognitive skills typically spotted in multilingual learners may be due to the interactive nature of knowledge within the multilingual mind (\citealt{HerdinaJessner2000, HerdinaJessner2002, Jessner2003, Jessner2008DST, Jessner2008Knowledge, Jessner2009}). As regards the subjects of this study, those with high proficiency in L2 French are apparently more aware than the low-proficient ones of the contrast between French and Italian in the complex infinitival clauses mentioned above, which makes the placement of clitics in an intermediate position less likely for the former than for the latter. Finally, the complete lack of instances of clitics placed before an infinitive in the learners with high proficiency in L2 Spanish and in those with high proficiency in Italian might be an indication of the fact that these learners are already past the stage in which clitics are placed in an intermediate position.

\section{Conclusion} % original word says {5.}, should probably be 6
\label{sec:sciutti:6}

On the whole, the experimental data have shed light on the roles of the subjects’ L1 and L2s in their process of acquisition of Italian clitics in complex infinitival clauses, as well as on the evolution over time of the acquisitional patterns of German-speaking learners of Italian as a non-native language, as their proficiency in the target language increases.

Clitics confirm themselves as a tricky syntactic feature to acquire, in that their properties are not fully mastered at intermediate level of proficiency in Italian. However, high proficiency in a Romance L2 – French or Spanish – is likely to influence the overall acquisition of Italian clitics, by enhancing their production and correspondingly reducing the occurrence of omissions. When it comes to specific categories of clitics like partitive and locative ones, high proficiency in a Romance language like French in which they are instantiated is likely to foster their production while reducing their omissions correspondingly. An in-depth investigation of some issues related to pronominal placement in the clauses which constitute the focus of this research has made it possible to further examine the role of French and Spanish as L2s, their interactions with the subjects’ L1 as well as the effect of varying proficiency in L2 French. Since the patterns of clitic placement in infinitival clauses are identical in Spanish and Italian, prior knowledge of the former apparently plays a facilitative role in the subsequent acquisition in Italian in this respect. On the contrary, prior knowledge of French, a language in which the patterns of clitic placement in infinitival clauses mostly differ from Italian, seems to have the effect of reinforcing an underlying tendency which some authors ascribe to the learners’ L1. In both cases, an L2 interacts with the L1 and, depending on the specific features instantiated in the L2, the outcome of such an interaction is either a speeding up or a slowing down of the acquisitional process.

A similar conclusion has been reached to account for another phenomenon spotted in the productions of the subjects proficient in L2 French and totally absent from those of the subjects proficient in L2 Spanish, namely instances of strong pronouns used in positions which are illicit for strong pronouns in Italian. This finding too has been compared with the existing data from previous studies pointing to similar pronominal productions in the utterances of German-speaking learners of French. It has been argued that clitics tend to be assimilated to Germanic weak pronouns at first, and that this temporary misinterpretation may well be reinforced by the instantiation of weak pronouns in French in the shape of phonological subject clitics.

Proficiency in L2 French plays a role in the acquisition of L3 Italian, too. Indeed, the comparison between the subjects with a high and a low proficiency in L2 French has revealed that a higher proficiency in this language generally leads to more target-like pronominal productions: fewer clitics are placed before an infinitive and fewer strong pronouns are used in positions which cannot be occupied by strong pronouns in Italian. High proficiency in L2 French may therefore have the effect of speeding up the process of clitic acquisition in L3 Italian.

Finally, the degree of proficiency in the target language is yet another factor which has repercussions on the acquisition of Italian clitics in complex clauses containing an infinitive: all in all, the subjects with an advanced proficiency in Italian outperform those with an intermediate proficiency in Italian, showing a more thorough mastery of the properties of cliticisation. This is apparent across all the experimental tasks and for all the phenomena investigated. However, a close examination of certain structures characterised by a very high degree of complexity, such as the clauses containing an infinitive governed by a causative verb, reveals that even advanced learners are still struggling with the multifaceted phenomenon of cliticisation of the Italian language.

{\sloppy\printbibliography[heading=subbibliography,notkeyword=this]}
\end{document}
