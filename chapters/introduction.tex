\documentclass[output=paper,colorlinks,citecolor=brown,nonflat]{langsci/langscibook}
\author{Camilla Bardel and Laura Sánchez \affiliation{}}
\title{Introduction}
\abstract{}
\IfFileExists{../localcommands.tex}{
  % add all extra packages you need to load to this file
\usepackage{tabularx}
\usepackage{url}
\urlstyle{same}

\usepackage{listings}
\lstset{basicstyle=\ttfamily,tabsize=2,breaklines=true}


%%%%%%%%%%%%%%%%%%%%%%%%%%%%%%%%%%%%%%%%%%%%%%%%%%%%
%%%                                              %%%
%%%           Examples                           %%%
%%%                                              %%%
%%%%%%%%%%%%%%%%%%%%%%%%%%%%%%%%%%%%%%%%%%%%%%%%%%%%
%% to add additional information to the right of examples, uncomment the following line
% \usepackage{jambox}
%% if you want the source line of examples to be in italics, uncomment the following line
% \renewcommand{\exfont}{\itshape}
\usepackage{langsci-optional}
%\usepackage{langsci-optional}
\usepackage{langsci-gb4e}
% \usepackage{langsci-lgr}
\makeatletter
\let\pgfmathModX=\pgfmathMod@
\usepackage{pgfplots,pgfplotstable}%
\let\pgfmathMod@=\pgfmathModX
\makeatother
\usepgfplotslibrary{colorbrewer}
\usetikzlibrary{fit}
%\usetikzlibrary{positioning}

\definecolor{lsDOIGray}{cmyk}{0,0,0,0.45}

\usepackage{xassoccnt}
\newcounter{realpage}
\DeclareAssociatedCounters{page}{realpage}
\AtBeginDocument, free-standing-units, input-open-uncertainty= , input-close-uncertainty= ,table-align-text-pre=false,uncertainty-separator={\,},group-digits=false,detect-inline-weight=math}
\DeclareSIUnit[number-unit-product={}]{\percent}{\%}
\makeatletter \def\new@fontshape{} \makeatother
\robustify\bfseries % For detect weight to work

  \newcommand{\appref}[1]{Appendix \ref{#1}}
\newcommand{\fnref}[1]{Footnote \ref{#1}}

\newenvironment{langscibars}{\begin{axis}[ybar,xtick=data, xticklabels from table={\mydata}{pos},
        width  = \textwidth,
	height = .3\textheight,
    	nodes near coords,
	xtick=data,
	x tick label style={},
	ymin=0,
	cycle list name=langscicolors
        ]}{\end{axis}}

\newcommand{\langscibar}[1]{\addplot+ table [x=i, y=#1] {\mydata};\addlegendentry{#1};}

\newcommand{\langscidata}[1]{\pgfplotstableread{#1}\mydata;}

\makeatletter
\let\thetitle\@title
\let\theauthor\@author
\makeatother

\newcommand{\togglepaper}[1][0]{
%   \bibliography{../localbibliography}
  \papernote{\scriptsize\normalfont
    \theauthor.
    \thetitle.
    To appear in:
    Change Volume Editor \& in localcommands.tex
    Change volume title in localcommands.tex
    Berlin: Language Science Press. [preliminary page numbering]
  }
  \pagenumbering{roman}
  \setcounter{chapter}{#1}
  \addtocounter{chapter}{-1}
  }

\usetikzlibrary{shapes.arrows,shadows}

% \tikzfading[name=arrowfading, top color=transparent!0, bottom color=transparent!95]
% \tikzset{arrowfill/.style={top color=white, bottom color=white}}
\tikzset{arrowstyle/.style={draw=black, thick, single arrow, minimum height=#1, single arrow,
single arrow head extend=.1cm}}

\newcommand{\tikzfancyarrow}[1][1cm]{\tikz[baseline=-0.5ex]\node [arrowstyle=#1]{} ;}

  %% hyphenation points for line breaks
%% Normally, automatic hyphenation in LaTeX is very good
%% If a word is mis-hyphenated, add it to this file
%%
%% add information to TeX file before \begin{document} with:
%% %% hyphenation points for line breaks
%% Normally, automatic hyphenation in LaTeX is very good
%% If a word is mis-hyphenated, add it to this file
%%
%% add information to TeX file before \begin{document} with:
%% %% hyphenation points for line breaks
%% Normally, automatic hyphenation in LaTeX is very good
%% If a word is mis-hyphenated, add it to this file
%%
%% add information to TeX file before \begin{document} with:
%% \include{localhyphenation}
\hyphenation{
affri-ca-te
affri-ca-tes 
}
\hyphenation{
affri-ca-te
affri-ca-tes 
}
\hyphenation{
affri-ca-te
affri-ca-tes 
}
  \togglepaper[1]%%chapternumber
  \addbibresource{../localbibliography.bib}
}{}

\begin{document}
\maketitle

This book is concerned with how languages are learned by someone who already speaks at least two languages. Hence, the authors of the different chapters look beyond the classical second language acquisition perspective, according to which the researcher, traditionally, is interested in how people with monolingual backgrounds learn a second language (L2) or how bilingual speakers use and process their two languages. Research into \textit{third language} (L3) acquisition or learning,\footnote{For the sake of convenience, the terms acquisition and learning will be used interchangeably in this introduction.}~a branch of multilingualism that studies how multilinguals learn an additional language, has grown strong during the last decades. In this research area, we reserve the term bilingualism for cases where two languages coexist in the mind of the individual, a first language (L1) and an L2, or two L1s. When two or more languages are present in the speaker’s mind, no perfect balance among these languages can be expected. Variation and dynamics concerning use, style and proficiency of the different languages an individual knows are characteristic for the multilingual language system.

In this volume, the L3 is viewed in the light of three factors: age, language proficiency and multilingualism itself. Age can be considered in different ways. Both the age of onset of learning the target language and that of previously acquired languages (as in simultaneous vs. sequential bilinguals) are of interest (see e.g., the empirical studies of \citeauthor{chapters/munoz}, \citeauthor{chapters/pfenninger} and \citeauthor{chapters/sanchez7} in this volume). Age and its interaction with multilingualism is focused on in the chapters by Carmen \citeauthor{chapters/munoz} and Simone \citeauthor{chapters/pfenninger}, both conducted in instructed settings, and it is also discussed in depth in the first chapter of the volume, a conceptual paper written by Laura \citeauthor{chapters/sanchez1}. Proficiency in the target language has been held as one of the key factors for the intricate crosslinguistic influence in L3 learning and use ever since the seminal paper from \citet{WilliamsHammarberg1998} was published. The proficiency level in the L2 has also been suggested to play a role for L3 development and for transfer from the background languages (i.e., L1 and L2, see e.g., \citealt{BardelLindqvist2007, SánchezBardel2017Transfer}). In this volume, an empirical study by \citeauthor{chapters/sanchez7} pursues the subject of proficiency in the L2, while Sandro \citeauthor{chapters/sciutti}, in his study, investigates proficiency in the L3 as well as in the L2 in the understudied area of clitic pronouns in L3 acquisition. In a study on the multilingual lexicon, Anna \citeauthor{chapters/gudmundson} investigates how an L2 in which learners have high proficiency can play a role in word associations in the L3, and finds that the L2 can mediate semantic access for L3 words.

One basic assumption in research on L3 learning is that multilingualism \textit{per se} (bilingualism included) enhances both further language learning and the potentially achieved proficiency in additional languages. It has been suggested that both L1 and L2 knowledge (\citealt{FlynnFoleyVinnitskaya2004, BerkesFlynn2016}), and the experience of second or foreign language learning (\citealt{Hufeisen2005, Jessner2006}), will benefit the learning of subsequent languages. Possible explanations of such positive effects of multilingualism would be the cognitive advantages in terms of language awareness and high degrees of metalinguistic knowledge and communicative skills that multilingual learners may have developed while learning and using multiple languages. Cummins’ (\citeyear{Cummins1976, Cummins1991}) \textit{Interdependence Hypothesis} concerning the role of literacy skills in L1 for L2 development and \textit{Threshold Hypothesis} for the positive effects of proficiency have been adopted by several L3 researchers (e.g., \citealt{Cenoz2003}), who suggest that proficiency in the L1 and in the L2 may affect the learning of an L3 positively. However, that multilingual language learning is complex and depends on a number of interacting factors becomes clear in \citeauthor{chapters/munoz}’ chapter, where age is shown to play an important role for young learners’ capacity to draw on cognates in the languages they know, and in \citeauthor{chapters/pfenninger}’s study that emphasizes the role of social and educational factors for successful multilingual development. To diverse degrees, all the papers in the volume deal with the complex relationship between age, proficiency and multilingualism in additional language learning.

Two linguistic areas that have a longstanding tradition in the L3 field are lexis and syntax. Among the six empirical papers in this volume we have included two chapters that specifically deal with lexical aspects (\citeauthor{chapters/gudmundson}; \citeauthor{chapters/munoz}) and three studies on syntax (\citeauthor{chapters/sanchez7}, \citeauthor{chapters/sciutti} and \citeauthor{chapters/stadt}). As for the biological age of the participants, two of the chapters are concerned with adult learners (\citeauthor{chapters/gudmundson}; \citeauthor{chapters/sciutti}) and four with young multilinguals learning an additional language in school contexts (\citeauthor{chapters/munoz}; \citeauthor{chapters/pfenninger}; \citeauthor{chapters/sanchez7} and \citeauthor{chapters/stadt}). The volume starts with two conceptual papers. The first chapter, \textsc{Multilingualism from a language acquisition perspective}, by \textsc{Laura \citeauthor{chapters/sanchez1},} is a state of the art of research into multilingualism with a special focus on the respective roles of age and proficiency in L3 acquisition. As such it offers a theoretical background to the content in the rest of the volume. Moreover, it presents a brief overview of research on crosslinguistic influence in L3 acquisition. The chapter draws an important distinction between two types of multilingual language learning. One is third or additional language learning by people who have previous experience of one or more non-native languages learned as adults, or at least after the critical period (CP). The other type is third or additional language learning by bilinguals from an early age. Especially the age factor, but also the proficiency factor, can be expected to come into play differently in these two types of multilingualism considering that in the first case, the L2 has been learned after the CP and in the second, two languages have been acquired before this phase in the individual’s cognitive and linguistic development. This can be assumed to be an important distinction to make when it comes to different conditions for processing, development and ultimate attainment of the languages that constitute the background knowledge and potential transfer sources in L3 learning
In the second chapter, also essentially theoretical, \textsc{The conceptualization of knowledge about aspect: From monolingual to multilingual representations, Rafael \citeauthor{chapters/salaberry}} looks into the grammatical category of aspect from the L3 perspective. The queries posed in this chapter concern the roles of the background languages and the differences in processing mechanisms used for implicit versus explicit knowledge (\citealt{EllisN2005}), or implicit competence versus explicit knowledge \citep{Paradis2009}, that may determine crosslinguistic influence. According to the author, the complex construct of aspect, with its semantic, syntactic and discursive facets, lends itself ideally for evaluating the potential effect of the L1 and the L2 on the developing L3, and for assessing two dimensions that have been identified in recent theoretical L3 models: typological proximity, on the one hand, and the processing mechanisms applied in implicit competence versus explicit knowledge, on the other. The range of linguistic representations of the perspective-driven notion of aspect and its prototypical and non-prototypical conceptualisations related to context make it a complex part of language to grasp in an L2, let alone in multilingual learning. As pointed out by \citeauthor{chapters/salaberry}, this complexity and the fact that the temporal-aspectual systems differ to various degrees between groups of languages, for instance when comparing Romance languages and Germanic languages, render aspect an interesting test case for the effect on the L3 of prior knowledge of and about languages that are similar to or different from the new language in this respect.


The few available empirical studies on L3 learning of aspect and a few recent L2 studies are reviewed in the chapter. These are interpreted as support for the claim that processing constraints associated with L2/L3 learning are distinct from those linked to the L1. According to the author, L3 data on aspect learning indicate that the L3 will mainly rely on the same processing mechanisms as those used in the L2. It is acknowledged that recent L2 studies point to an influence from the L1 in learning aspect in the new language and that there is an L1 effect across all subsequently acquired languages, at least when it comes to non-prototypical meanings of aspect, and as mentioned, there are few L3 studies on aspect. This may be related to its inherent semantic, syntactic and morphological complexity, a complexity which makes it hard to set up rigorous designs for comparison when multiple languages are involved. More empirical studies of this particular linguistic area are needed and consequently the paper ends with a call for more studies on aspect in multilingual learning.

The two conceptual papers summarized above are followed by two chapters presenting empirical studies on adult L3 learners, one on the multilingual lexicon (\citeauthor{chapters/gudmundson}) and one on L3 syntax (\citeauthor{chapters/sciutti}). Chapter 3, \textsc{The mental lexicon of multilingual adult learners of Italian L3: A study of word association behaviour and cross-lingual semantic priming}, by Anna \citeauthor{chapters/gudmundson}, is a partial replication of a study of bilingual speakers conducted by \citet{FitzpatrickIzura2011}, who found differences in types of bilinguals’ word associations in their L1 and L2. Widening the scope to three languages, \citeauthor{chapters/gudmundson} investigated the mental lexicon of multilingual speakers of Swedish L1, English L2 and Italian L3. All participants were unbalanced trilinguals in terms of proficiency, having started with Italian as adults and with relatively high proficiency in English and lower proficiency in Italian.

The aim of the study was to identify how word associations differ, in terms of association type and response time, in the native and the non-native languages. The effect of language status (L1, L2 or L3) and association category on reaction time and on the distribution of associations in different categories was measured in word association tasks in all three languages. Results showed a difference between the languages regarding the association distribution; for example, the proportion of equivalent meaning associations was larger in the L1 than in the L2, and larger in the L2 than in the L3. The proportion of non-equivalent meaning associations showed the opposite pattern, indicating a switch in the type of associations related to proficiency. Collocational associations were mainly made in the L1 and form-based associations were mainly made in the L3. There was also a difference regarding the speed of association, that is, participants associated faster in the L1 than in the L2 and in the L2 than in the L3, generally. As regards the speed related to the different association categories, though, the pattern was similar across all languages; reaction times were fast for collocational associations and equivalent meaning associations, and slower for non-equivalent meaning associations. Results suggest that the differences are due to differences in proficiency levels but that the basic mechanisms related to lexical representation and access are similar in all languages.

The study also investigated the effect of long-term semantic priming and lexical mediation between L2 and L3, that is, whether the activation of conceptual information of L3 words was mediated by corresponding word forms in the L2. The primes were English translation equivalents of stimulus words from the prior Italian word association task. The translation equivalents obtained shorter reaction times compared to control words, indicating that L2 English words were activated during the L3 Italian word association task. This result from trilingual speakers is interesting in relation to the one obtained by \citet{FitzpatrickIzura2011}, who found a semantic mediation effect in L2 from L1 word forms in bilingual speakers. \citeauthor{chapters/gudmundson}’s results from multilinguals contribute by recognizing that an L2 in which a learner has high proficiency can take on a similar role as the L1 in that it can mediate semantic access for L3 word forms in a similar way.

In the next chapter, \textsc{The acquisition of clitic pronouns in complex infinitival clauses by German-speaking learners of Italian as an L3: The role of proficiency in target and background language(s), Sandro \citeauthor{chapters/sciutti}} reports findings from a study on the acquisition of clitic pronouns in Italian as an L3 by L1 speakers of German with L2 knowledge of either French or Spanish. Whereas Romance languages like Italian, French and Spanish display different series of clitic pronouns, these are not present in German. The participants in this study, 20 German-speaking learners aged between 20 and 47, were grouped on the basis of their proficiency level in Italian (intermediate or advanced) and categorized according to their L2, either French or Spanish. The learners who had French as L2 (n=10) were further divided into a low and a high proficiency group (5 in each group), with respect to their self-assessed knowledge of French. The same distinction was not applicable for the group with Spanish as L2 (n=5), whose self-assessed proficiency was generally high. The learners’ performances in three experimental tests in Italian – one elicited production, one grammaticality judgment and correction task, and one written translation task – were analysed to determine whether the acquisition of clitics in clauses with infinitives was affected by the proficiency level in L3 Italian, by the specific L2 (French or Spanish) or by the proficiency level in the L2 (because the learners of Spanish had all self-assessed their L2 proficiency level as high, this was only applicable for the learners with French as L2). The analysis focused on overall production and avoidance of clitic pronouns as well as on their forms and placement. Results show that the degree of proficiency in both L2 and L3 seems to be of importance for the acquisition of clitics. They are generally difficult to acquire and their many morpho-syntactic properties are generally not completely mastered at an intermediate level of Italian, where they are often omitted or replaced with lexical determiner phrases. Learners with advanced proficiency in the target language showed a better mastery of all the properties of clitisation than those with intermediate proficiency. This was true across all the experimental tasks. Furthermore, an examination of clauses containing an infinitive governed by a causative verb, (e.g., \textsc{\textit{lo} faccio lavare in lavanderia} – ‘I will have it washed in the laundry’) revealed that difficulties with the multifaceted phenomenon of Italian clitics may remain at advanced levels. As for proficiency in French or Spanish, high proficiency in a Romance L2 seems to play a positive role for the production of clitics and the reduction of their omissions in Italian as an L3. The higher the proficiency in the L2, the more prone the learners seemed to be to transfer their knowledge about the existence of clitics from one Romance language to another. Especially for the Italian partitive and locative clitics \textit{ne} and \textit{ci}, high proficiency in French, where similar forms that correspond syntactically to the Italian ones exist, seems to foster their production in L3 Italian and to reduce the number of omissions. This was, otherwise, a common strategy of avoidance in the case of other learners. Also for the position of clitics, proficiency in L2 French played a role. When comparing learners with high versus low proficiency in L2 French, it was found that higher proficiency in French generally led to more target-like instances in Italian. Generally, it can be concluded that a high proficiency in both French L2 and Spanish L2 may have the general effect of enhancing the acquisition of clitics in Italian L3.

Shifting the focus to young learners, chapter 5 presents a study by \textsc{Carmen \citeauthor{chapters/munoz}, Cognate recognition by young multilingual language learners. The role of age and exposure}. In this study trends already observed in previous work by \citet{Muñoz2006Book, Muñoz2014Complexities} are confirmed: age is an important factor for language learning, in the sense that older learners have cognitive advantages over younger learners and that metalinguistic skills that develop with age support language learning. \citeauthor{chapters/munoz} investigates the recognition of cognates by two groups of young bilingual learners of English as their first foreign language (EFL), one group of 7 year-olds and one of 9 year-olds. The study fills a gap concerning the role of cognateness in vocabulary recognition by bilingual children learning a foreign language to which they have limited exposure. As the author points out, it is commonly acknowledged that lexical similarity between known and new languages will facilitate additional language learning \citep{Ringbom2007} and that cognates between L1 and L2 are relatively easy to recognise and learn (e.g. \citealt{EllisNBeaton1993, DeGrootvanHell2005}). Moreover, this facilitative effect has been observed more often in older than in younger learners. The role of cognates in young learners’ foreign language learning has not been considered much in previous research, with noteworthy exceptions such as \citet{Otwinowska2016} and \citet{GoriotEtAl2018}. With this study, gathering evidence from learners of English as an L3 in the Spanish-Catalan context, new light is shed on young learners’ ability to recognise cognates in an additional language. The study explores phonological cognates and, in particular, the role that age and amount of exposure to the target language play in the ability to recognise them.

The research questions that guided the study concerned the extent to which bilingual EFL learners recognise cognate words over non-cognate words and the respective roles of age and amount of exposure to English in cognate word recognition and non-cognate word recognition. In order to answer the research questions, the study examined how often young learners – 170 Spanish-Catalan bilingual children – recognised cognates and non-cognates in the Peabody Picture Vocabulary Test (\citealt{DunnDunn2007}) in its oral form, which categorises them based on their etymology. The participants, evenly distributed in terms of age (7 vs. 9 year-olds in grades 2 and 4, respectively) and gender (males vs. females), had received different amounts of curricular exposure at school. Indeed, some of them were even attending a school that taught CLIL (content and language integrated learning), which increased their amount of instruction hours in English. Following the methodology employed in previous studies (as in \citealt{MuñozCadiernoCasas2018}), the analysis of the data relied on the total number of words heard, the total number of cognates and non-cognates, and the indexes of cognate and non-cognate recognition. The results indicated that cognates were more frequently recognised than non-cognates in both the examined age groups. Furthermore, the results conceded an advantage to the older children in benefitting from the facilitation of cognates, which may turn into an asset in foreign language classrooms. Thus, older learners, benefitting from positive transfer, were shown to better use their L1 vocabulary knowledge to identify and use target language vocabulary. While age was the strongest determinant of cognate recognition, hours of exposure was a stronger predictor of non-cognate recognition. The significant age effect on the ability to recognise cognates, which is in line with findings from previous research from Muñoz with young bilinguals and young foreign language learners, suggests that cognate awareness develops substantially between the ages 7 and 9. The possibility to dissociate age and contact hours in this study yielded evidence that the age effect was stronger for cognate recognition. As \citeauthor{chapters/munoz} concludes, both age groups showed a large and significant difference in the proportion of correct answers to cognate items and non-cognate items. However, the older group outperformed the younger one in both types. The explanation of the advantage of the older group, as suggested by the author, may be that with age they have developed a higher level of metalinguistic skills.

In order to obtain a deeper understanding of the age factor, \textsc{Simone \citeauthor{chapters/pfenninger}} investigates age effects on additional language learning by comparing early bilinguals on the one hand and later bilinguals and monolinguals on the other when learning EFL in the German-speaking area of Switzerland. In this study the heterogeneity of bilingual populations and the importance of distinguishing between different types of bilinguals are highlighted. In her chapter, \textsc{Age meets multilingualism: Influence of starting age on L3 acquisition across different learner populations}, \citeauthor{chapters/pfenninger} approaches two questions related to age: first whether early bilinguals are more successful than later bilinguals and monolinguals when learning a new language at school, and second how literacy skills in the home language (or languages), affect the development of literacy in the foreign language. In order to answer these questions, \citeauthor{chapters/pfenninger} conducted a longitudinal study in Switzerland, in which the English proficiency development of 636 secondary school students was assessed through a series of oral and written tests of receptive as well as productive language skills. All students learned standard German and French at primary school, but only half of them had studied English from the third grade; the others had started with English five years later. Home languages in the bilingual groups were Spanish, Portuguese, Croatian, Serbian, Albanian, Arabic or Italian. All participants were between 13 and 14 at the first data collection time and in the range 18--19 at the second time, which occurred five years later. The findings suggest that age of onset played a different role in the different groups: monolinguals, simultaneous bilinguals, and sequential bilinguals were affected differently by age of onset effects, due to individual differences and socio-contextual factors. The results of the analyses revealed that an earlier age of onset was only beneficial, across a range of measures of productive and receptive EFL skills, for one specific learner group: simultaneous biliterate bilinguals who received substantial parental support. Monolinguals and non-biliterate bilinguals did not display benefits from earlier age of onset in the same way. For early bilinguals, the importance of sociolinguistic and educational factors for success, such as parents’ support and positive attitudes towards language learning and multilingualism, use of both languages at home, and sustainment of L1 literacy skills in early school years, is clearly highlighted by the results.

The two final chapters of the volume investigate the L3 syntax of learners of foreign languages in middle and secondary school. In the chapter \textsc{From L2 to L3, verbs getting into place: A study on interlanguage transfer and L2 syntactic proficiency}, by \textsc{Laura \citeauthor{chapters/sanchez7}}, the participants are early bilinguals (that is, with two L1s) who learn two foreign languages (L2 and L3) in parallel. With a difference of at least three years in the age of onset of the L2 and the L3, the study explores the role of L2 proficiency for transfer into the L3. Relatively few studies have focused on the L2 proficiency factor, which however appears to condition transfer from one non-native language to another; see for example also \citeauthor{chapters/sciutti}’s study in this volume. Nonetheless, whereas \citeauthor{chapters/sciutti} addressed the effects of general self-perceived proficiency, the study by \citeauthor{chapters/sanchez7} focuses on the effects of proficiency at the level of syntax, which was measured on the basis of the learners’ written productive knowledge of a set of structural properties related to the V2 (verb second) rule present in German. Data were retrieved using a story-telling task (\citealt{SánchezJarvis2008}) from a data set of 280 Spanish/Catalan learners of L3 English with knowledge of L2 German, aged 9--13. While learning German and English simultaneously at school, the participants used Spanish- and/or Catalan, to varying degrees, in their everyday lives. They had started learning German when they were 5 years old in a programme that integrated language and content in some subjects. At the time of testing, both their overall proficiency in German (as determined by the ‘German Placement Test’) and their syntactic proficiency in the structures tested were still generally low. Subsequently, at the age of 8, or later, the participants had started learning English. Differences in age and L3 overall proficiency, measured by means of a cloze test, were controlled in the tests used for the statistical treatment of the data.

The study examined whether syntactic proficiency in German had an effect on the timing, extent and type of transfer from L2 to L3. The research is innovative in that it analyses analogous structures used in the L3 and the L2. As mentioned, the study looks into a cluster of structural properties related to the V2 (verb second) rule. The V2 rule yields three characteristic word orders in German that differ from English, namely subject-verb inversion, discontinuous verb placement and verb final. Results show that two of these structures chosen for examination, discontinuous verb placement and verb final, transferred from the L2 into the L3. Transfer of these structures was found at low levels of syntactic proficiency, but also when syntactic proficiency in the L2 was high, which suggests that the specific structural properties that may be transferred to the L3 may either be fully acquired in the L2 or in the process of being acquired. Methodologically, this study highlights the necessity, in research on interlanguage transfer of syntax, to test and determine learners’ knowledge of particular structures in the L2. In fact, low syntactic proficiency in the L2 seemed to favour activation and negative transfer from the L2, participants having difficulties inhibiting unintended activation of a previously built up interlanguage. This finding aligns with the claim in the chapter by \citeauthor{chapters/pfenninger} that unstable knowledge of the L2 has an effect on the learning of the L3. Furthermore, the results lend further support to the extension of \citeapo{Cummins1991} \textit{Interdependence Hypothesis} to L3 learning and multilingualism.

Partly similar age groups participate in the study reported in the final chapter, and syntactic problems closely related to those of the previous chapter are also investigated in \textsc{L1 Dutch vs L2 English in the initial stages of L3 French acquisition: The case of verb placement}, by \textsc{Rosalinda Stadt, Aafke Hulk and Petra Sleeman}. The general aim of this study was to define the role of native and non-native background languages in the very initial stages of learning an L3 in the classroom. The setting was Dutch secondary school and the first weeks of study of French, a suitable scene for investigating the potential influence of Dutch as L1 and English as L2 on French L3, which the participants were also acquiring in parallel under two different input conditions. The number of participants, 1\textsuperscript{st} year learners of English, was 23 (selected out of 118 possible candidates on the basis of a language background questionnaire and the Anglia placement test). Learners were classified into two groups depending on whether they were enrolled in the mainstream Dutch curriculum (\textit{n} = 11), or in a bilingual stream programme (\textit{n} = 12), where they were exposed to English more intensively. Two syntactic error types were analysed in order to detect transfer either from Dutch or from English: errors based on V2 surface structures in sentences containing a sentence initial adverb (which would stem from the L1 Dutch) and errors based on the Adv-V word order in the middle field of the sentence (which would stem from the L2 English word order). A considerable amount of transfer from the L1 was found in both reception data from a grammaticality judgement task and production data from a gap-filling task designed by the authors. In previous studies (\citealt{StadtEtAl2016, StadtEtAl2018Exposure}), the authors had found a stronger transfer effect from the L2 English on the L3 French, which could be explained with a higher amount of L2 exposure compared to the current study. Differently from the previous studies, the participants had not been exposed to English in the daily school context at the time when the study was conducted.

In the previous studies, the preferred role for the L2 as transfer source had been identified in later stages of L3 development with 3\textsuperscript{rd} and 4\textsuperscript{th} year students. Such a strong L2 effect was not found in the beginners participating in this study. In order to explain the predominant role of the L1 here, it is argued that in the initial state, the learners were unable to make assumptions about word order in French L3, but resorted to their L1 Dutch, hypothesizing that Dutch and French share the same word order. Furthermore, it is suggested that the L2 needs to be activated, through exposure, for the L2 to override the L1 as transfer source. Moreover, it is argued that the grammatical judgement task might have been too difficult for the learners, having to cope with reading skills and morphosyntactic knowledge in the target language that they did not possess yet.

In summary, the chapters of this volume present together a wide range of theoretical positions and empirical evidence that represent important aspects of current directions in the field of L3 research, touching upon different age groups and proficiency levels, looking into diverse linguistic phenomena and language combinations, and studying additional language learning from a perspective where all background languages potentially play a role. Research into third or additional language learning by young learners or adults who have previous experience of one or more languages learned as children or adults adds to our knowledge about non-native language acquisition. In fact, as testified in this volume, much L3 research is about reviewing old knowledge about second language acquisition in the light of factors that are of importance for the complex multilingual mind: the age of onset of the additional language and that of previously acquired languages, social and affective factors, instruction, language proficiency and literacy, the typology of the background languages and the role they play in shaping the syntax and the lexicon and other components of a third language. These factors and others are intertwined in an intricate way and the L3 research area continues to call for more studies. It is our hope that the variety of ideas and results presented here will contribute to the development of the field.

\sloppy\printbibliography[heading=subbibliography,notkeyword=this]\end{document}
