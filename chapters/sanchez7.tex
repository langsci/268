\documentclass[output=paper,modfonts,nonflat, newtxmath]{langsci/langscibook}
\ChapterDOI{10.5281/zenodo.4138747}
\author{Laura Sánchez\affiliation{Stockholm University}}
\title{From L2 to L3, verbs getting into place: A study on interlanguage transfer and L2 syntactic proficiency}
\abstract{One of the least understood (and also least investigated) factors conditioning interlanguage transfer from a prior non-native language (L2) to a subsequent non-native language (L3) is proficiency in the source language of influence. The present study examined whether L2 German syntactic proficiency, defined here as level of development in a cluster of structural properties related to verb placement, had any effect on the occurrence of interlanguage transfer in the L3 English. More specifically, the research question guiding the study asked whether mastery of these properties (discontinuous verb placement, verb final and inversion) constrained the timing, extent and type of such transfer. The corpus analyzed comprised L2 and L3 data from Spanish\slash Catalan bilingual learners ($n = 238$) aged 9--13. A series of ANCOVAs were run on the data, keeping constant various measures of L2 syntactic proficiency and controlling for L3 overall proficiency and biological age as covariates. The results of these tests yielded a significant effect for verb final ($p = 0.046$) and subject-verb inversion ($p = 0.002$). Furthermore, an inverse relationship was found between L2 syntactic proficiency and interlanguage transfer, in that low L2 proficiency in the use of verb final and (especially) inversion was associated with an overgeneralization of discontinuous verb placement in embedded clauses in the L3.}
% \keywords{ {(interlanguage)} {transfer,} {crosslinguistic} {influence,} {L3} {acquisition,} {L2} {German,} {L3} {English}
\shorttitlerunninghead{From L2 to L3, verbs getting into place}
\begin{document}
\maketitle


\section{Introduction}
Ever since the pioneering work of \citet{WilliamsHammarberg1998} back in the late nineties, a growing body of empirical studies has been published on the influence of a prior non-native language in the linguistic repertoire of a speaker (L2) upon another non-native language subsequently learnt (L3), commonly referred to as {interlanguage} {transfer} (\citealt{DeAngelisSelinker2001}). Less attention, however, has been paid to investigating the factors that may potentially favour or constrain the occurrence of this specific kind of transfer. Within this line of inquiry, the study reported in this chapter sets out to examine the effects of L2 syntactic proficiency (henceforth, L2SP) on the interlanguage transfer of linguistic structures from the L2 during L3 processing and production. More specifically, the study tries to ascertain whether different levels of L2SP may contribute to increasing or decreasing the chances of interlanguage transfer occurring from the L2 to the L3.

The data investigated in this study come from a larger corpus that comprises written data in L2 German and L3 English. This corpus has been investigated in a series of studies on lexical \citep{Sánchez2015L2} and syntactic transfer (\citealt{Sánchez2012, Sánchez2015Background}, among others). The main conclusion reached in these studies singled out the L2 as the primary source language of influence, overriding the first languages (L1s) (Spanish and Catalan). With this as the point of departure, the present study expands the scope of previous work by the author, examining the effect that different levels of L2SP may have on the {timing} (i.e., chronology), {extent} (i.e., frequency), and {type} of interlanguage syntactic transfer (henceforth, ILST).

\section{Background of the study}\label{sec:sanchez7:2}

\subsection{Typological contrasts in syntax between German and English}\label{sec:sanchez7:2.1}

In this study, the source language of influence (L2 German) and the target language (L3 English) differ in the crucial linguistic feature selected for investigation, i.e., verb placement. As far as this is concerned, German is classified as head-final \citep{Beck1998} and complements are pre-verbal (object-verb, OV). As a consequence, its basic word order is SOV\footnote{For theoretical arguments on the potential mixed headedness of German see \citet{Abraham1992}.}  (subject-object-verb). In contrast, the L3 English is head-initial and complements are post-verbal (subject-verb-object, SVO), the same as in the two L1s of the participants, Spanish and Catalan. At a surface level, the head-final feature is associated with a number of structural properties, above all, subject-verb inversion, discontinuous verb placement, and verb final. These properties are presented in \tabref{tab:sanchez7:1}, alongside the clause category (main vs. embedded) where they occur.

%%please move \begin{table} just above \begin{tabular
\begin{table}
\caption{German OV structural properties and related clause categories (main vs. embedded)\label{tab:sanchez7:1}}
\begin{tabularx}{\textwidth}{Q l Q}
\lsptoprule
 Structural property & Clause type & Description\\
 \midrule
Inversion rule (INV) & Main & Subject-verb inversion when the subject does not occupy the first position in the clause\\
\tablevspace
Rule of discontinuous verb placement (SEP) & Main & The finite verb occupies the second position in the clause (V2) and the non-finite verb occupies the last position in the clause\\
\tablevspace
Verb final (VFINAL) & Embedded & The verb always appears at the end of the clause. In periphrastic constructions, the non-finite form precedes the finite form\\
\lspbottomrule
\end{tabularx}
\end{table}

As a result of the head-final feature, in German, the finite verb always appears in second position in main clauses. Hence, when a constituent other than the subject occupies the first position (as in the example given in \ref{ex:sanchez7:1}), there is subject-verb inversion (INV). For the sake of clarity, the verb phrases in these examples, adapted from \citet{Sánchez2016}, are italicized.

\ea%1
  \label{ex:sanchez7:1}
  \gll Gerade \textit{schreibt} Peter einen Brief.\\
    at.this.moment writes Peter a letter\\
  \glt `Peter is writing a letter at this moment.’
  \z

Another structural property of main clauses is the discontinuity of verb placement or verb separation (SEP). The SEP property separates the finite and non-finite verbs in periphrastic constructions, that is, complex verb phrases with lexical and auxiliary verbs. Hence, SEP affects thematic verbs such as tense \REF{ex:sanchez7:2} and modal \REF{ex:sanchez7:3} auxiliaries. The following examples illustrate the surface word orders associated with SEP.

\ea%2
  \label{ex:sanchez7:2}
  \gll Peter  \textit{hat} gerade  einen Brief  \textit{geschrieben}.  \\
    Peter has at.this.moment a letter written \\
  \glt ‘Peter has just written a letter.’
\ex %3
  \label{ex:sanchez7:3}
  \gll Peter  \textit{kann} ins Kino  \textit{gehen}.  \\
    Peter can to the cinema go\\
  \glt ‘Peter can go to the cinema.’
\z


In turn, verb final (VFINAL) in embedded clauses moves the verb phrase to the end of the clause (as in \ref{ex:sanchez7:4}). In periphrastic constructions, the non-finite form precedes the finite form (as in \ref{ex:sanchez7:5}).

\ea%4
  \label{ex:sanchez7:4}
  \gll  daß Peter heute ins Kino  \textit{geht}.   \\
    that Peter today to.the cinema  goes\\
  \glt ‘that Peter is going to the cinema today.’
\ex %5
  \label{ex:sanchez7:5}
  \gll daß  Peter  heute  ins Kino  \textit{gegangen}  \textit{ist}.\\
    that  Peter  today  to.the cinema  gone is \\
  \glt ‘that Peter has gone to the cinema today.’
  \z

\subsection{The corpus} %2.2


As indicated in the introduction, the data of the present study come from a larger corpus of written data compiled in various studies on ILST from L2 German to L3 English. The examples below (adapted from \citealt{Sánchez2011, Sánchez2015Background, Sánchez2016}) illustrate the occurrence of ILST of two OV structural properties in the corpus, namely, SEP and VFINAL. The occurrence of ILST of SEP separated finite and non-finite verbs in periphrastic constructions, and it affected main clauses with tense (\ref{ex:sanchez7:6}--\ref{ex:sanchez7:7}) and modal (\ref{ex:sanchez7:8}--\ref{ex:sanchez7:9}) auxiliaries. The examples here and in the remainder of the chapter are presented as learners wrote them, so mispellings and other inaccuracies have been preserved.


\ea%6
  \label{ex:sanchez7:6}
   The dog  \textit{has} got the eat {\textit{geeat}}
\ex %7
  \label{ex:sanchez7:7}
   The mother  \textit{is} tea {\textit{preparing}}
\ex%8
  \label{ex:sanchez7:8}
   We  \textit{can} in the garden {\textit{walk}}
\ex%9
  \label{ex:sanchez7:9}
     They  \textit{willen} a Picknick {\textit{eat}}!
\z



Besides ILST, \REF{ex:sanchez7:6} and \REF{ex:sanchez7:9} exhibit lexical transfer as well, suggesting co-occur\-rence of transfer at the level of syntax and at the interface between lexis and morphology, leading to the creation of lexical inventions. Specifically, the forms *\textit{geeat} \REF{ex:sanchez7:6} and *\textit{willen} \REF{ex:sanchez7:9} are blends of an L3 English lexeme and an L2 German morpheme. The form *\textit{geeat} is a blend of the English lexeme \textit{eat} and the German prefix \textit{ge-} (used for the formation of the past participle), and it represents a productive word formation pattern in clauses that exhibited ILST (along with others such as *\textit{geluncht}, *\textit{geeats}, *\textit{geeating}, *\textit{geeate}, *\textit{geeiten} or *\textit{geeatet}, to name just a few; see \citealt{Sánchez2015L2} for a full inventory). In turn, the form *\textit{willen} is a blend of the English lexeme \textit{will} and the German suffix \textit{-en}, used here for the formation of the 3rd person plural in present simple (the same as forms such as \textit{runnen}, \textit{walken} or \textit{maken}).

In some cases, ILST of SEP co-occurred also with INV, as in \REF{ex:sanchez7:10} and \REF{ex:sanchez7:11} below.

\ea%10
  \label{ex:sanchez7:10}
  Then  \textit{have} Tom the water \textit{genommen}\footnote{\textit{genommen} is a borrowing from the L2 German (equivalent to L3 English ‘taken’).}
\ex %11
  \label{ex:sanchez7:11}
 	and then  \textit{hat} a dog in then basket {\textit{go}}\footnote{\textit{hat} is a borrowing from the L2 German (equivalent to L3 English ‘has’).}
\z

On other occasions, SEP was also transferred to embedded clauses with periphrastic constructions, as in examples (\ref{ex:sanchez7:12}--\ref{ex:sanchez7:15}). \citet{Sánchez2016} explains this as an overgeneralization of SEP to linguistic contexts where VFINAL would have applied in the L2 German. The following examples show the ILST of SEP in embedded clauses of different types, above all nominal, causal, and relative.

\ea%12
  \label{ex:sanchez7:12}
  because the dog  \textit{has} he’s dinner {\textit{eat}}
\ex %13
  \label{ex:sanchez7:13}
  that the dog \textit{has} the lunch {\textit{eating}}
\ex%14
  \label{ex:sanchez7:14}
  who  \textit{has} the bread and the cake { \textit{eat}}
\ex%15
  \label{ex:sanchez7:15}
  because the dog  \textit{is} the food { \textit{eating}}
\z

As regards VFINAL, (\ref{ex:sanchez7:16}--\ref{ex:sanchez7:19}) below illustrate the occurrence of ILST of VFINAL in embedded clauses with both simple verb phrases (\ref{ex:sanchez7:16}--\ref{ex:sanchez7:17}) and periphrastic constructions (\ref{ex:sanchez7:18}--\ref{ex:sanchez7:19}). The latter co-existed with embedded clauses that exhibited an overgeneralization of SEP, as in examples (\ref{ex:sanchez7:12}--\ref{ex:sanchez7:15}) above.

\ea%16
  \label{ex:sanchez7:16}
  When the brothers in the picknick {\textit{are}}
\ex %17
  \label{ex:sanchez7:17}
  when the kids with his mother {\textit{spoke}}
\ex %18
  \label{ex:sanchez7:18}
  	because the dog the sandwich {\textit{eat}} {\textit{has}}
\ex %19
  \label{ex:sanchez7:19}
  that their dog in the bascket {{had}} ({{ponerse} {dentro}}\footnote{L1 Spanish \textit{ponerse dentro} can be roughly translated into English as `put (oneself) in or inside'. The learner in \REF{ex:sanchez7:19} resorts to a borrowing or code-switch as a compensatory strategy for the lack of the appropriate lexical item in the target language (L3 English). Interestingly, this borrowing is inserted in a clause whose frame corresponds to the L2 German and shows ILST of VFINAL as it would occur in clauses with periphrastic constructions in this language.})
\z


\subsection{{L2} {proficiency} {and} {interlanguage} {transfer} {in} {L3} {learning}}%2.3


The role of the proficiency factor on the occurrence of ILST in this corpus has been examined in various studies, focusing on the role of both L3 proficiency \citep{Sánchez2014} and of L2 proficiency (\citealt{Sánchez2011, SánchezBardel2017}). As for L2 proficiency, which is the focus of this chapter, these two studies employed different instruments to measure it and operationalized it in different ways. On the one hand, \citet{SánchezBardel2017} examined the role of {overall} proficiency, which they defined, following guidelines in the CEFR, as “competence put to use” (\citealt[187]{CouncilofEurope2001}). Hence, it was defined as global within the system “as a whole” (p. 42), and it was measured using a standardized general proficiency test, i.e. the German placement test. Both \citet{Sánchez2011} and \citet{SánchezBardel2017} concluded that this factor constrained the timing and extent of ILST.

\subsubsection{Interlanguage transfer at low L2 proficiency levels}%2.3.1

In this respect, \citet[24]{MontrulEtAl2011} claim that L2 proficiency “matters for the timing and extent of transfer in L3 acquisition”, and therefore “studies should also control for proficiency levels in the L2”. The highest incidence of ILST in previous studies (\citealt{Sánchez2011,SánchezBardel2017}) was concentrated primarily at low and (to a somewhat lesser extent) intermediate L2 proficiency levels, and both identified the achievement of an intermediate proficiency in the source language of influence as a turning point that triggered a linear fall in the extent of ILST in the L3. Additionally, based on the assumption that ILST occurred primarily at lower L2 proficiency levels, \citet{Sánchez2011} concluded that a high proficiency level in the source language of influence may not be a prerequisite for ILST to occur, thereby lending support to theoretical claims in previous studies (\citealt{DeAngelisSelinker2001,DeAngelis2007,Ringbom2007,Rast2010,Sánchez2012}). This finding was later confirmed by \citet{SánchezBardel2017}, who found that a low overall proficiency in the source language of influence suffices to exert a powerful impact on the L3.

To explain the relationship between low L2 proficiency and the extent of interlanguage transfer, an argument that has been cited is that shortcomings in L2 proficiency may cause a failure to effectively inhibit unintended language activation of the L2 during L3 processing and production and, consequently, may lead to a higher incidence of transfer (\citealt{Shanon1991,Dewaele2001,DeBot2004}). More specifically, it has been suggested that interlanguage underdevelopment may be the driving force behind transfer, “putting the target language (L3) at a higher risk of being influenced by another non-native language” (\citealt[241]{SánchezBardel2017}; also \citealt{Leung2003,Leung2005,Sánchez2011}). At the same time, however, it is also possible that interlanguage transfer might be more likely when the L2 proficiency of the learner is high (\citealt{Odlin1990,Dewaele1998,Dentler2000,Hammarberg2001,Ringbom2001,Ringbom2007}). Taking \citegen{Bouvy2000} claim that at least some knowledge of the element transferred from the L2 is necessary for transfer to occur, it makes sense to believe that some knowledge and control of linguistic structures at the level of syntax is a developmental prerequisite for the transferability of these structures. From this it would follow that advances in L2SP may have implications for the occurrence of ILST, as developmentally-related changes in the mental representation of properties of the L2 and their transferability may affect the extent of ILST and L3 interlanguage development \citep{Sánchez2011}. After all, as \citet[40]{HerdinaJessner2002} indicate, a significant change in proficiency in a language “will affect the development of LS\textsubscript{1} [language system 1], LS\textsubscript{2} [language system 2], etc.”

The relationship between L2SP and ILST depicted here resembles \citeapo{Cummins1979} {developmental} {interdependence} hypothesis, which postulates that proficiency in the L1 and the L2 are interrelated. According to Cenoz, this hypothesis might be extended to multilingual learning, especially in situations of asymmetric proficiency and development in two or more languages. She further argues that this hypothesis may be used to investigate the relationship between L2 and L3 proficiency (see also \citealt{Abunuwara1992, Chumbow1981, Thomas1988}), because “different degrees of proficiency in the first and second languages would affect the acquisition of the third (or fourth) language” \citep[46]{Cenoz2000}.

The approach taken to the interdependence of the L2 and the L3 and to the constraints of L2 proficiency on the occurrence of ILST in the studies by \citet{Sánchez2011} and \citet{SánchezBardel2017} cited above is heuristic, as it looks at L2 proficiency holistically. Holistic approaches highlight the unique configuration of multicompetence \citep{Cook1992} or composite competence (\citealt[60]{CouncilofEurope2001}) typical and exclusive of multilingual learners. Another way of looking at L2 proficiency in the corpus investigated here would be to adopt a different perspective by focusing not on general proficiency, but on proficiency in relation to specific aspects of language proficiency \citep{North1997}. This other way of examining L2 proficiency corresponds to analytic approaches (\citealt{HerdinaJessner2002, Stratilaki2006}), which try to identify the differential roles of each background language. Both heuristic and analytic approaches offer distinct yet complementary approaches, or as \citet[242]{SánchezBardel2017} propose, they would be different sides of the same coin. Hence, if the findings reported in \citet{Sánchez2011} and \citet{SánchezBardel2017} – both of which adopt a heuristic approach – are only one side of the coin, it is hoped that the other side of the story might be told in the present, analytic, study. Hence, as \citet{SánchezBardel2017} conclude, further research is necessary in order to clarify whether the decreases observed in interlanguage transfer in general (and in ILST in particular) are “caused by advances in general development in this language [the L2] or in domain-specific knowledge of the targeted structures in one or both languages” (\citealt[244]{SánchezBardel2017}).

To be able to objectively measure the effects of L2 proficiency at a domain-specific level (in syntax, in this case), “learners should be tested on the structure being studied [in the L3] in the background languages” (\citet[197]{FalkBardel2010}, and using “more precise measures” to obtain a better understanding of this factor (p. 211). The reasoning behind these considerations is consistent with \citeapo{Jarvis2000} claim that, for a rigorous analysis and identification of transfer, the design of the study needs to include analogous data both in the source and target languages. Furthermore, the data need to allow for a comparison that appraises the intra-group congruity of the learners’ linguistic performance in the source and the target languages in relation to a given linguistic feature. In L3 acquisition, this comparison involves searching for similarities between the L2 and L3 interlanguage performances of a given group of learners. Thus, this type of comparison is important because it reveals what it is in the L2 that motivates the L3 interlanguage performance (see \citealt[255]{Jarvis2000}). This is precisely what the present study attempts to do, first by examining how L2 proficiency at a domain-specific level in syntax (namely, L2SP) affects the occurrence of interlanguage transfer in the L3 also at this same level (i.e., ILST). Secondly, the study is based on the comparison of performance in L2 German OV structures with ILST of these properties in the L3 English.

Bearing in mind these considerations, the study presented in this chapter embraces a narrower definition of specific proficiency (L2SP) in the selected linguistic feature in the L2, namely, the OV structural properties of the L2 German. By adopting an analytic approach, a novelty of the study is that the methodology makes it possible to more easily tease apart and isolate differential effects for the source language of influence. This narrow, more focused account and measurement of L2 proficiency is consonant with the “assumed level of acquired knowledge” in \citegen[57]{HerdinaJessner2002} definition of proficiency in each language of the multilingual learner. Likewise, this account of proficiency is consonant also with \citegen[199]{Leung2003} definition of proficiency as knowledge in relation to “the steady state of a previously acquired (inter)language” (emphasis added; see also \citealt[40]{Leung2005}). Besides, as far as ILST is concerned, she further argues that what is transferred at the outset of L3 learning is fundamentally this knowledge, i.e. the steady state of previously acquired interlanguages. Along similar lines, \citet[115]{DeAngelis2007} points out that what the learner transfers to the L3 and what affects performance in this language is “the linguistic development reached in one language”.

\section{The present study and research question}
\label{sec:sanchez7:3}

The study presented here is a contribution to research on the role played by L2 proficiency on the occurrence of ILST in L3 learning, and the data analyzed are part of the corpus compiled by the author and described in detail in the background of the study. Adopting an analytic approach, the study sets out to examine how L2 proficiency at a domain-specific level and in relation to particular linguistic forms (OV structural properties) may affect the likelihood of occurrence of ILST. With this in mind, the research question guiding the study asks whether L2SP, defined here as mastery of the OV structural properties (SEP, VFINAL, INV) in the L2, has any effect on the occurrence of ILST in the L3. The research question has been formulated as follows:

\begin{description}
\item[RQ:] Does mastery of the OV structural properties in the source language of influence (L2 German) constrain the timing, extent, and type of ILST in the target language (L3 English)?
\end{description}

\section{Participants}

In order to answer the research question guiding the study, data were used from Spanish/Catalan learners (aged 9--13) of L3 English with prior knowledge of L2 German ($n = 280$), which they were learning simultaneously. They were born in Spanish and/or Catalan speaking homes and used these languages to different degrees in their everyday lives, although both languages are official and co-exist at the community level. Moreover, their parents were native speakers of these languages and most of them had little or no knowledge of German. The learners had started learning the L2 German at school at the age of 5, in a programme that combined language learning and content in subjects such as history, geography or arts. Exposure to this language was virtually limited to the school, with the exception of occasional extracurricuar activities organized by the school. Despite this early starting age and the type of exposure to this language, their overall proficiency in this language was generally low (\citealt{SánchezBardel2016}), as assessed by the standardized proficiency test employed in previous studies with this corpus and discussed in the background of the study (see \citealt{SánchezBardel2017} for a more detailed account). In contrast to the homogeneous exposure to the L2 in terms of formal input and starting age, the situation for the L3 English was more heterogeneous. The learners started learning English when they were 8 years old (or later), within a communication-oriented teaching programme. It is important to point out that younger learners did not necessarily have less exposure to the L3 English than their older peers. This is because recent changes in the school curriculum had moved up the starting age and the hours of instruction per week. Hence, it was possible for different-aged learners to have received the same amount of instruction, or even that younger learners were somewhat more instructed than their older peers. In any case, participants took only one to two 50-minute English lessons per week, so they had received a maximum of 165 hours of instruction in this language by the time the data were collected. The overview of the partipants, presented in \tabref{tab:sanchez7:2}, aims to outline their linguistic profile. Hence in this table (but not in the data analysis) they are grouped according to age and instructional time. However, in order to prevent any undesired variability in the results caused by these differences, the learners’ overall proficiency in the L3 and their biological age were controlled for (see \sectref{sec:sanchez:7}).

\begin{table}
\caption{Overview of the participants\label{tab:sanchez7:2}}
\begin{tabular}{r c c c}
\lsptoprule
Mean age & $n$ & L3 instructional time (hrs) & Age of onset\\
\midrule
9.9 & 70 & 33--66  & 8--9\\
10.9 & 50 & 33--66 & 9\\
11.9 & 50 & \hphantom{1}99--165  & 9\\
12.9 & 56 & 132--165 & 9--11\\
13.9 & 54 & 132--165 & 9--11\\
\lspbottomrule
\end{tabular}
\end{table}

\section{{Instruments} {and} {data} {collection} {procedures} }
\label{sec:sanchez7:5}

The battery of tests employed in the data collection consisted in a background language questionnare to select the participants, a picture story-telling task to elicit written production in the L3 English and L2 German, and a cloze test to measure their overall proficiency in the L3 English. It is important to clarify that even though proficiency in this language was not the factor investigated here, it was included in the study design because L3 proficiency was used as a control variable in the statistical tests run on the data, as explained below. All the tests were administered in class time and in the presence of the researcher. The data from each participant were collected in two sessions. In the first session, the learners filled in the questionnaire, carried out the narrative task in either English or German, and completed the cloze test. In the second session, they performed the narrative task in the other non-native language.

A questionnaire was administered to intact classes to select the participants. The main criteria for inclusion in the study were (i) for the learners to be native speakers of Spanish and/or Catalan, (ii) that German preceded English in the chronological order of acquisition, and (iii) that they would not speak any other language. This questionnaire was originally inspired by an already existing bilingual questionnaire (i.e., \citealt{BakerPrys1998}) used in the BAF Project by the GRAL research group\footnote{The BAF Project (\textit{Barcelona age factor}) was conducted within the GRAL research group (\textit{Grup de recerca d’adquisició de llengües}), of which the author is a collaborator.}, but it had been piloted and validated by the author \citep{Sánchez2011} to satisfy the needs specific to the multilingual acquisition setting analysed here, where the participants were trilingual (for further details on this questionnaire, see \citealt{Sánchez2015L2}).

The story-telling task used for the elicitation of written data in the L3 (and in the L2 for comparison purposes and to measure L2SP) was \textit{The dog story} \citep{Heaton1966}. It is based on visual stimuli in a picture series that comprises six strips, and it has proved suitable for the investigation of transfer (\citealt{SánchezJarvis2008}). As discussed earlier, having analogous data not only from the L3 but also from the L2 was necessary for appraising the congruity of the learners’ interlanguage performance \citep[55]{Jarvis2000} in the source and target languages in relation to the structural properties investigated here. By so doing, it would be possible to identify what it was in the L2 that motivated the L3 ILST behaviour. For this same reason, data could be analyzed only from participants who would produce the targeted linguistic feature both in the L3 and the L2. This could not be guaranteed beforehand because the uncontrolled nature of the story-telling task did not force learners to use the targeted feature or any other feature in particular. The administration of the writing task was counterbalanced in the two languages,  to avoid any order effect on performance. The task was time-controlled (ca. 15 minutes), and participants were not allowed to ask questions related to the vocabulary of the story, nor were they permitted to use a dictionary or any other reference tool.

With the purpose of having a proficiency measure in the L3 English, the participants carried out a 30-item cloze test based on the \textit{Little red riding hood}. The decision to use this test was grounded on strong beliefs that it is an indicator of overall proficiency in a foreign language (e.g. \citealt{Hanzeli1977, KatonaDornyei1993}). In addition, the cloze test had been validated before using it here \citep{Muñoz2006} by means of a reliability test conducted on data from the BAF Project in the GRAL research group. This reliability test had shown high and significant correlations of the results from the cloze test with other proficiency measures from the battery of tests employed in that project assessing auditory, phonological, grammatical, and lexical development, along with receptive and productive abilities. By using the cloze test to measure the overall L3 proficiency of the participants in this study, it was possible to compare learners of the same and different ages and instructional times.

\section{Data analysis}

The design of the study included a dependent variable that registered the raw frequency of occurrence of ILST in the learners’ L3 production, an independent variable of L2SP consisting of three measures, and two covariates. The inclusion of the covariates in the design was justified by the need to tease apart the effects of L2SP from other confounding factors it may interact with. Here these factors were the overall L3 proficiency and the age of the participants. Thus, the statistical test chosen in order to find out whether L2SP had an effect on the occurrence of ILST was the analysis of covariance (ANCOVA).

The qualitative analysis of ILST in the present study was based on data elicited by means of the story-telling task (i.e., the L3 English version), and it targeted two OV structural properties, namely, SEP and VFINAL. The reason to select these properties (but not INV) was that these two show the largest, more salient and unambiguous contrast in surface word order between the L2 German and the L3 English (\citealt[293]{Gawlitzek-MaiwaldTracy2005}), and also between the L2 German and the two L1s. More importantly, previous studies with this corpus revealed the virtual absence of inverted orders in the L3 English data (\citealt{Sánchez2010, Sánchez2011, Sánchez2012}), with a few exceptions such as those shown in \REF{ex:sanchez7:10} and \REF{ex:sanchez7:11} above. Hence, it would not have been possible to examine ILST of INV in this study.

The measurement of L3 overall proficiency was based on data from the cloze test. The cloze tests were scored by the researcher, and the raw scores\footnote{The mean score for correct answers was 8, and the maximum number of correct answers was 24 (out of the 30 items the cloze test consisted of).} recorded in a quantitative variable. The raw scores were saved as $z$-scores, and the variable recording these $z$-scores was later entered as a covariate in the ANCOVA. In turn, the measurement of L2SP was based on data elicited by means of the story-telling task (i.e., the L2 German version). The measurement of the L2SP of the participants relied on the assessment of the accuracy of their performance in the cluster of OV structural properties observed in the L2 German. Measuring L2 proficiency precisely on the basis of performance in relation to these structures made it possible to have a more sound definition of L2 proficiency in relation to domain-specific knowledge. On the other hand, by gathering data in the source language of influence and in the target language, it was possible to test the participants on the structure being studied both in the L3 and the L2.

Since L2SP was going to be used as the fixed factor in the ANCOVA test, L2SP was operationalized as a categorical variable (which was necessary because it was included in the ANCOVA as a fixed effect). To this aim, L2SP was measured using dichotomous variables that coded the learners’ accurate or inaccurate use of SEP, VFINAL, and INV in three separate variables. Because the uncontrolled nature of the story-telling task did not force learners to use the targeted linguistic feature, some learners did not use any of these structural properties, some used only some of them, and others all three. On the other hand, because the writing task was time-limited and the participants’ overall proficiency in this language was low, as pointed out earlier, the length of their texts was relatively short (the mean and maximum lengths of the texts, for example, were 91 and 251 words, respectively). Likewise, their interlanguage was still under development and exhibited the unsystematic co-occurrence of grammatical and ungrammatical syntactic constructions. In any case, what was important for the purposes of the present study was to detect inaccurate uses of these structures in the L2, to be able to compare them with the transferred structures in the L3.

In each of the three dichotomous variables, if all the instances of the structural property in question were correct, the learner was given a 1. On the contrary, if one or more of these instances were incorrect, the learner was given a 0. Also for each variable separately, learners who did not use any of the properties were coded with a 2 and removed from subsequent analyses. These three variables were later used in order to divide the sample into two levels of L2SP in each of them, that is, low and high (for those coded 0 and 1, respectively). The breakdown of participants into two proficiency levels for each OV structural property is presented in \tabref{tab:sanchez7:3}.

This table shows the number of participants per level and structural property, and also the percentage they represented in each case. As can be seen, there was a ceiling effect in SEP, because nearly all participants were assigned to the high proficiency group (98.1\%). The assignment of participants to different proficiency levels in VFINAL was also unbalanced (77\% in the high proficiency group), though not as much as in SEP. A more balanced distribution of participants was found in INV (with 49.2\% and 50.8\% in the low and high proficiency groups, respectively). The data from the 280 participants in the study were examined. Even though the vast majority of them used SEP, VFINAL and/or INV in their L2 German story-telling task, 42 participants had to be removed from the working sample because they did not produce analogous structures in the the L3 English and L2 German, which was a necessary condition for the analysis of L2SP and ILST. Therefore, the final number of participants in the study was 238.

\begin{table}
\caption{Participants’ classification according to L2SP in each OV structural property\label{tab:sanchez7:3}}
\begin{tabular}{l r S[table-format=2.1] rS[table-format=2.1]r}
\lsptoprule
&\multicolumn{2}{c}{Low} &\multicolumn{2}{c}{High} &\\\cmidrule{2-3}\cmidrule{4-5}
Structural property & $n$ & {\%} & $n$ & {\%} & Total\\
\midrule
SEP    & 4 & 1.9 & 208 & 98.1 & 212\\
VFINAL & 50 & 23 & 167 & 77 & 217\\
INV    & 117 & 49.2 & 121 & 50.8 & 238\\
\lspbottomrule
\end{tabular}
\end{table}

\section{Results}

The report of the results is presented in what follows. It starts with an overview of the data to illustrate the different types of ILST encountered in the data and how they patterned across different levels of L2SP in each structural property. This is followed by an explanation of the results obtained in the series of ANCOVAs that were run on the data with the purpose of ascertaining whether L2SP had an effect on ILST.

The types of ILST encountered in the data included SEP in main clauses (as in \ref{ex:sanchez7:6}--\ref{ex:sanchez7:9} in \sectref{sec:sanchez7:2}), VFINAL in embedded clauses with simple verb phrases (\ref{ex:sanchez7:16}--\ref{ex:sanchez7:17}) and periphrastic constructions (\ref{ex:sanchez7:18}--\ref{ex:sanchez7:19}), and also overgeneralization of SEP in embedded clauses with a periphrastic construction (\ref{ex:sanchez7:12}--\ref{ex:sanchez7:15}). This last type of ILST is of interest here because it reflects that what learners were able to transfer was often an L2 rule still under development. For the present purposes, what is relevant is the comparison of analogous data by the same participant not only in the target language (the L3 English), but also in the source language of influence (the L2 German). The examples below, (\ref{ex:sanchez7:20}--\ref{ex:sanchez7:23}), illustrate this type of ILST in the L3 (i.e. overgeneralization of SEP in embedded clauses) along with the participants’ use of the corresponding structure in the L2. As can be seen in these examples, some of the participants did not have full mastery of VFINAL in embedded clauses in the L2. Consequently, instead of applying the VFINAL rule to embedded clauses, they overgeneralized SEP in the L2, and this is also what they transferred to the L3. Each pair of examples illustrates data from one and the same participant, with the purpose of showing congruence in the production in the L2 and the L3.

\ea%20
  \label{ex:sanchez7:20}
  \begin{xlist}
  \ex L3 English: because they \textit{will} your pastel\footnote{\textit{pastel} is the L1 Spanish term for L3 English `cake'.} \textbf{\textit{eat}}

  \ex L2 German:\\
  \gll {\dots} und sehen das seinen Hund \textit{hat} alles  \textit{gegesen}\\
     {\dots} and see   that their dog   has  all  eaten \\
  \glt ‘{\dots} and see that their dog has eaten all’
  \end{xlist}
\ex %21
  \label{ex:sanchez7:21}
  \begin{xlist}
  	\ex L3 English: beacos the dog \textit{has} the food \textbf{\textit{eat}}
  	\ex L2 German:\\
  	\gll das  der Hund  {hat}  das essen  {gegessen}\\
  		that  the dog   has  the food  eaten \\
  	\glt ‘that the dog has the food eaten’
  \end{xlist}
\ex %22
  \label{ex:sanchez7:22}
  \begin{xlist}
  	\ex L3 English:  weil\footnote{\textit{weil} is the L2 German term for English `because'.} the dog \textit{has} the food \textbf{\textit{eat}}

  	\ex L2 German:\\
  	\gll weil der Hund \textit{hatte} alles  \textit{gegessen}\\
  		because  the dog   had   all   eaten\\
  	\glt ‘because the dog had eaten all’
  \end{xlist}
\ex %23
  \label{ex:sanchez7:23}
  \begin{xlist}
  	\ex L3 English: because the dog \textit{had} the sandwiches \textbf{\textit{eated}}

  	\ex L2 German:\\
  	\gll weil     der Hund  \textit{hat}  die Sandwiches  \textit{gegessen}\\
   because  the dog   has  the sandwiches  eaten\\
  	\glt ‘because the dog has eaten the sandwiches’
  \end{xlist}
  \z

Similarly, the incomplete mastery of INV in the L2, the source language of influence, was related to a higher incidence of ILST in the L3. Specifically, learners who exhibited uninverted orders in the L2 (nearly half the sample 49.2\%, as can be seen in \tabref{tab:sanchez7:3}) also showed evidence of ILST in the L3 involving SEP in main clauses (ex. \ref{ex:sanchez7:24}--\ref{ex:sanchez7:25}), VFINAL in embedded clauses with simple verb phrases (\ref{ex:sanchez7:26}--\ref{ex:sanchez7:27}) and overgeneralization of SEP in embedded clauses (\ref{ex:sanchez7:28}--\ref{ex:sanchez7:29}).

\ea%24
  \label{ex:sanchez7:24}
  \begin{xlist}
  	\ex L3 English: Then dog has got the eat \textbf{\textit{geeat}}
  	\ex L2 German:\\
  	\gll Am letztes Bild die zwei Kindern \textit{sehen} das…\\
  	in.the last picture the two children see that\\
  	\glt ‘In the last picture the two children see that…’
  \end{xlist}
\ex %25
  \label{ex:sanchez7:25}
  \begin{xlist}
  	\ex L3 English: Theyr mother{is} they a map \textbf{\textit{giving}}
  	\ex L2 German:\\
  	\gll Eines Tages zwei Kinder \textit{wollten} ein picknick machen\\
  	one day two children wanted a picnic {to make}\\
  	\glt ‘One day two children wanted to make a picnic’
  \end{xlist}
\ex%26
  \label{ex:sanchez7:26}
  \begin{xlist}
  	\ex L3 English:when they with the mum \textbf{\textit{talk}}
  	\ex L2 German:\\
  	\gll dannach Geschister \textit{nahmen} den Korb\\
  		later {the siblings} took the basket\\
  	\glt ‘later the siblings took the basket’
  \end{xlist}
\ex %27
  \label{ex:sanchez7:27}
  \begin{xlist}
  	\ex L3 English: when the dog the sandwiches \textbf{ \textit{eat}}
  	\ex L2 German:\\
  	\gll wenn sie mit seiner Mutter waren, der Hund \textit{sprang}…\\
  		when they with their mother were, the dog jumped…\\
  	\glt ‘when they were with their mother, the dog jumped…’
  \end{xlist}
\ex %28
  \label{ex:sanchez7:28}
  \begin{xlist}
  	\ex L3 English: that the dog \textit{has} all the lunch \textbf{\textit{eating}}

  	\ex L2 German:\\
  	\gll {\dots} dann \textit{die} \textit{Mutter} \textit{gibt} ihnen ein bisschen essen\\
  	…then the mother gives them a little bit of food\\
  	\glt ‘…then the mother gives them a little bit of food’
  \end{xlist}
\ex %29
  \label{ex:sanchez7:29}
  \begin{xlist}
  	\ex L3 English: because the dog \textit{have} the food \textbf{\textit{eat}}
  	\ex L2 German:\\
  	\gll Am Schluss die Kinder \textit{haben} nichts zu essen\\
   		in.the end the children have nothing to eat\\
  	\glt ‘In the end che children have nothing to eat’
  \end{xlist}
  \z


\tabref{tab:sanchez7:4} shows the different types of ILST encountered in the data, alongside their raw frequency and percentage of occurrence in the data. As the data in the table below show, ILST in main and embedded clauses occurred to nearly the same extent. However, in embedded clauses ILST patterned differently depending on whether it involved VFINAL in a simple verb phrase (28.1\%) or an overgeneralization of SEP in periphrastic constructions (22.1\%). In contrast, the occurrence of VFINAL in periphrastic constructions was only marginal (0.7\%), because the participants overgeneralized SEP instead.

\begin{table}
\caption{Types of ILST in the L3. Raw frequency and percentage\label{tab:sanchez7:4}}
\fittable{\begin{tabular}{l r r}
\lsptoprule
	& \multicolumn{2}{l}{Occurrences of ILST}\\
\cmidrule{2-3}
Structural property & {Freq.} & \%\\
\midrule
SEP (main clauses) & 138 & 49.1\\
\tablevspace
VFINAL with simple VPs (embedded clauses) & 79 & 28.1\\
\tablevspace
VFINAL with periphrastic constructions (embedded clauses) & 2 & 0.7\\
\tablevspace
Overgeneralisation of SEP (embedded clauses) & 62 & 22.1\\
\lspbottomrule
\end{tabular}}
\end{table}

The rest of the section summarises the results from a quantitative perspective. \tabref{tab:sanchez7:5} presents the descriptive statistics of the occurrence of ILST in the L3 at different levels of L2SP (raw frequency of occurrence, mean and standard deviation) in each structural property in the L2. A close inspection of the means reveals that ILST consistently occurred more frequently in low than in high levels of L2SP, and this is true regardless of the L2 structural property examined.

\begin{table}
\caption{Descriptive statistics of the occurrence of ILST in the L3 across different levels of L2SP in each property\label{tab:sanchez7:5}}
\begin{tabularx}{\textwidth}{XXrrrr}

\lsptoprule
 &	& & \multicolumn{3}{c}{Occurrences of ILST}\\
\cmidrule{4-6}
\multicolumn{2}{l}{Structural property} & \# of learners at each L2SP level & {Freq.} & {M} & {SD}\\
\midrule
SEP & Low &  4 &  7 &  1.75 & 0.5 \\
 	& High & 208 & 241 & 1.16 & 1.11\\
\tablevspace
VFINAL & Low &  50 &  81 &  1.62 &  1.19\\
		& High & 167 & 179 & 1.07 & 1.06\\
\tablevspace
INV & Low & 117 &  166 &  1.42 &  1.13\\
	& High & 121 & 115 & 0.95 & 1.02\\
\lspbottomrule
\end{tabularx}
\end{table}

For a more comprehensive view of the picture, a more exhaustive analysis of other aspects related to the occurrence of ILST is now offered. To gain further insight into the relationship between L2SP and ILST (or lack thereof) in the L3, another way of looking at the data was to examine the non-occurrence of ILST in empirically-relevant contexts where it could occur and how this non-occurrence related to L2 proficiency levels in each structural property. Let us remind the reader that, as indicated in the background of the study, contexts where the targeted linguistic structures could be observed included simple verb phrases in embedded clauses, and periphrastic constructions in both main and embedded clauses. They constituted the empirically-relevant contexts for the analysis of ILST. \tabref{tab:sanchez7:6} presents the distribution of participants who transferred (``transfer condition'') and those who did not (``no transfer condition'') in the L3 at different levels of L2SP in each structural property.

 The information is to be read vertically and across conditions. The figures on VFINAL indicate that at low levels of L2SP, the vast majority of the participants transferred (82\%), whereas only 18\% did not. A very similar pattern was observed for INV also at a low level of L2SP (80.3\% who transferred vs. 19.7\% who did not). In turn, at high levels of L2SP the percentage of learners who do not transfer is somewhat higher than at low levels, both for VFINAL (35\%, against the 18\% at low levels) and for INV (43\%, against the 19.7\% at low proficiency levels). The same distribution of participants across transfer conditions was found for SEP. However, because of the unbalanced distribution of participants in the low ($n = 4$) and high ($n = 208$) levels of L2SP in this structural property the percentages are not representative, and the means in \tabref{tab:sanchez7:5} may be more informative.

\begin{table}
\caption{Distribution of participants across transfer conditions at different levels of L2SP in each property\label{tab:sanchez7:6}}
\begin{tabular}{l *{6}{S[table-format=2.1]}}
\lsptoprule
	 & \multicolumn{2}{c}{L2 SEP} & \multicolumn{2}{c}{L2 VFINAL} & \multicolumn{2}{c}{L2 INV}\\

%\cmidrule{2-7}
Condition & {Low} & {High} & {Low} & {High} & {Low} & {High}\\
\midrule
Transfer & & & & & & \\
 $n$ & 4 & 141 &  41 &  108 & 94 &  69\\
 \% & 100 & 67.8 & 82 & 65 & 80.3 & 57\\
 \tablevspace
No transfer & & & & & & \\
  $n$ & 0 & 67 & 9 & 59 & 23 & 52\\
 \% & 0 & 32.2 & 18 & 35 & 19.7 & 43\\
 \midrule
Total  & 4 & 208 & 50 & 167 & 117 & 121\\

\lspbottomrule
\end{tabular}
\end{table}

The information in \tabref{tab:sanchez7:6} is complemented by \tabref{tab:sanchez7:7}, which is to be read horizontally. Firstly, the ``transfer condition'' column shows the sum, percentage and mean occurrence of ILST per level of L2SP and OV structural property, calculated over the total number of empirically-relevant contexts (third column). Secondly, the ``no transfer condition'' shows the sum and mean of contexts where ILST did not occur. As the figures show, ILST in the L3 occurred much more frequently at low than at high levels of L2SP, which was true of the three structural properties. On the contrary, at high levels of L2SP the percentage of contexts where ILST could have occurred but did not is higher than at low proficiency levels.

\begin{table}
\caption{Distribution of empirically-relevant contexts across transfer conditions at different levels of L2SP\label{tab:sanchez7:7}}
\begin{tabular}{l *{2}{S[table-format=2.0]S[table-format=2.1]S[table-format=1.2]} S[table-format=3.0] }
\lsptoprule
& \multicolumn{3}{c}{Transfer condition} & \multicolumn{3}{c}{No transfer condition} & \\
\cmidrule(lr){2-4} \cmidrule(lr){5-7}
Structural property & {Raw} & {\%} & {M} & {Raw} & {\%} & {M} & {Total}\\
\midrule
SEP &&&&&&&\\
 Low & 7 & 77.8  & 1.75  & 2 & 22.2 & 0.5 & 9\\
 High & 214 & 46.9 & 1.16 & 273 & 53.1 & 1.32 & 514\\
  \tablevspace
VFINAL &&&&&&&\\
Low & 81 &  61.4 &  1.62 & 51 & 38.6 & 1.04 & 132\\
High & 179 &  42.3 & 1.07 & 244 & 57.7 & 1.46 & 423\\
 \tablevspace
INV & & & & & & & \\
 Low & 166 & 56.7 & 1.42 & 127 & 43.3 & 1.09 & 293\\
 High & 115 & 38.3 & 0.95 & 185 & 61.7 & 1.53 & 300\\
\lspbottomrule
\end{tabular}
\end{table}

Three series of ANCOVAs were run on the data, each using a different measure of L2SP, namely, one for each L2 structural property (SEP, VFINAL, INV). For all analyses, the cloze test scores that measured overall proficiency in the L3 English were used as a covariate, along with the age of the learners, which was added to the model as a second covariate. Not surprisingly, the ANCOVA testing the effect of L2SP in SEP was not statistically significant ($p = 0.809$), in all likelihood because of the ceiling effect in this independent variable. The ANCOVAs run using L2SP in the other two structural properties as independent variables turned out to be statistically significant. Specifically, there was a significant difference in the extent of ILST between different levels of L2SP whilst adjusting for L3 overall proficiency and age. This difference held for VFINAL ($F(1,213) = 4.047$, $p = 0.046$) and INV ($F(1,238) = 9.738$, $p = 0.002$).

\section{Discussion and conclusions}
\label{sec:sanchez:7}\largerpage

The research question guiding the study asked whether L2SP, operationalized in this study as the mastery of the OV structural properties in the source language of influence (L2 German), constrained the timing, extent, and type of ILST in the target language (L3 English). Based on the results, it may be confirmed that L2SP did constrain ILST in various ways. In addition to this, the analysis of analogous data in the L3 and the L2 (\citealt{Jarvis2000, FalkBardel2010}) constituted a methodological advancement and a novel aspect of the study. The following paragraphs discuss the relationship between L2SP at a domain-specific level in the source language of influence and ILST in the target language.

In view of the results reported in the preceding section, the main finding is that a high level of L2SP in the structural property transferred was not a prerequisite for ILST to occur (as will be discussed below). That said, it was also true that, at least in some cases, a given structure was a more likely candidate for ILST if the corresponding level of L2SP was high, as in the case of SEP. As the figures in \tabref{tab:sanchez7:3} show, virtually all participants exhibited a correct application of SEP in the L2 German (98.1\%), which suggests full mastery of the element transferred. The extent of ILST in the L3 was rather high, as suggested by the inspection of the means (mean: 1.16, \tabref{tab:sanchez7:5}). In contrast, the relationship between ILST and L2SP was different for VFINAL and INV, with the highest incidence of ILST being concentrated at low levels. In the case of VFINAL, this was true despite the somewhat unbalanced distribution of participants across low (23\%) and high (77\%) proficiency levels. Hence, the mean occurrence of ILST in the low level group was higher than in the high level group (1.62 vs. 1.07).

In the case of INV, the participants were more evenly distributed according to L2SP, with nearly half the sample in the high proficiency group (50.8\%) and the remaining 49.2\% in the low group. Here again, the mean occurrence of ILST in the L3 was higher for participants with a low level of L2SP (1.42 vs. 1.07 in learners with a higher L2SP). This finding lends further support to \citegen[211]{FalkBardel2010} claim that what is known “at a low proficiency stage of L2 is easy to transfer into an L3 of the same character”. It also supports claims that a high proficiency level in the L2 is not necessary for this L2 to become a source language of influence (\citealt{DeAngelis2007,Rast2010, SánchezBardel2017}), as anticipated in the opening of this section. To confirm the tendencies and patterns of ILST observed here, the design of future studies should target L3 learners with a lower level of L2SP in SEP and also L3 learners with a higher level in INV (not only in German, but also in other SOV languages). Such research would provide empirically pertinent contributions in order to obtain a deeper understanding of the interaction between L2 an L3 proficiency.

The inverse relationship between level of L2SP and ILST in the L3 portrayed here was further corroborated when crosstabulating the occurrence or non-oc\-cur\-rence of ILST at different levels of L2SP (\tabref{tab:sanchez7:6}) in these two structural properties (i.e., VFINAL and INV). Likewise, it was also found that a higher L2SP in these properties was associated with an increase in the number of contexts where ILST did not occur (\tabref{tab:sanchez7:7}). Conversely, at low levels of L2SP in VFINAL, most participants were found to transfer (82\%), whereas only a 18\% of them did not. In turn, the percentage of participants who transferred was lower at higher levels of L2SP (65\%). This tendency was nearly identical to the one of INV (\tabref{tab:sanchez7:6}). Here, at low levels of L2SP 80.3\% of participants transfered, against 19.7\% in the group of more proficient participants. In contrast, at higher levels barely 57\% of the participants transferred, against the 80.3\% of participants who transferred at low levels of L2SP. Of course, a 57\% is still a high percentage but it is definitely smaller than at low levels of L2SP, and as pointed out above, further research should shed more light on this.

In light of the results here, what can be suggested is that low L2SP seemed to favour activation and transfer from the L2, and that participants would have a hard time blocking, that is, inhibiting the unintended activation of a previous interlanguage (\citealt{Shanon1991, Dewaele2001, DeBot2004, Sánchez2011, SánchezBardel2017}). The results also point to the competition of structures from the L2 German interlanguage with structures from the L3 English interlanguage. In fact, the number of participants who transferred and the high extent of ILST at low levels of L2 development lend support to the suggestion that the weakest language (L3 English) was processed \citep{Abunuwara1992} via another non-native language (L2 German). Certainly, in some clauses in the L3 exhibiting ILST, the thematic verb was a lexical invention where lexemes and morphemes from the two interlanguages had been blended (ex. \textit{geeat} or \textit{willen}) in the formation of various verbal forms such as present and past participles \citep{Sánchez2015L2}. The co-activation of information at the lexeme and the lemma levels was not systematically explored in the present study, but it would be worth being explored in forthcoming studies that examine the co-occurrence of (interlanguage) transfer at the interface between different linguistic levels.

A critical question of the present study was the concurrent acquisition of both the source language of influence and the target language, as the two were being studied at school at the time the data were collected. Yet, the development in the two was asynchronic, and for reasons explained in the description of the participants, the development of the L3 English lagged behind that of the L2 German. What is important here is that this asynchrony had a great impact on the timing, extent and type of ILST \citep{MontrulEtAl2011}. The constraints imposed on ILST by L2SP in VFINAL and INV suggest a relationship between L2 and L3 development consistent with \citeapo{Cummins1979} developmental interdependence hypothesis. Thus, this interdependence would speak in favour of the extension of Cummins’ hypothesis to L3 learning and multiligualism (\citealt{Chumbow1981, Thomas1988, Abunuwara1992, Cenoz2000}). Furthermore, such a relationship is bolstered by the strong interlanguage connections (\citealt{Bartlet1989,DechertRaupach1989, Dechert2006, HallEtAl2009}) between the two interlanguages, both of which were still under development especially in relation to the structural properties investigated here.

 Likewise, because the L2 was itself still an interlanguage under development, the process of learning where to place the verb in the L2 (a process that involved the acquisition of the cluster of structural properties) slowed down the pace of L3 learning. Therefore, it would be reasonable to argue that ILST was manifested here also as a proficiency constraint in the form of a delay in restructuring in the L3 (especially considering also the finding that ILST in the L3 occurred, above all, at low levels of L2SP in these two structural properties). Morever, due to concurrent acquisition of the two non-native languages and because the two of them were still underdeveloped, the participants had to cope with incompatible surface orders in their L2 and their L3. Having to cope with these incompatibilities and unlearning the syntactic regularities they had acquired (or were in the process of acquiring) for another non-native language (the L2) exacerbated this delay.

A collateral effect of the asynchronic development of the L2 and the L3, and another L2 proficiency constraint on the occurrence of ILST, was that gains in the development of VFINAL and INV in the L2 seemed to trigger a parallel development in the equivalent structures in the L3 and a decrease in the incidence of ILST. From this it follows that gains in the L2 would enhance the learners’ sensitivity to notice the equivalent structures in the L3, in the same way as advances in the German of bilingual children sensitize them to corresponding structures in English (\citealt[29]{TracyGawlitzek-Maiwald2005}; also \citealt{Gawlitzek-Maiwald2001}). In this sense, the L2-L3 analogous data analyzed in this study indicate that once INV was acquired in the L2, the acquisition of verb placement in the L3 sped up as well. Furthermore, once INV was acquired in the L2, the extent of ILST was substantially reduced, as implied by the fact that the lowest incidence of ILST was found in data of participants at a high level of L2SP in INV. This was taken as evidence in favour of interpreting the acquisition of INV in the L2 German as a precursor or forerunner that accelerates the pace of acquisition of verb placement in L3 English. Thus, the evidence assembled apparently shows that development in the L3 lagged behind that of L2 German at least for a certain period of time that lasted until INV was acquired in the L2.

From a qualitative point of view, another constraint of the level L2SP on the occurrence of ILST in the data was the type of transfer. In previous studies it has been claimed that the occurrence of ILT is conditioned \citep{Bouvy2000}, in particular, by “the linguistic development reached in one language” (\citealt[115]{DeAngelis2007}) or the “steady state” of another interlanguage, as \citeauthor{Leung2003} puts it (2003: 199). This has important implications for our understanding of the kind of structures that could be transferred from the L2 to the L3 considered in the present study. On the one hand, nearly half the occurrences of ILST in the L3 involved SEP in main clauses (49.1\%), a structural property that the learners could already apply to main clauses in their L2 without any problem. On the other hand, around half of the other occurrences were in embedded clauses but they did not necessarily involve VFINAL, as would have been the case in the L2 German. Instead, while VFINAL was transferred in L3 embedded clauses with simple verb phrases (representing 28.1\% of the total amount of ILST occurrences), an overgeneralization of SEP (22.1\%) was found in virtually all embedded clauses with periphrastic constructions (62 out of 64). It is worth noting that the OV property overgeneralised the most by these L3 learners was SEP, of which they had full mastery in their L2 when used in main clauses. Equally, participants who were still at a low level of L2SP in VFINAL and still overgeneralized SEP in L2 embedded clauses with periphrastic constructions could only transfer this L2 developmental pattern to the L3. In other words, what these learners were transferring was a structural property that was still under development in their L2, and therefore at a higher risk of being influenced by another non-native language (\citealt{Groseva1998, Sánchez2011}). This lends further support to \citegen[241]{SánchezBardel2017} claim that interlanguage underdevelopment might be “the driving force behind transfer”. In addition to this, the comparison of L2-L3 analogous data revealed a clear intra-group congruity \citep{Jarvis2000} between the L2 and the L3 interlanguage performances, which made it possible to identify what it was in the L2 that motivated the L3 interlanguage behaviour.

{\sloppy\printbibliography[heading=subbibliography,notkeyword=this]}
\end{document}
